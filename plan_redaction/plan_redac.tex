\documentclass[10pt]{article}

\usepackage[utf8]{inputenc}
\usepackage[french]{babel}

\usepackage{enumitem}

\usepackage{tikz}
\def\checkmark{\tikz\fill[scale=0.4](0,.35) -- (.25,0) -- (1,.7) -- (.25,.15) -- cycle;} 


\begin{document}

\begin{center}
\Huge{\underline{Plan de Rédaction - Manuscrit de Thèse}}
\end{center}

Différents chapitres et avancées des rédactions :
\\

\begin{itemize}
\item[-] Introduction générale : à rédiger

\item[]

\item[I)] \textbf{\underline{Simulation d'Ecoulements Bouillants par CFD}}

\item[]

\item[1)] Présentation de NCFD : Quasi-complet (quelques détails à rajouter)

\item[2)] Présentation \& Analyse données exp. DEBORA : \checkmark

\item[3)] Calculs NCFD DEBORA : à terminer

\item[]

\item[II)] \textbf{\underline{Développement d'un nouveau modèle de partition du flux}}

\item[]

\item[4)] Biblio partition du flux : \checkmark quasi complet (traduction rapport biblio 1A faite)

\item[5)] Dynamique des bulles en paroi (forces, croissance, glissement, etc.) : \checkmark

\item[6)] Autres lois de fermeture (temps d'attente, densité site, interactions, aire de quenching, etc.) : \checkmark

\item[7)] Formulation finale et validation du modèle : à terminer (Figures OK)

\item[8)] Vers la prediction du CHF (courte biblio perspective pour évoquer des résultats récents et les qqs méthodes envisageables) : à rédiger

\item[]

\item[II)] \textbf{\underline{Vers la géométrie industrielle}}

\item[]

\item[9)] Présentation des essais promoteurs et résultats : à rédiger (Figures OK)

\item[10)] Simulations NCFD DEBORA Prom et AGATE Prom : à rédiger (Figures OK)

\item[]

\item[-] Conclusion Générale : à rédiger
\end{itemize}



\end{document}