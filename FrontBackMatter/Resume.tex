% Resume

%\renewcommand{\abstractname}{Abstract} % Uncomment to change the name of the abstract

\pdfbookmark[1]{Résumé}{Résumé} % Bookmark name visible in a PDF viewer

\begingroup
\let\clearpage\relax
\let\cleardoublepage\relax
\let\cleardoublepage\relax

\chapter*{Résumé}
Dans un Réacteur à Eau Pressurisée (REP), la chaleur dégagée par le combustible nucléaire est transférée à l’eau du circuit primaire, pressurisée à 150 bars pour éviter son ébullition. Cependant, en situation accidentelle, elle peut entrer en régime d’ébullition nucléée pouvant s’intensifier jusqu’à atteindre la crise d’ébullition. Ce point de transition quasi-instantané entre l’ébullition nucléée et l’ébullition en film entraîne la formation d’une couche de vapeur stable sur les crayons combustible, associée à une forte augmentation de leur température pariétale créant un risque de rupture de leur gaine. La prédiction du flux critique (flux de chaleur auquel se produit la crise d’ébullition) représente donc un enjeu de sûreté majeur et est actuellement réalisée à l’aide de corrélations expérimentales spécifiques à une configuration, n’incluant pas de représentation fine de la physique de l’ébullition. 

\npar


Cette thèse s’intéresse à la modélisation de la physique de l’ébullition à l’échelle locale dite « CFD » (Computational Fluid Dynamics), à laquelle il est possible de réaliser des simulations d’écoulements bouillants avec une discrétisation spatiale de l’ordre du millimètre. Le code maison NEPTUNE\_CFD, proposant une description eulerienne des écoulements multiphasiques à changement de phase, est l’outil de référence de EDF R&D pour investiguer ces problématiques aux échelles locales. 

\npar

Dans un premier temps, des simulations d’écoulements bouillants convectifs en tube vertical sont réalisées avec NEPTUNE\_CFD. Des comparaisons avec l’expérience DEBORA (écoulement bouillant de réfrigérant R12 en similitude REP sur plusieurs adimensionnels) ont permis une évaluation du code dans des conditions similaires au cas industriel. Les résultats obtenus sont globalement en accord avec l’expérience mais présentent des écarts notables sur le diamètre des bulles et la température paroi. Cette dernière est calculée au travers du modèle d’ébullition en paroi de NEPTUNE\_CFD dit à « Partition du Flux Pariétal » (Heat Flux Partitioning), où le flux appliqué est découpé entre plusieurs mécanismes de transfert de chaleur (convection, évaporation, conduction instationnaire, etc.). 

\npar

Le coeur des travaux de thèse a alors consisté en la construction d’un nouveau modèle de Partition du Flux, avec pour objectif une prise en compte plus fine de la phénoménologie de l’ébullition en considérant notamment le glissement des bulles. Une modélisation de la dynamique des bulles en paroi a été développée par une approche mécaniste décrivant les forces appliquées sur la bulle. Les formulations de certaines forces (masse ajoutée, traînée, etc.) ont été réévaluées et permettent une prédiction satisfaisante des diamètres de détachement et des vitesses de glissement à basse et haute pression. Le modèle de Partition du Flux a été complété par une évaluation des nombreuses lois de fermetures requises (temps d’attente, densité de sites de nucléation, etc.) par comparaison avec des mesures expérimentales tirées de la littérature. Le nouveau modèle ainsi développé a ensuite été validé par comparaison avec des mesures de température de paroi et implémenté dans NEPTUNE\_CFD.

\npar

La prédiction du flux critique s’ancre en perspective de ces développements. Des observations expérimentales récentes décrivent la crise d’ébullition à l’aide de paramètres physiques inclus dans le modèle de Partition du Flux. Un critère basé sur la proportion de surface occupée par les bulles a été testé avec l’ancien modèle de NEPTUNE\_CFD et semble proposer un comportement qualitativement cohérent.

\npar

Enfin, on s’intéresse à une configuration de type tube avec des ailettes de mélange similaires à celles présentes en coeur de REP. Les simulations NEPTUNE\_CFD montrent des écarts significatifs à l’expérience sur la prédiction du taux de vide à cœur. Des simulations monophasiques montrent une surestimation de la rotation du liquide, pouvant expliquer la trop grande accumulation de vapeur dans le cas bouillant.
\endgroup			

\npar

\npar


\textbf{\underline{mots-clés :}} changement de phase, transfert de chaleur, mécanique des fluides, écoulements multiphasiques, simulations numériques
\vfill