% Acknowledgements

\pdfbookmark[1]{Acknowledgements / Remerciements}{Acknowledgements / Remerciements} % Bookmark name visible in a PDF viewer

\begin{flushleft}{\slshape    
Great success !} \\ \medskip
--- \textbf{Borat Sagdiyev}
\end{flushleft}
\begin{flushright}{\slshape    
Science sans conscience n'est que ruine de l'âme.} \\ \medskip
--- \textbf{Fran\c{c}ois Rabelais}
\end{flushright}

\npar




\bigskip

%----------------------------------------------------------------------------------------

\begingroup

\let\clearpage\relax
\let\cleardoublepage\relax
\let\cleardoublepage\relax

\chapter*{Acknowledgements / Remerciements}

Après trois années de doctorat, je me demande si la section des remerciements n'est pas parmi celles auxquelles j'ai le plus réfléchi. Chaque moment passé avec des proches ou petite pensée nostalgique n'incite qu'à ajouter une ligne pour quelqu'un. Cette thèse ne fera donc pas exception : préparez-vous pour des remerciements à rallonge ! 

\noindent Put your acknowledgements here.\\

\noindent Many thanks to everybody who already sent me a postcard!\\

\noindent Regarding the typography and other help, many thanks go to Marco Kuhlmann, Philipp Lehman, Lothar Schlesier, Jim Young, Lorenzo Pantieri and Enrico Gregorio\footnote{Members of GuIT (Gruppo Italiano Utilizzatori di \TeX\ e \LaTeX )}, J\"org Sommer, Joachim K\"ostler, Daniel Gottschlag, Denis Aydin, Paride Legovini, Steffen Prochnow, Nicolas Repp, Hinrich Harms, Roland Winkler, and the whole \LaTeX-community for support, ideas and some great software.

\bigskip

\noindent\emph{Regarding \mLyX}: The \mLyX\ port was initially done by
\emph{Nicholas Mariette} in March 2009 and continued by
\emph{Ivo Pletikosi\'c} in 2011. Thank you very much for your work and the contributions to the original style.

\endgroup