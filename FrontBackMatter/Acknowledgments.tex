% Acknowledgements

\pdfbookmark[1]{Acknowledgements / Remerciements}{Acknowledgements / Remerciements} % Bookmark name visible in a PDF viewer

\begin{flushleft}{\slshape    
Great success !} \\ \medskip
--- \textbf{Borat Sagdiyev}
\end{flushleft}
\begin{flushright}{\slshape    
Science sans conscience n'est que ruine de l'âme.} \\ \medskip
--- \textbf{Fran\c{c}ois Rabelais}
\end{flushright}

\npar




\bigskip

%----------------------------------------------------------------------------------------

\begingroup

\let\clearpage\relax
\let\cleardoublepage\relax
\let\cleardoublepage\relax

\chapter*{Acknowledgements / Remerciements}

Après trois années de doctorat, je me demande si la section des remerciements n'est pas parmi celles auxquelles j'ai le plus réfléchi. Chaque moment passé avec des proches ou petite pensée nostalgique n'incite qu'à ajouter une ligne pour quelqu'un. Cette thèse ne fera donc pas exception : préparez-vous pour des remerciements à rallonge ! 


\npar

First of all, I want to deeply thank Emilio Baglietto and Dirk Lucas for having accepted to read and evaluate my thesis work as referees. I truly enjoyed the time we spent together before and after the defense and it was a true honor to have both of you as members of the jury. 

Merci à Michel Gradeck d'avoir accepté le rôle de président du jury ainsi qu'à Guillaume Bois pour ses questions pertinentes le jour de la soutenance. De manière générale : Merci à tous les membres du jury pour votre bienveillance et l'intérêt que vous avez porté à mes travaux !

\npar

Ensuite, mes premières penseés vont évidemment à mes trois encadrants avec qui j'ai eu la chance de travailler pendant ces trois années : Catherine, Stéphane P. et Stéphane M. Leurs qualités tant scientifiques qu'humaines ont clairement été les piliers de ces travaux de doctorat.

\npar
Catherine, il va sans dire que cette thèse n'aurait pas connu cet aboutissement sans ton expérience et ton regard affuté sur la physique des écoulements multiphasiques et des changements de phase. En plus d'être la "spécialiste de bulles" (comme j'aimais le dire à mes proches), ta gentillesse, ton humanité et ta présence ont été d'une aide inestimable et je t'en suis profondément reconnaissant. T'avoir comme directrice de thèse fut une réelle chance et est sans aucun doute une des principales raisons m'ayant donné l'envie de m'essayer à une carrière académique ! J'espère sincèrement pouvoir à nouveau échanger et travailler avec toi à l'avenir.

\npar

Stéphane P., avec qui je ne compte plus les discussions captivantes tant autour du nucléaire que de la science en général ou encore de politique depuis que je suis arrivé en stage dans le groupe I8C.  Ton soutien permanent durant ces trois années, la liberté d'action que tu m'as laissée ainsi que tes remarques et ton expertise d'une pertinence rare ont su me laisser appréhender le travail de recherche dans les meilleures conditions possibles. En ajoutant à cela tous les très bon moments passés ensemble que ça soit de la salle café jusque dans les rues de Los Angeles, tu es à la fois un collègue, un tuteur et un ami. Merci infiniment pour tout.

\npar

Stéphane M., dont l'expérience en tant qu'encadrant de thèse n'est plus à prouver, merci pour ta sympathie naturelle, ton positivisme à toute épreuve et ta curiosité permanente. Avoir une personne intéressée et bienveillante comme toi fut une réelle source de motivation pendant ces trois années. Merci pour tout !

\npar

Merci beaucoup à Erwan pour son aide et son intérêt constant pour mes travaux, en particulier sur le cas AGATE sur lequel nous avons beaucoup échangé ainsi que sur les subtilités de \textit{code\_saturne} !

\npar

Merci sincèrement à Damien pour sa disponibilité, sa sympathie, son écoute et son aide. Je pense qu'un doctorant ne peut guère rêver meilleur chef de groupe !

\npar

Un merci tout particulier à Vladimir sans qui les dernières figures de ce manuscrit n'auraient jamais vu le jour, merci infiniment d'avoir rendu possible le pont entre l'aspect académique et industriel appliqué de ces travaux de thèse !

\npar

Il va sans dire que la vie à Chatou ne serait pas la même sans toutes les personnes qui en font un lieu de travail aussi agréable ! En commençant pas le groupe I8C dont l'ambiance me manquera à coup sûr. Des mercis à foison à:

\begin{itemize}
\item Morgane, la passionnée de jeux de sociétés et tout aussi fan  d'Harry Potter que des générateurs de vapeur ;

\item Ueva, le thaïtien aussi bon en développement de code qu'en lever de coude, sans qui la formation à Zürich n'aurait pas eu la même saveur ;

\item Christophe, sans qui je n'aurais jamais vu une crise d'ébullition de mes propres yeux. Merci pour ton intérêt et ton efficacité sans faille sur la FIL ! Promis on finira ça avec un film à la caméra rapide ;)

\item Olivier, dont la sympathie et la bonne humeur n'ont d'égal que son efficacité comme coach à la salle de sport ;

\item Igor, 

\item PV, 

\item Franck, avec qui les discussions politiques sont tout aussi houleuses qu'intéressantes !

\item Eric, dont l'expertise 

\item Mugurel, le spécialiste des turbines et combattant de haut vol

\item Antoine, 

\item Muriel

\item Pierre

\item Jacques

\item Martin

%\item Gauthier, la relève des doctorants I8C

\item André

\item Pauline

\item Franck D

\item Thomas

\item Qingqing

\item Sébastien

\item Pierre (IMFT ?)
\end{itemize}

Autres MFEE:

\begin{itemize}
\item Nasser, qui a transformé mon été de 3e année en 
\item Chaï
\item William
\item Aurélien
\item Joël
\item Mickael
\end{itemize}

Doctorants:

\begin{itemize}
\item Bilal
\item Hector
\item Jacques
\item Martin
\item Gauthier
\item Guillaume
\item Antoine
\item Roger
\item Gaëtan
\item Clément
\item Théo
\item Sami (PRISME)
\item Elisa
\item Bastien
\item Guillaume
\item Jean-Paul
\end{itemize}



Les potes EDF : Hector, Jacques, Martin, Gauthier, Elisa, Antoine, Guillaume, etc.





Cette thèse n'aurait jamais vu le jour sans ceux qui m'ont donné l'opportunité de faire mon stage de fin d'étude à MFEE en 2019 : Jérémie et Romain. Merci à tous les deux pour m'avoir initié au milieu de la recherche appliquée au nucléaire, pour votre soutien et pour vos enseignements qui m'ont grandement servi ces trois années !

\npar

Anciens :

\begin{itemize}
\item Simon
\item Enrico
\item Joël
\end{itemize}


Un grand merci aux personnes de l'IMFT avec qui j'ai pu discuter science et recherche lors de mes quelques passages, en particulier Frédéric et Julien ! 

\npar

Merci aux personnes du CEA avec qui j'ai pu avoir des échanges durant cette thèse : Tanguy, Sébastien, Corentin, Antoine et Alan.

\npar




Ce doctorat est l'aboutissement d'une vie d'étudiant, et ne saurait exister sans l'ensemble des professeurs et enseignants dont j'ai eu la chance de recevoir les enseignements et qui ont construit ma passion pour la science. Un très grand merci (chronologiquement) à : Abdelaziz Benzidia, Jean-François Matte, Charles Vix, Gaëlle Mulard, Antoine Senger, Pascal Guelfi et Emmanuel Plaut. 

\npar

Un merci tout particulier à (camarade) Rainier et Jean-François, qui m'ont mis le pied à l'étrier pour la recherche et la CFD pendant le projet de recherche en deuxième année d'école d'ingénieur. J'ai déjà hâte de notre prochaine bière ensemble à Nancy !


\npar

Merci à ma team Peinkess, Mehdess et Messmer, qui en plus d'être des frères avec qui il est tout aussi passionnant de parler science et politique que de clubber chez Kalle Malle, ont fait le déplacement depuis Liverpool et Oldenburg pour assister à ma soutenance. J'ai maintenant deux voyages de prévus pour aller assister aux vôtres en Y.

\npar

Une grosse pensée va pour les potes d'école, dont le soutien et l'intérêt pour mes bulles les a poussés jusqu'à écouter toute ma soutenance ! Merci Chloé (la fraté), Cécilia (la soeur de chicha), Micess (le ienchoss), Kot (mon gros dégueulasse), le B (frère de manif et de soirée),  Solti et Nico (que je démonte sur Smash à une main), Bucquet (envi de jammer), Iris (mon aubergiste toulousaine), Martin M. (le parrain sûr) et Martin P (le sosie de Robespierre)


\begin{itemize}
\item Potes d'école de manière générale (diane, martin, tous ceux venus à la soirée)
\item Team coloc : Rody, Younes, Marouane, Alae
\item Justine
\item Clara
\end{itemize}


Le FC Chômage:

\begin{itemize}
\item Burno
\item Suzie
\item Zooky
\item Loutre
\item Beaugosse
\item Bouboule
\item Lulu
\item Anto
\end{itemize}

mif:

\begin{itemize}
\item Suzanne
\item Pierrot
\item Fanfan
\item Papa
\item Maman
\item Elina
\end{itemize}



Louise
\endgroup