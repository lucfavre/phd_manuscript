% Acknowledgements

\pdfbookmark[1]{Acknowledgements / Remerciements}{Acknowledgements / Remerciements} % Bookmark name visible in a PDF viewer

\vspace*{4cm}


\begin{flushleft}{\slshape    
Great success !} \\ \medskip
--- \textbf{Borat Sagdiyev}
\end{flushleft}

\vspace*{2cm}

\begin{flushright}{\slshape    
Science sans conscience n'est que ruine de l'âme.} \\ \medskip
--- \textbf{Fran\c{c}ois Rabelais}
\end{flushright}

\npar




\bigskip

\clearpage
%----------------------------------------------------------------------------------------

\begingroup

\let\clearpage\relax
\let\cleardoublepage\relax
\let\cleardoublepage\relax

\chapter*{Acknowledgements / Remerciements}

Après trois années de doctorat, je me demande si la section des remerciements n'est pas parmi celles auxquelles j'ai le plus réfléchi. Chaque moment passé avec des proches ou petite pensée nostalgique n'incite qu'à ajouter une ligne pour quelqu'un. Cette thèse ne fera donc pas exception : préparez-vous pour des remerciements à rallonge ! 


\npar

First of all, I want to deeply thank Emilio Baglietto and Dirk Lucas for having accepted to read and evaluate my thesis work as referees. I truly enjoyed the time we spent together before and after the defense and it was a true honor to have both of you as members of the jury. 

Merci à Michel Gradeck d'avoir accepté le rôle de président du jury ainsi qu'à Guillaume Bois pour ses questions pertinentes le jour de la soutenance. De manière générale : Merci à tous les membres du jury pour votre bienveillance et l'intérêt que vous avez porté à mes travaux !

\npar

Ensuite, mes premières penseés vont évidemment à mes trois encadrants avec qui j'ai eu la chance de travailler pendant ces trois années : Catherine, Stéphane P. et Stéphane M. Leurs qualités tant scientifiques qu'humaines ont clairement été les piliers de ces travaux de doctorat.

\npar
Catherine, il va sans dire que cette thèse n'aurait pas connu cet aboutissement sans ton expérience et ton regard affuté sur la physique des écoulements multiphasiques et des changements de phase. En plus d'être la "spécialiste de bulles" (comme j'aimais le dire à mes proches), ta gentillesse, ton humanité et ta présence ont été d'une aide inestimable et je t'en suis profondément reconnaissant. T'avoir comme directrice de thèse fut une réelle chance et est sans aucun doute une des principales raisons m'ayant donné l'envie de m'essayer à une carrière académique ! J'espère sincèrement pouvoir à nouveau échanger et travailler avec toi à l'avenir.

\npar

Stéphane P., avec qui je ne compte plus les discussions captivantes tant autour du nucléaire que de la science en général ou encore de politique depuis que je suis arrivé en stage dans le groupe I8C.  Ton soutien permanent durant ces trois années, la liberté d'action que tu m'as laissée ainsi que tes remarques et ton expertise d'une pertinence rare ont su me laisser appréhender le travail de recherche dans les meilleures conditions possibles. En ajoutant à cela tous les très bon moments passés ensemble que ça soit de la salle café jusque dans les rues de Los Angeles, tu es à la fois un collègue, un tuteur et un ami. Merci infiniment, pour tout.

\npar

Stéphane M., dont l'expérience en tant qu'encadrant de thèse n'est plus à prouver, merci pour ta sympathie naturelle, ton positivisme à toute épreuve et ta curiosité permanente. Avoir une personne intéressée et bienveillante comme toi fut une réelle source de motivation pendant ces trois années. Merci !

\npar

Un grand merci tout particulier pour leur soutien à :

\begin{itemize}

\item Erwan pour son aide et son intérêt constant pour mes travaux, en particulier sur le cas AGATE sur lequel nous avons beaucoup échangé ainsi que sur les subtilités de \textit{code\_saturne} !

\item Damien pour sa disponibilité, sa sympathie, son écoute et son aide. Je pense qu'un doctorant ne peut guère rêver meilleur chef de groupe !

\item Vladimir sans qui les dernières figures de ce manuscrit n'auraient jamais vu le jour, merci infiniment d'avoir rendu possible le pont entre l'aspect académique et industriel appliqué de ces travaux de thèse !

\item Christophe, sans qui je n'aurais jamais vu une crise d'ébullition de mes propres yeux. Merci pour ton intérêt et ton efficacité sans faille sur la FIL ! Promis on finira ça avec un film à la caméra rapide ;)

\end{itemize}

\npar

Il va sans dire que la vie à Chatou ne serait pas la même sans toutes les personnes qui en font un lieu de travail aussi agréable ! En commençant pas le groupe I8C dont l'ambiance me manquera à coup sûr. Des mercis à foison à : Morgane (promis je te débloquerai sur Zelda un jour), Ueva (sans qui la formation à Zürich n'aurait pas eu la même saveur), Olivier, PV, Igor, Franck M., Eric, Mugurel, Antoine, Muriel, Pierre, André, Pauline, Franck D., Thomas, Qinqing et Sébastien.

\npar

Une pensée aussi aux ex-I8C partis vers de nouveaux horizons depuis : Simon, Enrico et Joël !

\npar

Un grand merci à tous les autres MFEE avec qui j'ai pu échanger et passer des bons moments, je pense en particulier à Nasser (avec qui passer l'été de 3\up{e} année aux pauses cafés du soir a tout autant retardé mon manuscrit que sauvé ma santé mentale), Chaï (pour son aide sur NEPTUNE\_CD), William (pour ses retours sur mes travaux et présentations), Aurélien, Joël et Michaël.

\npar

Pour les nombreuses bières, partages de galères et soutiens indéfectibles, un immense merci à mes co-doctorants I8C:

\begin{itemize}
\item Jacques, co-bureau éternel et spécialiste hardware informatique de référence. Vivre les mêmes enfers avec NEPTUNE\_CFD nous a lié à tout jamais je pense... J'espère qu'on pourra se refaire une soirée comme chez toi à coups de discussions jusqu'au bout de la nuit !

\item Martin, le seul capable de cumuler une pratique surhumaine du sport en parallèle d'un doctorat en bataillant avec les modes de Graetz, tout en étant sûrement la personne la plus gentille de toute l'île des impressionnistes !

\item Gauthier, la relève qui va rendre THYC plus rapide que jamais  (entre deux parties de jeux de société). 

\item Pierre, avec qui j'ai déjà hâte de boire un coup à nouveau à Toulouse pour débattre science, potins et politique !
\end{itemize}

\npar

Parmi tous les doctorants, il y en a qui resteront définitivement des amis bien plus que des collègues:

\begin{itemize}
\item Bilal, le frère, le sat depuis le stage de fin d'études. Merci d'avoir été là toutes ces années et d'être encore là aujourd'hui. J'ai déjà hâte de notre prochaine cohabitation en appart à parler politique ma3a jwane et à déconner ensemble.

\item Hector, le meilleur partenaire de soirées, jeux, sport, bar, déconnades, et bien sûr CFD de toute l'île des Impressionnistes. En plus d'être un DJ d'exception et un scientifique de haut vol, tu es juste un pilier dans cette histoire donc merci bae. Je sens que cette prochaine année et demie de post-doc à Chatou promet encore de régaler ! (envi de coder)

\item Elisa, la fraté toulousaine toujours al à la salle et camarade indéfectible de jeux de rôles. Je suis sûr que tu vas nous sortir des PIV de l'espace pour ta thèse que tu finiras en grandes pompes ! 
\end{itemize}

\npar

Merci à tous les autres doctorants du site, qu'ils soient MFEE (Guilhem, Guillaume, Antoine, Roger, Gaëtan, Théo et Clément) ou LNHE (Guillaume, Bastien et JP) pour tous les bons moments passés ensemble !


\npar 

Cette thèse n'aurait jamais vu le jour sans ceux qui m'ont donné l'opportunité de faire mon stage de fin d'étude à MFEE en 2019 : Jérémie et Romain. Merci à tous les deux pour m'avoir initié au milieu de la recherche appliquée au nucléaire, pour votre soutien et pour vos enseignements qui m'ont grandement servi ces trois années !


\npar

Un grand merci aux personnes de l'IMFT avec qui j'ai pu discuter science et recherche lors de mes quelques passages, en particulier Frédéric et Julien ! 

\npar

Merci aux personnes du CEA avec qui j'ai pu avoir des discussions durant cette thèse et avec qui j'espère continuer à échanger à l'avenir : Tanguy, Sébastien, Corentin, Antoine et Alan. 

\npar

Merci à Frédéric Le Quéré et Frédéric Praslon de l'Université Gustave Eiffel pour m'avoir permis d'enseigner un semestre en mathématiques pour la L1 PC/SPI. Cette expérience fut une des plus enrichissantes que j'aie pu vivre et m'a définitivement convaincu de me tourner vers le domaine académique et l'enseignement.

\npar


Ce doctorat est l'aboutissement d'une vie d'étudiant, et ne saurait exister sans l'ensemble des professeur.e.s dont j'ai eu la chance de recevoir les enseignements et qui ont construit ma passion pour la science. Un très grand merci (chronologiquement) à : Abdelaziz Benzidia, Jean-François Matte, Charles Vix, Gaëlle Mulard, Antoine Senger, Pascal Guelfi et Emmanuel Plaut. 

\npar

Un merci tout particulier à (camarade) Rainier et Jean-François, qui m'ont mis le pied à l'étrier pour la recherche et la CFD pendant le projet de recherche en deuxième année d'école d'ingénieur. J'ai déjà hâte de notre prochaine bière ensemble à Nancy !


\npar

\npar

Merci à ma team Peinkess, Mehdi et Messmer, qui en plus d'être des frères avec qui il est tout aussi passionnant de parler science et politique que de clubber chez Kalle Malle, ont fait le déplacement depuis Liverpool et Oldenburg pour assister à ma soutenance. J'ai maintenant deux voyages de prévus pour aller assister aux vôtres en Y. On l'écrira un jour cet article en commun !

\npar

Une grosse pensée va aux amis de l'école, dont le soutien et l'intérêt pour mes bulles les ont poussés jusqu'à écouter toute ma soutenance ! Merci à Chloé (la fraté), Cécilia (la soeur de chicha), Micess (le ienchoss), Kot (mon gros dégueulasse), le B (frère de manif),  Solti (que je démonte sur smash à une main), Bucquet (envi de jammer), Iris (mon aubergiste toulousaine), Martin M. (le parrain sûr) et Martin P. (le sosie de Robespierre).

\npar

Un grand merci à tous les potes qui ont été présents de près ou de loin toutes ces années : 

\begin{itemize}
\item La bande d'école d'ingé : Keller, Diane, Nico (L. et P.), Juliette, Théophile et tous les autres de cette grande bande de fous !

\item Les sats et colocs à vie Younes, Rody, Marouane, Alae et Hirvin

\item La team Bilnancy : Catherine, Bousti, Farès, Mathieu, Thomas, Nadir et Lothaire

\item L'ékip ta7 Boudonville et l'école primaire : Justine et Clara

\item La mi,f les meufs sûres, les amours : Maru et Marion

\end{itemize}

\npar

\'Evidemment une vague d'amour et d'olives ininterrompues pour mon FC Chômage à tout jamais : Burno et Suzie (avec qui la coloc pendant cette thèse fut tout autant une source de folie, de motivation et d'amour), Zooky (fan number one de farine de blé), Jean-Loutre (mon antéchrist malchanceux), Lulu (survivant aux déboires de Ferrari et des Canadiens), Sydney (soleil de beau gosse qui a embelli mon séjour aux US), Bouboule (pano poivrot) et Anto (la pornstar).

\npar

Merci à Clo (Swagbaiouch à jamais) pour la force et le soutien depuis Marseille bb. 

\npar

Merci à Myriam, camarade de lutte, pour le plus beau cadeau de thèse : le décapsuleur CGT.

\npar

\npar

Merci à Dany et Raz pour leurs lives Twitch réguliers qui m'ont permis de tenir bon pendant la rédaction. Le cocktail d'émotions mêlant concentration, fatigue, politisation, explosions de rire et avis tout autant basés que désastreux fut plus efficace que tout pour me donner la force. (désolé d'aspirer à devenir prof.........)

\npar

\npar

Comment ne pas finir par la famille ! Un immense merci à Suzanne, Pierrot et Fanfan pour être venus depuis la Bourgogne et la Franche-Comté pour venir m'écouter parler de mes bulles en anglais ! 

\npar

Et évidemment, un merci infini et éternel à mes parents. Pour ce que vous êtes, pour ce que vous avez fait de moi, pour votre présence, vos encouragements, votre confiance et votre présence qui continuent aujourd'hui d'être ma chance et mon plus grand soutien. 

Merci papa pour m'avoir mis les pieds sur le chemin de la science, m'avoir tant appris, pour m'y avoir laissé faire mon parcours sans jamais me l'imposer, pour m'avoir soutenu, encouragé et conseillé dès que j'en avais besoin. Pour toutes ces heures passées à discuter en mêlant conneries, recherche, science, et la vie dans son entièreté : sans elles, je ne serais pas devenu la personne que je suis aujourd'hui.

Merci maman pour ton écoute, ton intérêt et ton attention constante tout en me laissant une liberté totale dans mes choix. Toujours prête à venir aider au moindre pépin, au moindre changement de situation : sans toi, ma vie ne serait clairement pas aussi sereine et agréable qu'elle ne l'est depuis toutes ces années. Tu as fait tout ça en supportant mes blagues répétées et insupportables, une maman comme on en fait pas deux ! 

\npar

Merci à Elina, ma s\oe ur, pour les innombrables fous rires (et disputes!) qu'on a pu avoir ensemble, pour nos discussions passionnées de politique, pour sa présence, son humour et son soutien. Tu es une petite s\oe ur par l'âge mais presque une grande s\oe ur par ton esprit et ton intelligence. Hâte de te voir devenir une juriste de génie zebi ! 

\npar

Cette famille est indescriptible de par sa perfection, donc merci à mes trois gros pour tous ces moments chalereux, amusants, sérieux, touchants et passionnants que l'on a vécus tous les quatre (et pour tous ceux à venir).

\npar

\npar


Enfin, merci à celle qui m'a soutenu tout du long de cette thèse, à celle qui m'a vu dans mes meilleurs comme mes pires moments et qui a toujours su trouver les mots pour me préserver et me motiver. Merci Louise pour tout l'amour que tu me donnes et pour me faire vivre des moments aussi incroyables avec toi. Maintenant c'est à toi d'aller au bout de ta thèse qu'on célébrera comme il se doit dans un an et demi !
\endgroup