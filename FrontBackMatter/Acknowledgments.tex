% Acknowledgements

\pdfbookmark[1]{Acknowledgements / Remerciements}{Acknowledgements / Remerciements} % Bookmark name visible in a PDF viewer

\begin{flushleft}{\slshape    
Great success !} \\ \medskip
--- \textbf{Borat Sagdiyev}
\end{flushleft}
\begin{flushright}{\slshape    
Science sans conscience n'est que ruine de l'âme.} \\ \medskip
--- \textbf{Fran\c{c}ois Rabelais}
\end{flushright}

\npar




\bigskip

%----------------------------------------------------------------------------------------

\begingroup

\let\clearpage\relax
\let\cleardoublepage\relax
\let\cleardoublepage\relax

\chapter*{Acknowledgements / Remerciements}

Après trois années de doctorat, je me demande si la section des remerciements n'est pas parmi celles auxquelles j'ai le plus réfléchi. Chaque moment passé avec des proches ou petite pensée nostalgique n'incite qu'à ajouter une ligne pour quelqu'un. Cette thèse ne fera donc pas exception : préparez-vous pour des remerciements à rallonge ! 

\npar

Mes premières penseés vont évidemment à mes trois encadrants avec qui j'ai eu la chance de travailler pendant ces trois années : Catherine, Stéphane P. et Stéphane M. Leurs qualités tant scientifiques qu'humaines ont clairement été les piliers de ces travaux de doctorat.

\npar
Catherine, il va sans dire que cette thèse n'aurait pas connu cet aboutissement sans ton expérience et ton regard affuté sur la physique des écoulements multiphasiques et des changements de phase. En plus d'être la "spécialiste de bulles" (comme j'aimais le dire à mes proches), ta gentillesse, ton humanité et ta présence ont été d'une aide inestimable et je t'en suis profondément reconnaissant. T'avoir comme directrice de thèse fut une réelle chance et est sans aucun doune une des principales raisons m'ayant donné l'envie de m'essayer à une carrière académique ! J'espère sincèrement pouvoir à nouveau échanger et travailler avec toi à l'avenir.

\npar

Stéphane P., avec qui je ne compte plus les discussions captivantes tant autour du nucléaire que de la science en général ou encore de politique depuis que je suis arrivé en stage dans le groupe I8C.  Ton soutien permanent durant ces trois années, la liberté d'action que tu m'as laissée ainsi que tes remarques et ton expertise d'une pertinence rare ont su me laisser appréhender le travail de recherche dans les meilleures conditions possibles. En ajoutant à cela tous les très bon moments passés ensemble que ça soit de la salle café jusque dans les rues de Los Angeles, tu es à la fois un collègue, un tuteur et un ami aux multiples qualités. Merci infiniment pour tout.

Stéphane M., 



\noindent Put your acknowledgements here.\\

\noindent Many thanks to everybody who already sent me a postcard!\\

\noindent Regarding the typography and other help, many thanks go to Marco Kuhlmann, Philipp Lehman, Lothar Schlesier, Jim Young, Lorenzo Pantieri and Enrico Gregorio\footnote{Members of GuIT (Gruppo Italiano Utilizzatori di \TeX\ e \LaTeX )}, J\"org Sommer, Joachim K\"ostler, Daniel Gottschlag, Denis Aydin, Paride Legovini, Steffen Prochnow, Nicolas Repp, Hinrich Harms, Roland Winkler, and the whole \LaTeX-community for support, ideas and some great software.

\bigskip

\noindent\emph{Regarding \mLyX}: The \mLyX\ port was initially done by
\emph{Nicholas Mariette} in March 2009 and continued by
\emph{Ivo Pletikosi\'c} in 2011. Thank you very much for your work and the contributions to the original style.

\endgroup