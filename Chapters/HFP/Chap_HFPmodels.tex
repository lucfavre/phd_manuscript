% Chapter X

\chapter{Existing Heat Flux Partitioning Models} % Chapter title

\label{ch:name} % For referencing the chapter elsewhere, use \autoref{ch:name} 

%----------------------------------------------------------------------------------------

\section{Kurul \& Podowski (1990)}


In their original work published in 1990, Kurul \& Podowski proposed a first complete closure for the wall heat flux partition.


\npar
They considered the applied heat flux to be divided between three mechanisms :

\begin{itemize}
\item A liquid single-phase heat flux $\phi_{c,l}$ ;
\item A boiling heat flux $\phi_{e}$ to represent phase change from liquid to vapor ;
\item A quenching heat flux $\phi_{q}$ to represent the effect of a bubble leaving the wall and being replaced by cold liquid.
\end{itemize}

The total wall heat flux being thus expressed as :

\begin{align}
\phi_{w}=\phi_{c,l}+\phi_{e}+\phi_{q}
\end{align}

Each flux being expressed as follows : 

\begin{align}
\phi_{c,l}=A_{c,l} \rho_{l} c_{p,l}U_{l,\delta} \St_{l,\delta}\parth{T_{w}-T_{l,\delta}}\\
\label{eq:phie_KP}
\phi_{e}=\frac{1}{6}\pi {D_{b}}^{3}\rho_{v}h_{lv}fN_{sit}\\
\phi_{q}=t_{q}fA_{q}\frac{2\lambda_{l}\parth{T_{w}-T_{l,\delta}}}{\sqrt{\pi \eta_{l} t_{q}}}
\end{align}

%------------------------------------------------

\subsection{Basu (2000)}

Content

%------------------------------------------------

\subsection{Yeoh (2006)}

Content

%----------------------------------------------------------------------------------------

\section{Gilman (2017)}

Content

\section{Kommajosyula (2020)}

