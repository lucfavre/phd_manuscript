% Chapter X

\chapter{Introduction to Heat Flux Partitioning} % Chapter title

\label{chap:HFP_bib} % For referencing the chapter elsewhere, use \autoref{ch:name} 

\minitoc

%----------------------------------------------------------------------------------------
\section{Introduction}

\subsection{Empirical Approaches}

In the field of heat transfer and boiling, establishing a physical relationship between the applied heat flux at the wall $\phi_{w}$ and the wall temperature $T_{w}$ has always been a primal goal for system design and safety analysis. Traditional approaches for single-phase flows rely on the estimation of the heat transfer coefficient $h_{SP}$ defined as:

\begin{equation}
\phi_{w} = h_{SP} \parth{T_{w} - T_{L}}
\end{equation}
where $T_{w}$ and $T_{L}$ are the wall and bulk liquid temperature. 

\npar
Estimating $h_{SP}$ is usually done using dedicated correlations depending on flow parameters such as the hydraulic Reynolds number $\Re_{D_{h}}$, the fluid Prandtl number $\Pr_{L}$, etc. Correlations of Dittus-Boelter (Eq \ref{eq:dittus}) or Gnielinski (Eq. \ref{eq:gnielinski}) that were presented earlier are typical example of widely used expressions to estimate $h_{SP}$.

\npar

For boiling multiphase flows, different types of empirical approaches have been developed through the history. For instance, some authors proposed direct correlations relating wall heat flux and temperature such as:

\begin{itemize}
\item Jens \& Lottes \cite{jens_analysis_1951}:

\begin{equation}
\phi_{w} = \parth{ \frac{\Delta T_{w} }{25} e^{P / 62}	}^{4}
\label{eq:jens_correl}
\end{equation}

\item Thom \etal \cite{thom_boiling_1967}:
\begin{equation}
\phi_{w} = \parth{ \frac{\Delta T_{w} }{22.65} e^{P / 87}	}^{2}
\label{eq:thom_correl}
\end{equation}
\end{itemize} 
with $P$ in bar and $\phi_{w}$ in MW/m\up{2}.

\npar

On the other hand, some models are based on the estimation of heat transfer coefficients similar to single-phase approaches. They mostly rely on the separation between a single phase $h_{SP}$ and nucleate boiling $h_{NB}$ heat transfer coefficients. Well-known correlations of this type are proposed by Chen \cite{chen_correlation_1966}, Gungor \& Winterton \cite{gungor_general_1986} or Liu \& Winterton \cite{liu_general_1991}. For instance, Chen correlation writes:

\begin{equation}
\phi_{w} = \phi_{SP} + \phi_{NB} = h_{SP}\parth{T_{w}-T_{L}}F + h_{NB}\parth{T_{w}-T_{sat}}S
\end{equation}
with $F$ an amplification factor for the single phase heat transfer due to bubble agitation and $S$ a suppression term hindering the nucleate boiling under the effect of bulk liquid flow.

\npar
The single phase part $h_{SP}$ is calculated using Dittus-Boelter \cite{dittus_heat_1985} formulation (Eq. \ref{eq:dittus}) while the nucleate boiling part is:

\begin{equation}
h_{NB} = 0.00122 \frac{\lambda_{L}^{0.79} c_{p,L}^{0.45} \rho_{L}^{0.49}}{\sigma^{0.5} \mu_{L}^{0.29} h_{LV}^{0.24}\rho_{V}^{0.24}} \Delta T_{w}^{0.25} \parth{P_{sat}\parth{T_{w}} - P_{sat}\parth{T_{sat}}}^{0.75}
\end{equation}
along with the factors $F$ and $S$:

\begin{align}
\frac{1}{X_{tt}} & = \parth{\frac{x}{1-x}}^{0.9} \parth{\frac{\rho_{L}}{\rho_{V}}}^{0.5} \parth{\frac{\mu_{V}}{\mu_{L}}}^{0.1} \\
F &= 
\begin{cases}
1\ &\text{if}\ X_{tt}\geq 1\\
2.35 \parth{\dfrac{1}{X_{tt}} + 0.213}^{0.736} &\text{if}\ X_{tt}\leq 1
\end{cases}
\\
%
S &= \dfrac{1}{1+ 2.53\times 10^{6} \Re_{D_{h}}^{1.17}}
\end{align}
where $X_{tt}$ is the Lockhart-Martinelli parameter \cite{lockhart_proposed_nodate} and $x$ the vapor quality.

\npar

Each of those correlations has been developed for given experimental conditions such as high pressure flows up to 150 bar for Thom and Jens \& Lottes correlations while Chen correlation is dedicated for low to moderate pressure ($\sim 35\ $bar).

\subsection{First Heat Flux Partitioning Approaches}

The spirit of splitting the different heat transfer mechanisms (single-phase, nucleate boiling) using different heat transfer coefficients is actually a premise to what is nowadays called the "Heat Flux Partitioning" approach. Bowring \cite{bowring_physical_1962} is among the first author who actually expressed the total heat flux at the wall as separate contributions between different heat transfer mechanisms by writing:

\begin{equation}
\phi_{w} = \phi_{e} + \phi_{a} + \phi_{SP}
\end{equation}
where $\phi_{e}$ is the latent-heat transfer associated to evaporation, $\phi_{a}$ the convection due to bubble agitation and $\phi_{SP}$ the single-phase heat transfer to the liquid. 

\npar

Making steps towards a more mechanistic description of boiling, he wrote:

\begin{equation}
\phi_{e} = N_{sit} f V_{b} \rho_{V} h_{LV}
\end{equation}
where $N_{sit}$ is the nucleation site density on the surface, $f$ the bubble nucleation frequency, $V_{b}$ the bubble volume. Those parameters need dedicated description and modeling based on experimental measurements. Nonetheless, an empirical parameter is introduced by Bowring to estimate the bubble-agitation term $\phi_{a}$ as:

\begin{align}
\phi_{w} &= \phi_{SP} + \parth{1+\varepsilon}\phi_{e}\\
\varepsilon &= \dfrac{\phi_{a}}{\phi_{e}} \approx \frac{\rho_{L}c_{p,L}}{\rho_{V}h_{LV}} \parth{T_{w} - T_{L}}
\end{align}
where $\varepsilon$ has to be experimentally estimated. We can notably observe that its expression is that of a Jakob number based on the temperature difference between the wall and the liquid temperature.

\npar

This idea of splitting the wall heat flux and try to precisely model each heat transfer mechanisms using detailed parameters (nucleation site density, frequency, nucleated bubble diameter, etc.) has been since a very active matter of research in the field of boiling heat transfer \cite{del_valle_subcooled_1985, judd_comprehensive_1976}. With increasing experimental insights allowing to access local parameters relevant for the physics at stake close to the wall \cite{maity_effect_2000, richenderfer_investigation_2018, yoo_experimental_2016, kossolapov_experimental_2021, sugrue_experimental_2014}, the development of such models has become increasingly important both for analytic and simulation grounds. Indeed, the clear separation of the different heat fluxes is a very useful mathematical formulation for multiphase CFD computations where each term can be used as a phase-related source term at the wall \cite{guelfi_neptune_2007, gilman_development_2014}.

\npar

In the following sections, we chronologically present four different Heat Flux Partitioning models developed between 1990 and 2020. For each model, we detail the mathematical formulation of the different fluxes along with associated physical assumptions and highlight the parameters that require a dedicated closure law.


\section{Kurul \& Podowski (1990)}
\label{sec:hfp_kurul}


In their original work published in 1990, Kurul \& Podowski \cite{kurul_multidimensional_1990} proposed a complete closure for the wall heat flux partitioning. Their model is among the most referred to by many authors and has been particularly used in CMFD codes due to its simple formulation and closure laws.

\begin{note*}{}
The wall boiling model of NCFD mostly rely on the Kurul \& Podowski formulation as presented in Chapter \ref{chap:ncfd}.
\end{note*}


Kurul \& Podowski considered the applied heat flux to be divided between three mechanisms (Figure \ref{fig:KP_hfp}):

\begin{itemize}
\item A liquid single-phase convective heat flux $\phi_{c,L}$ ;
\item A boiling heat flux $\phi_{e}$ ;
\item A quenching heat flux $\phi_{q}$ induced by bubble movement when leaving the surface.
\end{itemize}

\begin{figure}[!h]
\centering
\includegraphics[width=0.6\linewidth]{img/HFP/KP/KP_hfp.PNG}
\caption{Sketch of the HFP considered by Kurul \& Podowski (by Manon \cite{manon_contribution_2000}).}
\label{fig:KP_hfp}
\end{figure}

The total wall heat flux being :

\begin{align}
\phi_{w}=\phi_{c,L}+\phi_{e}+\phi_{q}
\end{align}

The convective heat flux is expressed as :
\begin{align}
\phi_{c,L}=A_{c,L} \rho_{L} c_{p,L}U_{L,\delta} \St_{L,\delta}\parth{T_{w}-T_{L,\delta}}
\label{eq:KP_hfp_phicL}
\end{align}
with $A_{c,L}$ the portion of the wall unaffected by boiling, $U_{L}$ the liquid velocity, $\St$ the Stanton number and $\delta$ a location in the buffer layer.

\begin{note*}{}
The choice of Kurul \& Podowski of taking the liquid properties at the buffer layer is not further detailed in their work. It could however be questioned when comparing the bubble size with $\delta$. Moreover, one should keep in mind that the chosen mechanisms were dedicated to pool boiling description rather than flow boiling.
\end{note*}


\npar
Assuming bubbles are spherical and leave the surface at diameter $D_{lo}$, they write:
\begin{align}
\phi_{e}=\frac{1}{6}\pi {D_{lo}}^{3}\rho_{V}h_{LV}fN_{sit}\\
\label{eq:KP_hfp_phie}
\end{align}
with $N_{sit}$ the nucleation site density and $f$ the nucleation frequency.

\npar
The quenching heat flux occurring over the wait time $t_{w}$ between two nucleated bubbles is computed as:  
\begin{align}
\phi_{q}=t_{w}fA_{q}\frac{2\lambda_{L}\parth{T_{w}-T_{L,\delta}}}{\sqrt{\pi \eta_{L} t_{w}}}
\label{eq:KP_hfp_phiq}
\end{align}

This expression corresponds to the time-average (over a time $t_{w}$) conductive heat flux from a surface at $T_{w}$ towards a semi-infinite liquid medium initially at $T_{L,\delta}$ as expressed by Del Valle and Kenning \cite{del_valle_subcooled_1985}.

\npar
They also estimate the portion of the surface affected by the bubbles as:

\begin{align}
A_{q}=\mathrm{min}\parth{1\ ;\ F_{A}\pi R_{lo}^{2} N_{sit}}=1-A_{c,L}
\label{eq:KP_hfp_Aq}
\end{align}
where $F_{A}=4$ accounts for the bubble area of influence when leaving the surface.

\npar

One of the main hypothesis of the model is also to suppose that the bubble departure frequency $f$ is directly related to the wait time $t_{w}$ by neglecting the bubble growth time as:

\begin{equation}
f = \frac{1}{t_{w}}
\end{equation}

\textbf{\underline{Required closure relationships : }} $N_{sit}$, $f$ (or $t_{w}$), $D_{lo}$ 

%------------------------------------------------

\section{Basu, Warrier \& Dhir (2005)}
\label{sec:hfp_basu}

In 2005, Basu \etal \cite{basu_wall_2005, basu_wall_2005-1} proposed a new HFP model together with a series of experiments to further study the different closure relationships required in their approach. This model was meant to account for finer descriptions of the multiple phenomena at stake in subcooled flow boiling. In particular, they account for bubble sliding and merging and thus distinguish bubble departure diameter $D_{d}$ (leaving the nucleation site) and lift-off diameter $D_{lo}$ (leaving the wall).

\npar
Their approach consists of separating the boiling flow in three regions (Figure \ref{fig:basu_hfp_zones}):

\begin{itemize}
\item Before Onset of Nucleate Boiling (ONB) zone, where only liquid forced convection occurs, yielding:

\begin{equation}
\phi_{w}=h_{c,L} \parth{T_{w}-T_{L}}
\label{eq:basu_hfp_preONB}
\end{equation}

\item Zone between the ONB and the OSV, prior to observing a net amount of vapor with bubble lifting off the surface. The heat flux is still totally transferred to the liquid, but the equivalent convective heat transfer coefficient $\overline{h_{c,L}}$ is supposed to be enhanced by 30\% due to the presence of bubbles on the wall:

\begin{equation}
\phi_{w}=\overline{h_{c,L}} \parth{T_{w}-T_{L}}\approx 1.3 h_{c,L}\parth{T_{w}-T_{L}}
\label{eq:basu_hfp_postONB}
\end{equation}

Basu \etal compute the ONB wall temperature as:

\begin{align}
&T_{w, ONB}= T_{sat} + \frac{4 \sigma T_{sat}}{D_{c}\rho_{V} h_{LV}}\\
%
&D_{c}= \sqrt{\frac{8 \sigma T_{sat} \lambda_{L}}{\rho_{V} h_{LV} \phi_{w}}}\parth{ 1- \exp{ - \theta^{3} - 0.5 \theta}  }
\label{eq:basu_hfp_ONB}
\end{align}
where $D_{c}$ represents the diameter of active cavities and $\theta$ the static contact angle.

\item Post-OSV (Onset of Significant Void) zone, where bubbles now leave the surface towards the bulk flow and the other parts of the HFP appear \ie the boiling and quenching fluxes. The beginning of OSV is defined by Basu \etal using a critical liquid temperature $T_{L,OSV}$ as:

\begin{equation}
T_{L,OSV}=T_{sat}-0.7\exp{-0.065 \frac{D_{d}h_{c,L}}{\lambda_{L}}}\frac{\phi_{w}}{h_{c,L}}
\label{eq:basu_hfp_TLOSV}
\end{equation}

\end{itemize}


\begin{figure}[h]
\centering
\subfloat[Heat transfer zones subdivision]{
\includegraphics[width=0.5\linewidth]{img/HFP/Basu/zones.PNG}
\label{fig:basu_hfp_zones}
}
\subfloat[Considered sliding and merging mechanisms]{
\includegraphics[width=0.5\linewidth]{img/HFP/Basu/slide.PNG}
\label{fig:basu_hfp_sliding}
}
\caption{Sketch of the heat transfers zones and bubble behavior considered by Basu \etal. (Adapted from \cite{basu_wall_2005}). $Q$ represents the heat transfer.}
\label{fig:basu_hfp_sketch}
\end{figure}
	

The hypothesis of Basu \etal is that the heat flux is first transferred to the superheated liquid close to the wall (by convection and transient quenching), part of which contributing to the evaporation through the liquid-vapor interface. The remaining heat is transferred to the bulk liquid ($\phi_{L}$) either from the superheated liquid layer or bubble condensation. The whole heat transfer mechanism thus writes:

\begin{equation}
\phi_{w} = \phi_{c,L}+\phi_{q} = \phi_{e} + \phi_{L}
\label{eq:basu_hfp_phitot}
\end{equation}

In order to estimate the quenching heat flux associated to bubble sliding and lift-off, Basu \etal consider two cases: 

\begin{itemize}
\item[1)] Bubble sliding from departure ($D=D_{d}$) to lift-off ($D=D_{lo}$) ;
\item[2)] Bubble coalescence with neighboring sites before departure.
\end{itemize} 

Those two cases are distinguished using the average distance between nucleation sites $s$, which they suppose equal to $1/\sqrt{N_{sit}}$.

\subsection{Case 1 : Bubble Sliding, $D_{d}<s$}

In this case, the bubble will grow up to its departure diameter $D_{d}$ and slide over a length $l_{sl,0}$ before lifting-off. If $l_{sl,0}<s$, the bubble will slide up to its lift-off diameter $D_{lo}$ and leave the wall without colliding with other bubbles. On the contrary, if $l_{sl,0}\geq s$ the sliding bubble will merge with bubbles growing on their nucleation site, inducing a sudden growth of the bubble diameter that can exceed $D_{lo}$ and thus trigger lift-off after sliding over a reduced length $l_{sl}<l_{sl,0}$. Those assumptions are summarized on Figure \ref{fig:basu_hfp_sliding}.

\npar
If bubble coalescence occurs, the number of bubbles lifting-off the surface is lower than the actual number of nucleating sites. Basu \etal thus define a reduction factor $R_{f}$ that damps the total nucleation site density:

\begin{align}
R_{f}&=
\begin{dcases}
\dfrac{s}{l_{sl}} = \dfrac{1}{l_{sl}\sqrt{N_{sit}}} & \text{if } l_{sl,0} \geq s \\
1 & \text{if } l_{sl,0}<s
\end{dcases}
\label{eq:basu_hfp_Rf}
\end{align}  


Regarding bubble sizes, they suppose that bubbles coalesced by a sliding bubble while growing have a diameter $D_{d}$ \ie they were close to departure. %(in reality, the coalesced bubble would have a diameter $D<D_{d}$). 
This results in a bubble of diameter $D=\parth{D_{sl}^{3} + D_{d}^{3}}^{1/3}$ which will trigger lift-off if $D>D_{lo}$. Consequently, a sliding bubble is allowed to merge with numerous bubbles before lifting off. Noting $N_{merg}$ the number of coalesced bubble and $D_{N}$ the resulting bubble diameter, the sliding distance is:

\begin{equation}
l_{sl}=N_{merg}s + l_{D_{N}\rightarrow D_{lo}}
\end{equation}
where $l_{D_{N}\rightarrow D_{lo}}$ is the remaining distance to slide if $D_{N}<D_{lo}$, being $0$ if $D_{N} \geq D_{lo}$.

\npar

The surface swiped by the sliding bubble is then expressed as $A_{sl} = C\overline{D}l_{sl}$ with $\overline{D}$ the average bubble diameter during sliding and $C$ the ratio between the bubble diameter and its foot, expressed  based on measurements from Maity \cite{maity_effect_2000} as :

\begin{equation}
C=1-\exp{2-\theta ^{0.6}}
\end{equation}

After observing in their experiments that $D_{d}\approx 0.5D_{lo}$, Basu \etal choose:

\begin{equation}
\overline{D}=\frac{D_{lo}+D_{d}}{2}\approx 0.75D_{lo}
\end{equation}


Defining:
\begin{equation}
t^{*}=\parth{\dfrac{\lambda_{L}}{h_{c,L}}}^{2} \dfrac{1}{\pi \eta_{L}}
\label{eq:basu_hfp_tstar}
\end{equation} the time at which transient conduction heat transfer becomes equal to forced liquid convection, the resulting quenching heat transfer $\phi_{q}$ is computed as:

\begin{equation}
\phi_{q} = \frac{1}{t_{w}+t_{g}} \int_{0}^{T}\frac{\lambda_{L}}{\sqrt{\pi \eta_{L}t}}\parth{T_{w}-T_{L}}A_{sl}R_{f}N_{sit}dt
\end{equation}
where $T=t^{*}$ if $t^{*}<t_{w}+t_{g}$ (forced convection dominates at some point during a nucleation cycle which total time is the wait time $t_{w}$ plus the bubble growth time $t_{g}$) or $T=t_{w}+t_{g}$ if $t^{*}\geq t_{w}+t_{g}$ (transient conduction dominates over the whole nucleation cycle).

\npar
The liquid convective heat transfer is therefore:

\begin{equation}
\phi_{c,L} = \overline{h_{c,L}}\parth{T_{w}-T_{L}}A_{c,L} + \overline{h_{c,L}}\parth{T_{w}-T_{L}}A_{sl}R_{f}N_{sit}\parth{1-\mathrm{min}\parth{1\ ;\ \frac{t^{*}}{t_{w}+t_{g}}}}
\label{eq:basu_hfp_phicL}
\end{equation}
with $A_{c,L} = 1 - A_{sl}R_{f}N_{sit}$ the wall area unaffected by bubbles.

\npar
The boiling heat flux is:

\begin{equation}
\phi_{e} = \rho_{V}h_{LV}\frac{\pi}{6}D_{lo}^{3}R_{f}N_{sit}\frac{1}{t_{w}+t_{g}}
\label{eq:basu_hfp_phie}
\end{equation}


\subsection{Case 2 : Bubble Coalescence without Sliding, $D_{d}\geq s$}

Under higher wall superheat, the subsequent rise in the nucleation site density $N_{sit}$ can lead to boiling regimes where bubbles coalesce with each other at early stages of their lifetime \ie while still attached to their nucleation site. This situation is accounted for by Basu \etal in the case when $D_{d} \geq s$ by considering immediate lift-off of coalesced bubble at radius $D > D_{lo}$. In this case, the total density of bubbles leaving the surface is lower than $N_{sit}$ and is thus reduced using:

\begin{equation}
R_{f} = \frac{s^{3}}{D_{lo}^{3}}
\end{equation}

Under this massively coalescing regime, the entire surface will experience quenching due to bubble lift-off all over the heater. Depending on the values of $t^{*}$, we have:

\begin{align}
\phi_{q} &= 
\begin{dcases} \dfrac{1}{t_{w}+t_{g}} \int_{0}^{t^{*}}\dfrac{\lambda_{L}}{\sqrt{\pi \eta_{L}t}}\parth{T_{w}-T_{L}}dt & \text{if } t^{*}<t_{w} \\
%
\dfrac{1}{t_{w}+t_{g}} \crocht{ \int_{0}^{t_{w}}\dfrac{\lambda_{L}}{\sqrt{\pi \eta_{L}t}}\parth{T_{w}-T_{L}}dt + \int_{0}^{T}\dfrac{\lambda_{L}}{\sqrt{\pi \eta_{L}t}}\parth{T_{w}-T_{L}}\crocht{1-S_{b}N_{sit}}dt} & \text{if } t^{*}\geq t_{w} 
\end{dcases}
\end{align}

\begin{align}
\phi_{c,L} &= 
\begin{dcases} \overline{h_{c,L}}\parth{T_{w}-T_{L}}\frac{t_{w}-t^{*}}{t_{w}+t_{g}} + \overline{h_{c,L}}\parth{T_{w}-T_{L}}\crocht{1-A_{b}N_{sit}}\frac{t_{g}}{t_{w}+t_{g}} & \text{if } t^{*}<t_{w} \\
%
\overline{h_{c,L}}\parth{T_{w}-T_{L}}\crocht{1-A_{b}N_{sit}}\frac{t_{w}+t_{g}-t^{*}}{t_{w}+t_{g}} & \text{if } t^{*}\geq t_{w} 
\end{dcases}
\end{align}
with $A_{b} = \dfrac{\pi \parth{Cs}^{2}}{4}$.
\npar

The boiling heat flux still expressed as Eq. \ref{eq:basu_hfp_phie}.

\npar

\begin{remark*}{}
We can note that contrary to  Kurul \& Podowski model, the bubble nucleation frequency is computed without neglecting the bubble growth time $t_{g}$ and is thus expressed as:

\begin{equation}
f = \dfrac{1}{t_{g} + t_{w}}
\end{equation}
\end{remark*}

\textbf{\underline{Required closure relationships : }} $N_{sit}$, $t_{w}$, $t_{g}$, $D_{d}$, $D_{lo}$, $l_{sl,0}$, $h_{c,L}$. 


%----------------------------------------------------------------------------------------

\section{Gilman (2017)}

A more recent HFP model dedicated to CFD simulations has been proposed in Gilman PhD work \cite{gilman_development_2014} and summarized in Gilman \& Baglietto \cite{gilman_self-consistent_2017}. Among the different evolution proposed in their work, we can mention:

\begin{itemize}
\item A probabilistic law to account for static interaction between nucleation sites ;
\item A force-balance approach to compute the bubble departure and lift-off diameters ;
\item A generic law for the enhanced forced convection coefficient accounting for bubble presence ; 
\item The presence of a modified quenching term accounting for local wall superheat beneath a bubble dry spot.
\end{itemize}

The total heat flux is partitioned between the liquid forced convection $\phi_{c,L}$, the solid quenching $\phi_{q,s}$, the quenching due to bubble sliding $\phi_{q,sl}$ and the evaporation flux $\phi_{e}$. Yielding:

\begin{equation}
\phi_{w} = \phi_{c,L} + \phi_{q,s} + \phi_{q,sl} + \phi_{e}
\label{eq:gilman_hfp_phiw}
\end{equation} 

\begin{figure}[!h]
\centering
\includegraphics[width=0.8\linewidth]{img/HFP/Gilman/mind_map.png}
\caption{Heat Flux Partitioning mind-map description by Gilman \cite{gilman_development_2014} (Adapted from \cite{gilman_self-consistent_2017}).}% $q''$ are the heat fluxes and $sc$ means "sliding conduction".
\label{fig:gilman_hfp}
\end{figure}


The convective term is computed in a way similar to Basu \etal \cite{basu_wall_2005} in Eq. \ref{eq:basu_hfp_phicL}:

\begin{equation}
\phi_{c,L} = h_{c,L}\parth{1-A_{sl}N_{sit,a}}\parth{T_{w}-T_{L}} + \overline{h_{c,L}}A_{sl}N_{sit,a}\parth{1 - \frac{t^{*}}{t_{w}+t_{g}}}\parth{T_{w}-T_{L}}
\label{eq:gilman_hfp_phicL}
\end{equation}
where $N_{sit,a}$ is the active nucleation site density that will generate sliding bubbles, that can differ from the empirical value of available sites $N_{sit}$ usually computed by a correlation. The time $t^{*}$ is the same time as in Eq. \ref{eq:basu_hfp_tstar}, $A_{sl}$ the bubble sliding area, $t_{w}$ and $t_{g}$ respectively the wait and bubble growth times.

\npar

The actually active nucleation site density $N_{sit,a}$ (that will generate bubbles) is considered by Gilman to be smaller than $N_{sit}$ by considering a static interaction between the available sites \ie the fact that a bubble laying on a site may be blocking nucleation from an other site laying beneath its foot. Following a so-called Complete Spatial Randomness (CSR) approach, they express the probability $\mathcal{P}$ to find a site under a growing bubble of radius $R_{d}$ as:

\begin{equation}
\mathcal{P} = 1-e^{-N_{b}\pi R_{d}^{2}}
\label{eq:gilman_hfp_pinter}
\end{equation} 
where $N_{b} = \dfrac{t_{g}}{t_{w}+t_{g}}N_{sit} = t_{g} f N_{sit}$ is the density of bubbles covering the heated surface.

The number of active sites is then computed as:

\begin{equation}
N_{sit,a} = \parth{1-\mathcal{P}}N_{sit} = \exp{-\dfrac{t_{g}}{t_{w}+t_{g}}N_{sit}\pi R_{d}^{2}} N_{sit}
\label{eq:gilman_hfp_nsita}
\end{equation}

This value is then reduced by Gilman to obtain the density of sites generating sliding bubbles $N_{sit,a}^{*}$ using a reduction factor accounting for sliding bubble coalescence (similar to Basu \etal in Eq. \ref{eq:basu_hfp_Rf}):

\begin{equation}
N_{sit,a}^{*}=R_{f}N_{sit,a} = \frac{s}{l_{sl,0}+s} N_{sit,a}
\end{equation}
where $l_{sl,0}$ is the sliding length of a single bubble and $s=1/\sqrt{N_{sit,a}}$.


The sliding quenching term is also computed in a way similar to Basu \etal as:

\begin{align}
\phi_{q,sl}&=\frac{2\lambda_{L}\parth{T_{w}-T_{L}}}{\sqrt{\pi \eta_{L}t^{*}}} t^{*}fA_{sl}N_{sit,a}^{*}
\label{eq:gilman_hfp_phiq} \\
%
A_{sl} &= \overline{D}l_{sl} = \frac{D_{d}+D_{lo}}{2}\parth{N_{merg}s+l_{D_{N}\rightarrow D_{lo}}}
\end{align}

Regarding the boiling heat flux, Gilman splits it in two contributions respectively associated with the inception of nucleation ($\phi_{e,in}$) and liquid microlayer evaporation ($\phi_{e,ML}$) :

\begin{align}
\phi_{e} &= \phi_{e,in} + \phi_{e,ML} \\
%
&= \frac{4}{3}\pi R_{d}^{3}\rho_{V} h_{LV} \frac{1}{t_{w}+t_{g}}N_{sit,a} + V_{ML} \rho_{L}h_{LV} \frac{1}{t_{w}+t_{g}}N_{sit,a}\\
%
V_{ML} &= \frac{2}{3}\pi \parth{\frac{R_{d}}{2}}^{3}\delta_{max}
\end{align}
where $\delta_{max}=2\ \mu\text{m}$ is the largest bubble microlayer thickness based on experiments from Gerardi \cite{gerardi_investigation_2009}.

\npar

Finally, the solid quenching term is written as:

\begin{align}
& \phi_{q,s} = \rho_{w} c_{p,w} V_{q} \Delta T_{q} \frac{1}{t_{g}+t_{w}} N_{sit,a}\\
%
& V_{q} = \frac{2}{3} \pi r_{w}^{2}
\end{align}
with subscript $w$ denoting the wall properties, $r_{w}$ the dry patch radius and $\Delta T_{q}=2\ K$ the extra wall superheat as suggested by Gerardi \etal \cite{gerardi_study_2010}.

\npar

\textbf{\underline{Required closure relationships : }} $N_{sit}$, $t_{w}$, $t_{g}$, $D_{d}$, $D_{lo}$, $l_{sl,0}$, $h_{c,L}$, $\overline{h_{c,L}}$. 



\section{Kommajosyula (2020)}
\label{sec:hfp_komma}

The last HFP model we will look through in this Chapter was proposed by Kommajosyula \cite{kommajosyula_development_2020} in the continuation of Gilman \cite{gilman_development_2014} work. In particular, he proposed:

\begin{itemize}
\item A finer modeling of the bubble microlayer evaporation ;
\item Simpler empirical correlations for closure parameters such as bubble departure diameter and wait time ;
\item A modification of the probabilistic approach for the static interaction between sites.
\end{itemize}



\begin{figure}[!h]
\centering
\includegraphics[width=0.8\linewidth]{img/HFP/komma/hfp_komma.PNG}
\caption{Heat Flux Partitioning considered by Kommajosyula \cite{kommajosyula_development_2020}.}% $q''$ are the heat fluxes and $sc$ means "sliding conduction".
\label{fig:komma_hfp}
\end{figure}



The total heat flux is partitioned by Kommajosyula between four fluxes : the forced liquid convection $\phi_{c,L}$, the sliding conduction $\phi_{q,sl}$, the evaporation $\phi_{e}$ and a direct convective flux to the vapor $\phi_{vap}$. Using the wall dry area $S_{dry}$, he writes:

\begin{equation}
\phi_{w} = \parth{1-S_{dry}}\parth{\phi_{c,L} + \phi_{q,sl} + \phi_{e}} + S_{dry} \phi_{vap}
\end{equation}

The liquid convective term is expressed as:

\begin{equation}
\phi_{c,L} = \parth{1-S_{sl}}h_{c,L}\parth{T_{w} - T_{L}}
\end{equation}
where $S_{sl}$ is the surface covered by sliding bubbles.

\npar

Assuming that the transient conduction during quenching operates during the time $t^{*}$ (Eq. \ref{eq:basu_hfp_tstar}), Kommajoysula rewrites the total quenching flux using the forced-convection heat transfer coefficient as:

\begin{align}
\phi_{q,sl} &= 2 h_{c,L} \underbrace{ A_{sl} N_{sit,a} \frac{t^{*}}{t_{g}+t_{w}} }_{S_{sl}} \parth{T_{w}-T_{L}}\\
A_{sl} &= \dfrac{1}{\sqrt{N_{sit,a}}} \frac{D_{d} + D_{lo}}{2}
\label{eq:komma_phiq}
\end{align}
with $N_{sit,a}$ the active nucleation site density, $D_{d}$ and $D_{lo}$ respectively the bubble departure and lift-off diameters.

\begin{note*}{}
At first glance, it seems surprising to see $h_{c,L}$ (convection coefficient) in the quenching term which is of transient conduction nature. This is due to the assumption that the quenching time is $t^{*}=\parth{\dfrac{\lambda_{L}}{h_{c,L}}}^{2} \dfrac{1}{\pi \eta_{L}}$ which once replaced in Eq. \ref{eq:gilman_hfp_phiq} under the square root at the denominator yields Eq. \ref{eq:komma_phiq}.
\end{note*}

\npar

The total sliding area fraction $S_{sl}$ is then expressed as:

\begin{equation}
S_{sl} = \min{1\ ;\ A_{sl}N_{sit,a}t^{*}f}
\end{equation}

The evaporation heat flux is based on Gilman approach by splitting it between inception and microlayer evaporation:

\begin{align}
\phi_{e} &= \phi_{e,in} + \phi_{e,ML} \\
&= \frac{4}{3}\pi \parth{\frac{D_{in}}{2}}^{3}\rho_{V} h_{LV} f N_{sit,a} + V_{ML} \rho_{V} h_{LV} f N_{b}
\end{align}


The microlayer terms are modeled based on the contact angle $\theta$ and the capillary number $\Ca = \dfrac{\mu_{L} U_{b}}{\sigma}$, where $U_{b}$ is the bubble interface velocity, as:

\begin{align}
D_{in} &= D_{d}~ \max{1\ ;\ 0.1237\Ca^{-0.373}\sin{\theta}}\\
V_{ML} &= \delta_{ML}D_{ML}^{2} \frac{\pi}{12} \parth{2 - \parth{\frac{D_{dry}}{D_{ML}}}^{2} - \frac{D_{dry}}{D_{ML}}}
\end{align}
with $D_{ML} = D_{in} / 2$ and $\dfrac{D_{dry}}{D_{ML}} = \max{1\ ;\ 0.1237\Ca^{-0.373}\sin{\theta}}$.

\npar

The microlayer thickness $\delta_{ML}$ is expressed as:

\begin{equation}
\delta_{ML}= 4 \times 10^{-6} \sqrt{\frac{\Ca}{\Ca_{0}}}
\end{equation}
with $\Ca_{0} = 2.16 \times 10^{-4} \Delta T_{w}^{1.216}$.


\npar

The active nucleation site density $N_{sit,a}$ is computed based on the Complete Spatial Randomness approach proposed by Gilman. However, it is expressed using Lambert's W-function $\mathcal{W}$ as:

\begin{equation}
N_{sit,a} = \frac{\mathcal{W}\parth{f t_{g} \pi \dfrac{D_{d}^{2}}{4} N_{sit}}}{f t_{g} \pi \dfrac{D_{d}^{2}}{4}}
\end{equation}
with $\mathcal{W}$ being approximated to avoid an implicit solving of this equation.

\npar

\textbf{\underline{Required closure relationships : }} $N_{sit}$, $t_{w}$, $t_{g}$, $D_{d}$, $D_{lo}$, $h_{c,L}$. 


%was proposed by Zhou \etal \cite{zhou}. It is one of the most recent available in the literature and was built along with associated experiments for validation. In particular they compute separate heat flux contributions for static ($st$) or sliding bubbles ($sl$), yielding a total heat flux:


%
%\begin{equation}
%\phi_{w} = \phi_{c,L} + \parth{\phi_{e,st}+ \phi_{e,st}} + \parth{\phi_{e,sl}+\phi_{q,sl}}
%\end{equation}
%
%An interesting aspect of Zhou \etal work is the presence of a condensation term in the evaporation heat fluxes, written as:
%
%\begin{align}
%\phi_{e,st} =& \rho_{V}h_{LV}\frac{4}{3}\pi R_{d}^{3} N_{sit,a} f  + h_{cond}\parth{T_{sat}-T_{L}}A_{cond} N_{sit,a} f \parth{t_{g}-t_{s}} \\
%%
%\phi_{e,sl} =& \frac{\pi}{6}\rho_{V}h_{LV}\parth{D_{lo}^{3}-D_{d}^{3}}fN_{sit}^{*} + h_{cond} \parth{T_{sat}-T_{L}}A_{cond}N_{sit}^{*}t_{sl}f\\
%%
%A_{cond} =& \pi D_{d} \mathrm{max}\parth{R_{d}\parth{1+\cos{\theta}}-y_{sat}\ ;\ 0}
%\end{align}
%where $y_{sat}$ is the wall distance where the liquid is at saturation temperature, $t_{s}$ the moment when $D_{b}=y_{sat}$ and $N_{sit}^{*}$ the number of sliding bubbles.
%
%The condensation heat transfer coefficient is computed using the correlation from Ranz \& Marshall:
%
%\begin{equation}
%h_{cond} = \frac{\lambda_{L}}{D_{d}}\parth{2+0.6\Re_{b}^{0.5}}\Pr_{L}^{0.3}
%\end{equation}
%
%Following similar approaches to HFP models from Basu or Gilman, the quenching heat flux for static bubbles is expressed as:
%
%\begin{align}
%\phi_{q,st}&=
%\begin{dcases}
%t_{w}f\frac{2\lambda_{L}\parth{T_{w}-T_{L}}}{\sqrt{\pi \eta_{L}t_{w}}}A_{b}N_{sit} & \text{if } t^{*} \geq t_{w}\\
%%
%t^{*}f\frac{2\lambda_{L}\parth{T_{w}-T_{L}}}{\sqrt{\pi \eta_{L}t^{*}}}A_{b}N_{sit} + h_{c,L}\parth{T_{w}-T_{L}}A_{b}N_{sit}^{*}f\parth{t_{w}-t^{*}} & \text{if } t^{*}<t_{w}
%\end{dcases}
%\end{align}
%with $A_{b}= \underbrace{K_{b}}_{=2} \pi R_{d}^{2}$.
%
%And for the sliding bubbles:
%
%\begin{align}
%\phi_{q,sl}&=
%\begin{dcases}
%\frac{2\lambda_{L}\parth{T_{w}-T_{L}}}{\sqrt{\pi \eta_{L}\parth{t_{g}+t_{w}}}}A_{sl}N_{sit}^{*} & \text{if } t^{*} \geq t_{g}+t_{w}\\
%%
%t^{*}f \frac{2\lambda_{L}\parth{T_{w}-T_{L}}}{\sqrt{\pi \eta_{L}t^{*}}}A_{b}N_{sit}^{*} + h_{c,L}\parth{T_{w}-T_{L}}A_{sl}N_{sit}^{*}f\parth{t_{g}+t_{w}-t^{*}} & \text{if } t^{*}<t_{w}
%\end{dcases}
%\end{align}
%with $A_{sl}=\underbrace{K_{sl}}_{=2}\dfrac{D_{d}+D_{lo}}{2}l_{sl}$ and $l_{sl}=\mathrm{min}\parth{l_{sl,0}\ ;\ \dfrac{1}{\sqrt{t_{g}fN_{sit}}} }$.
%
%Finally, the forced liquid convective heat flux is computed:
%
%\begin{equation}
%\phi_{c,L} = h_{c,L}\parth{1-A_{b}N_{sit}-A_{sl}N_{sit}^{*}}\parth{T_{w}-T_{L}}
%\end{equation}
%
%\npar


\section{Conclusion}


As we have seen, the HFP approach is widely studied since decades and is still an active field of work to hopefully reach an exhaustive and precise modeling of all the heat transfer phenomena that are at stake in wall boiling. While most models are similar to each other regarding terms such as single-phase convection or transient conduction heat transfer, they may differ in their methods to account for bubble sliding, microlayer evaporation or modeling of the bubble nucleation cycle.

\npar

Moreover, each of those approaches requires a certain number of closure laws to compute important physical parameters among which we can in particular cite:
\begin{itemize}
\item The nucleation site density $N_{sit}$ ;
\item The bubble nucleation period / frequency : $f=\dfrac{1}{t_{g}+t_{w}}$ ;
\item The bubble departure and lift-off diameters $D_{d}$ and $D_{lo}$.
\end{itemize}

\npar

In this global framework, we will propose the development of a new HFP model by first trying to have a close look at boiling bubble dynamics to propose an acceptable modeling of a bubble lifetime growing and sliding on the wall using most recent experimental and numerical results from the literature (Chapter \ref{chap:bub_dyn}). Then, we will tackle the problem of the closure laws aiming to compute important yet highly uncertain parameters ($N_{sit}$, $t_{w}$, etc.) before proposing a final formulation for the HFP that will be compared to experimental measurements (Chapter \ref{chap:HFP_Assembling}). In order to complete the description of the aforementioned models, we will highlight throughout the development when a given closure law is used in one of the presented models.
%
%In each of the 4 presented HFP models, there is a number of parameters that still need to be computed in order for the model to be fully expressed. Those parameters are often the nucleation site density, bubble diameters, etc. To do so, closure relationships are used and differ from one model to another. Here we want to sum up the different choices of the authors to point the variety of possibilities that exist.
%
%\subsection{Nucleation Site Density : $N_{sit}$}
%
%The value of the nucleation site density is a very sensitive parameter that controls the intensity of the boiling heat transfer. Unfortunately, its value can vary over a very large range of value depending on the operating conditions and heater material and is thus expressed using experimental correlations.
%
%\npar
%In their work, Kurul \& Podowski used the law of Lemmert and Chawla \cite{lemmert}:
%
%\begin{align}
%N_{sit}=\crocht{210\parth{T_{w}-T_{sat}}}^{1.8}
%\end{align}
%
%
%\npar
%Later, Basu \cite{basu} correlated her own experimental results to obtain:
%
%\begin{align}
%N_{sit}=&
%\begin{dcases}
%0.34\crocht{1-\cos{\theta}} {\Delta T_{w}}^{2} & \text{if } \Delta T_{w,ONB}<\Delta T_{w} < 15\ K\\
%3.4\times 10^{-5}\crocht{1-\cos{\theta}} {\Delta T_{w}}^{5.3} & \text{if } \Delta T_{w} > 15\ K
%\end{dcases}
%\end{align}
%
%\npar
%
%\npar
%Finally, Zhou \etal followed a approach similar to Basu \etal by correlating their own experimental data:
%
%\begin{align}
%N_{sit} =& N_{0}\parth{1-\cos{\theta}}\crocht{\exp{\mathrm{f}\parth{P} \Delta T_{w} } -1 }
%\mathrm{f}\parth{P} = 0.218~\ln{\dfrac{P}{P_{0}}}+0.1907
%\end{align}
%with $N_{0}=55~395.26\ \mathrm{m}^{-2}$ and $P_{0}=1.01\ \mathrm{bar}$.
%
%
%
%\subsection{Bubble Departure Frequency $f$, Growth Time $t_{g}$ and Wait Time $t_{w}$}
%
%
%Historically, Kurul \& Podowski modeled the BDF according to Cole \cite{cole} which derived an expression using photographic observations of bubble nucleation, further verified by Ceumern \& Lindenstjerna for pool boiling of water at pressures up to 8 bar. :
%
%\begin{align}
%f = \sqrt{\frac{4}{3} \frac{g \parth{\rho_{L}-\rho_{V}}}{\rho_{L}D_{d}} }
%\label{eq:f_Cole}
%\end{align}
%
%Then, assuming that the growth time of the bubble before departure is small compared to the wait time before a new bubble nucleates ($t_{g,d} \ll t_{w}$) gives :
%
%\begin{align}
%t_{w} \approx \frac{1}{f}
%\end{align}
%
%
%Although the assumption of a negligible growth time is true in different cases, notably when bubble departure diameter is small \ie they leave their nucleation site nearly instantly, this is not generally true and the value of the ratio $\dfrac{t_{g,d}}{t_{w}}$ can vary by decades depending on the thermal-hydraulics conditions.
%
%Thus, later models considered separate modeling of the growth and wait time in order to compute the BDF as :
%
%\begin{align}
%f = \frac{1}{t_{g,d} + t_{w} }
%\label{eq:frequency}
%\end{align}
%
%
%In this scope, Basu \etal \cite{basu} correlated their own exprimental results to obtain :
%
%\begin{align}
%t_{g} =& \frac{D_{d}^{2} }{45~e^{-0.02~\Ja_{L}} \eta_{L} \Ja_{w}} \label{eq:tg_Basu}
% \\
%t_{w} =&  139.1\ {\Delta T_{w}}^{-4.1}
%\label{eq:tw_Basu}
%\end{align}
%
%
%Gilman \& Baglietto \cite{gilman} chose use the growth law of Zuber \cite{zuber1961} and the total BDF of Cole (Eq. \ref{eq:f_Cole}) to close the relation \ref{eq:frequency}.
%
%\begin{align}
%t_{g,d} = \frac{\pi {R_{d}}^{2}}{4b^{2} \Ja_{w} \eta{L}},\ \ b=1.56
%\end{align}