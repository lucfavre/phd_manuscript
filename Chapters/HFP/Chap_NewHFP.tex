% Chapter X

\chapter{Assembling and Validation of a New Heat Flux Partitioning Model} % Chapter title

\label{ch:NewHFP} % For referencing the chapter elsewhere, use \autoref{ch:name} 

%----------------------------------------------------------------------------------------


\section{General description of the model}
\label{sec:model}

The main goal of such a model is to provide a way to compute the wall temperature $T_{w}$ resulting from the applied wall heat flux $\phi_{w}$, or the other way around.

\npar
In order to try to be as extensive as possible regarding the different heat transfer mechanisms at stake, the wall heat flux is supposed to be split between 4 different contributions (Figure \ref{fig:HFP}) :

\begin{itemize}
\item A convective heat flux towards the liquid phase, unaffected by the presence of bubbles on the heater surface : $\phi_{c,l}$
\item A boiling heat flux, representing the energy removed from the wall to grow a bubble up to its lift-off diameter : $\phi_{b}$
\item A quenching heat flux, accounting for transient heat transfer to the liquid phase when bubbles slide or lift-off from the wall : $\phi_{q}$
\item A convective heat flux towards the vapor phase, representing the heat transfer occurring through the dry areas of the surface beneath the bubbles : $\phi_{c,v}$
\end{itemize}



\begin{figure}[h!]
\centering
\fbox{

\begin{tikzpicture}[scale=3.0]

\coordinate (O) at (0,0);
\coordinate (A1) at (1,0);
\coordinate (A2) at (2,0);
\coordinate (A3) at (3,0);
\coordinate (A) at (4,0);


%Sections and wall
\draw (O) -- (A);
\draw ($(O)-(0,0.03)$) -- ($(A)-(0,0.03)$);
\foreach \i in {0,...,19}
{
\draw (\i*0.2,0) -- (\i*0.2+0.05,-0.03);
}



%\draw[dashed, gray!70!white] (A1) --++ (0,1);
%\draw[dashed, gray!70!white] (A2) --++ (0,1);
%\draw[dashed, gray!70!white] (A3) --++ (0,1);



%Flow arrows
\foreach \i in {1,...,12} 
{
\coordinate (Oloc) at ($(O)+(-0.1,\i/13)$);
\draw[->,>=latex, gray!50!blue] (Oloc)--++({ln(1+0.05*\i)},0);
}

\draw ($(Oloc) + ({ln(1+0.05*12)},0)$) node[below right]{${\overline{U_{L}}}$};


%Liquid heat flux

\coordinate (Ophi) at (0.4,0);

\draw[->,>=latex, thick, blue!70!gray] ($(Ophi)+(0,-0.1)$)--($(Ophi)+(0,+0.1)$);
\draw ($(Ophi)+(0,-0.15)$) node{${\phi_{c,L}}$};


%Boiling heat flux
\coordinate (Ob) at (1.0,0);

\tikzmath{\alph = 40; \alphrad= \alph * pi / 180; \ray=0.15; \rayw = sin(\alphrad r) * \ray;}; %Geom values

\coordinate (Oarc) at ($(Ob)+({\ray * sin(\alphrad r)},0)$);
\shade[ball color = gray!40, opacity = 0.4] (Oarc) arc({-(pi/2-\alphrad) r}:{(pi+pi/2-\alphrad) r}:\ray);

\draw (Oarc) arc({-(pi/2-\alphrad) r}:{(pi+pi/2-\alphrad) r}:\ray);


\draw[->,>=latex, thick, brown!80!black] plot [smooth, tension=0.5] coordinates {($(Oarc)+(0,-0.1)$) ($(Oarc)+(0.05,+0.03)$) ($(Oarc)+(-0.05,+0.1)$)};

\coordinate (Oarc2) at ($(Oarc) + (-2*\rayw,0)$);
\draw[->,>=latex, thick, brown!80!black] plot [smooth, tension=0.5] coordinates {($(Oarc2)+(0,-0.1)$) ($(Oarc2)+(-0.05,+0.03)$) ($(Oarc2)+(+0.05,+0.1)$)};

\draw ($(Ob)+(0,-0.15)$) node{${\phi_{e}}$};



%Vapor convective flux
\coordinate (Ob) at (1.75,0);

\tikzmath{\alph = 40; \alphrad= \alph * pi / 180; \ray=0.2; \rw=\ray * sin(\alphrad r);};

\coordinate (Oarc) at ($(Ob)+(\rw,0)$);
\shade[ball color = gray!40, opacity = 0.4] (Oarc) arc({-(pi/2-\alphrad) r}:{(pi+pi/2-\alphrad) r}:\ray);

\draw (Oarc) arc({-(pi/2-\alphrad) r}:{(pi+pi/2-\alphrad) r}:\ray);

\draw[red, thick, densely dashed] (Oarc) --++ (-2*\rw, 0);

\draw[->,>=latex, thick, red!80!black] ($(Ob)+(0,-0.1)$)--($(Ob)+(0,+0.1)$);
\draw ($(Ob)+(0,-0.15)$) node{${\phi_{c,V}}$};



%Quenching heat flux
\coordinate (Ob2) at (2.75,0);

\tikzmath{\alph = 40; \alphrad= \alph * pi / 180;
\dalph=20; \dalphrad=\dalph*pi/180;
\alphadvrad=\alphrad - \dalphrad;
\alphrecrad=\alphrad + \dalphrad;
\ray=0.2; 
\rayadv=\ray *(1+cos(\alphrad r))/(1+ cos(\alphadvrad r);
\rayrec=\ray *(1+cos(\alphrad r))/(1+ cos(\alphrecrad r);};


\coordinate (Cb2) at ($(Ob2)+(0.03,{\ray * cos(\alphrad r)})$);
\draw[green!50!black,->, >=latex] (Cb2)--++({\ray+0.08},0); \draw ($(Cb2)+({\ray+0.08},0)$) node[above]{$\overline{U_{b}}$ }; %Bubble velocity


\coordinate (Oarc) at ($(Ob2)+({\ray * sin(\alphrad r)},0)$);

\shade[ball color = gray!40, opacity = 0.4] (Oarc) arc ({-(pi/2-(\alphadvrad)) r}:{(pi/2) r}:\rayadv) arc ({(pi/2) r}:{(pi+pi/2-(\alphrecrad)) r}:\rayrec);

\draw (Oarc) arc ({-(pi/2-(\alphadvrad)) r}:{(pi/2) r}:\rayadv) arc ({(pi/2) r}:{(pi+pi/2-(\alphrecrad)) r}:\rayrec);

\coordinate (Ob2) at (3.5,0.5);
\tikzmath{\ray=0.25;};

\shade[ball color = gray!40, opacity = 0.4] (Ob2) circle(\ray);
\draw (Ob2) circle(\ray);



\tikzmath{\rayspi=0.07;};

\draw[->,>=stealth,gray!50!blue] plot[domain=0:3.2,smooth,xshift=65,yshift=3] ({(\x *pi) r}:{\rayspi*(1-\x/6)}) ;
\draw[->,>=stealth,gray!50!blue] plot[domain=0:3.2,smooth,xshift=70,yshift=5] ({(\x *pi) r}:{\rayspi*(1-\x/6)}) ;

\coordinate (Ophisl) at ($(Oarc) - (\rayadv+\rayrec+0.05,0)$);
\draw[->,>=latex, thick, orange!90!gray] ($(Ophisl)+(0,-0.1)$)--($(Ophisl)+(0,+0.1)$);
\draw ($(Ophisl)+(0,-0.15)$) node{${\phi_{q,sl}}$};


\tikzmath{\rayspi=0.08;};

\draw[->,>=stealth,gray!50!blue] plot[domain=0:3.2,smooth,xshift=94,yshift=3] ({(\x *pi) r}:{\rayspi*(1-\x/6)}) ;
\draw[->,>=stealth,gray!50!blue] plot[domain=0:3.2,smooth,xshift=105,yshift=3] ({(\x *pi) r}:{\rayspi*(1-\x/6)}) ;


\coordinate (Ophiq) at ($(Ob2) - (0,0.5)$);
\draw[->,>=latex, thick, orange!90!gray] ($(Ophiq)+(0,-0.1)$)--($(Ophiq)+(0,+0.1)$);
\draw ($(Ophiq)+(0,-0.15)$) node{${\phi_{q,lo}}$};

\end{tikzpicture}

}
\caption{Sketch of the considered HFP}
\label{fig:HFP}
\end{figure}


The supposed mechanisms yields the total wall heat flux partioning (\ref{eq:HFP}) :

\begin{align}
\label{eq:HFP}
\phi_{w}=\phi_{c,l} + \phi_{e} + \phi_{q} + \phi_{c,v}
\end{align}


In the following subsections, we focus out analysis on each term to detail its modeling.

\npar


\subsection{Convective heat fluxes}
\label{subsec:conv_HF}

The convective heat fluxes towards the liquid phase $\phi_{c,l}$ and the vapor phase $\phi_{c,v}$ can be written using an associated heat transfer coefficient (\ref{eq:conv_HF}) :

\begin{align}
\label{eq:conv_HF}
\phi_{c,l}=\orangemath{a_{c,l}} \bluemath{h_{c,l}}\parth{\redmath{T_{w}}-\bluemath{T_{l}}} ~\text{ and }~
\phi_{c,v}=\orangemath{a_{c,v}} \bluemath{ h_{c,v}}\parth{\redmath{T_{w}}-\bluemath{T_{v}}}
\end{align}



\subsection{Boiling heat flux}

The total energy associated with the nucleation of a bubble with a volume $V_{b}$ can be expressed as $V_{b}\rho_{v}h_{lv}$. If one knows the nucleation frequency $f$ at which bubbles are generated along with the nucleation site density on the heater surface $N_{sit}$, the resulting heat flux associated with the nucleation phenomenon can thus be written as (\ref{eq:boil_HF}) :

\begin{align}
\label{eq:boil_HF}
\phi_{b}&=\bluemath{N_{sit}} \bluemath{f} \orangemath{V_{b}} \rho_{v} h_{lv}
\end{align}

\subsection{Quenching heat flux}

The quenching heat flux accounts for the transient heat transfer which occurs when cold liquid is brought close to the wall when a bubble slides or lifts-off, thus disrupting the previously established thermal boundary layer.

\textsc{Del Valle} \& \textsc{Kenning} have supposed that this kind phenomenon can be represented as a semi-infinite transient heat transfer between the liquid at $T_{l}$ and the wall at $T_{w}$. Solving the conductive heat transfer problem yields an instantaneous heat flux expressed as Eq.~\ref{eq:inst_quench_HF}.

\begin{align}
\label{eq:inst_quench_HF}
\phi_{q}\parth{t} = \frac{\lambda_{l}\parth{\redmath{T_{w}}-\bluemath{T_{l}}} }{\sqrt{\pi \eta_{l} t} }
\end{align}


Therefore, we can average this heat flux over a time $t_{w}$, during which the quenching operates, and ponderating it both by the portion of the affected heater area $a_{q}$ and the fraction of quenching time over a total bubble nucleation cycle $t_{w}f$, yielding : 

\begin{align}
\phi_{q}&=\orangemath{a_{q}} \bluemath{t_{w}} \bluemath{f} \frac{1}{\bluemath{t_{w}} } \int_{0}^{  \bluemath{t_{w}} } \frac{\lambda_{l}\parth{\redmath{T_{w}}-\bluemath{T_{l}} }}{\sqrt{\pi \eta_{l} t }} = \orangemath{a_{q}} \bluemath{t_{w}} \bluemath{f} \frac{2 \lambda_{l} \parth{\redmath{T_{w}} - \bluemath{T_{l}} } }{\sqrt{\pi \eta_{l} \bluemath{t_{w}} } }
\end{align}


\subsection{Needed closure relationships}

After expressing each heat flux components of the global partitioning, the resulting formulations yields a first list of parameters for which closure relationships (or at least precise definition) are needed. Terms previously highlighted in \textcolor{orange}{orange} will be given a specific definition, terms in \textcolor{blue}{blue} require a closure law, wall temperature is indicated in \textcolor{red}{red}.

\npar

The different terms needing further development are listed below :

\begin{itemize}
\item The fractions of the heater area ponderating convective and quenching heat transfers : $a_{c,l}$, $a_{c,v}$ and $a_{q}$(\ref{sec:geometry})
\item The convective heat transfer coefficients : $h_{c,l}$ and $h_{c,v}$ (Section \ref{sec:HTC})
\item The nucleation site density over the heater surface : $N_{sit}$ (Section : \ref{sec:NSD})
\item The nucleation frequency, which includes both the growth time $t_{g}$ (Section \ref{sec:bubble_growth}) of a bubble and the waiting time $t_{w}$ ({\color{red} Je dois encore proposer une modélisation pour $t_{w}$, à discuter}) : $f=1/\parth{t_{g}+t_{w}}$
\item The total bubble volume $V_{b}$ (\ref{subsec:geom_bub}) generated until its lift-off, thus including the modeling of the bubble lift-off diameter : $D_{lo}$ 
\item The phases temperature : $T_{l}$ (\ref{subsec:liq_temp}) and $T_{v}$ (\ref{subsec:vap_temp})
\end{itemize}

