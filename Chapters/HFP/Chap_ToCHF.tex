% Chapter 1

\chapter{Perspectives Towards Critical Heat Flux Prediction} % Chapter title

\label{ch:to_CHF} % For referencing the chapter elsewhere, use \autoref{ch:introduction} 

%----------------------------------------------------------------------------------------

In this last Chapter of this part dedicated to the modeling of wall boiling, we brielfy discuss historical approaches to represent the Boiling Crisis and propose a perspective use of Heat Flux Partitioning models as a mean to detect the Critical Heat Flux.


\section{Previous Modeling of the Boiling Crisis}

Historically, since the pioneering observations of Nukiyama \cite{nukiyama} who identified the maximum boiling heat flux, the question of explaining the triggering of the Boiling Crisis has been thoroughly investigated by many researchers who attempted to propose various physical explanations and modelings aimed to estimate the value of the CHF.


\subsection{Empirical Approaches}

As a direct way of estimating the CHF, empirical approaches have remained the preferred solution for engineering problems to tackle the Boiling Crisis issue. Indeed, according to Groeneveld \cite{groenveld}, more than 1000 dedicated CHF correlations for water and heated tubes were reported in 2007. As mentioned in the Introduction, current safety analysis in the nuclear industry actually rely on specific correlations tied to a given core / fuel assembly geometry and usually depend on the pressure $P$, the mass flux $G$, the inlet quality $x_{eq,in}$ and the thermal /heated diameter $D_{th}$. For instance, Westinghouse company developed the so-called "Tong-67" correlation \cite{tong_67} based on data-fitted optimization. 

\npar

Following the huge number of CHF experimental tests that were conducted over the past decades, Groeneveld \etal \cite{groeneveld} proposed to simply gather a very large number of measurements in order to come up with a "Look-Up Table", which directly tabulates the CHF values depending on the operating conditions. Though computationally efficient, this approach is not extendable to any other conditions except those covered by the table data. As a result, their work is limited to external vertical flow of water around circular tubes (\eg fuel rods).  


\subsection{Physical Phenomenology Approaches}


In 1948, Kutatzladze \cite{kutateladze} proposed dimensional analysis to tackle the question of the CHF, based on the interaction of three "energetic scales" respectively associated to surface tension force, gravity and turbulence from which he derived:

\begin{equation}
\phi_{w,CHF} = K h_{LV} \rho_{V} \parth{\frac{\sigma \parth{\rho_{L} - \rho_{V}}g}{\rho_{V}^{2}}}^{1/4}
\label{eq:chf_kutateladze}
\end{equation} 
where $K=0.16$ was analytically estimated.

\npar

This formulation of the CHF found a true success due to its analytic nature and its good performance for moderate to high pressure measurements. The coherent behavior of Kutateladze law has been once more recently confirmed by Kossolapov \cite{kossolapov_experimental_2021} whose CHF measurements dependency with pressure were fairly reproduced.

\npar

Kutateladze approach was further developed by Zuber \cite{zuber_1958} who proposed a physical interpretation of the  Boilig Crises as an hydrodynamic instability. Considering vapor columns coming out of the heated surface (Figure \ref{fig:chf_zuber}), he supposed that the Boiling Crisis was triggered under a combination of Rayleigh-Taylor and Kelvin-Helmoltz instabilities. Assuming the vapor columns were regularly spaced by a distance equal to the Rayleigh-Taylor instability wavelength $\lambda_{D}$, the Boiling Crisis is supposed to be triggered by the merging of those vapor columns due to the emergence of Kelvin-Helmoltz instabilities. He then could analytically derive:

\begin{equation}
\phi_{w,CHF} = K h_{LV} \rho_{V} \parth{\frac{\sigma \parth{\rho_{L} - \rho_{V}}g}{\rho_{V}^{2}}}^{1/4} \parth{1 + \dfrac{\rho_{L}}{\rho_{L}+\rho_{V}}}^{1/2}
\label{eq:chf_zuber}
\end{equation} 
where $K=\pi / 24 \approx 0.131$.

\npar


Other approaches are based on different physical mechanism. For instance, Lee \& Mudawar \cite{lee_mudawar} propose to describe the boiling crisis phenomenon as the evaporation of a very thin liquid sublayer trapped between the wall an elongated slug of vapor (Figure \ref{fig:chf_lee}), following experimental observations %\cite{Mesler (1976), Molen  Galjee (1978), Bhat et al. (1983), Serizawa
%(1983), Hino  Ueda (1985) and Mudawwar et al. (1987)}.

\npar

On the other hand, Weisman \& Pei \cite{weisman_pei} proposed that the boiling crisis was due to lack of turbulent transport of the bubbles from the wall towards the bulk, resulting in a bubble layer close to the wall that will trigger coalescence near the heater and isolating it from the liquid phase. Assuming bubbles were of ellipsoidal shapes at CHF, they considered that boiling crisis occurred when the maximum packing density was reached, \ie having a wall void fraction:

\begin{equation}
\alpha_{CHF} = 0.82
\end{equation} 

Although this criterion is very suitable for CMFD applications \cite{mimouni_chf, cmfd_chf}, the critical void fraction close to the wall can vary in large ranges (down to 30\% at high subcooling) according to experimental observations \cite{bruder}.


\section{Recent Approaches and Advances for CHF Prediction}



The question of the boiling crisis occurrence and CHF value is still a topic of active research nowadays. Recently, new experimental observations have allowed the access to wall-related boiling measurements \cite{kossolapov_experimental_2021} \cite{richenderfer_experimental_2018} \cite{bloch_study_2016} and permitted to better understand the phenomenology behind the trigger of the boiling crisis.

\npar

\subsection{Dry Patch Formation}

As detailed by Kossolapov \cite{kossolapov_experimental_2021}, a significant number of different experiments studying the wall boiling at CHF have demonstrated that the occurrence of the boiling crisis was similar at various pressures and associated to the formation of an irreversibly growing dry patch \cite{AAA} (FIGURES?). Such observations are leveraging elements of mechanistic behavior of bubbles right before the CHF. First attempts of modeling this phenomenon were proposed by Han \& No \cite{han_no} prior to the aforementioned experimental observations. Based on a random distribution of the bubbles on the heater surface (similar to the Poisson distributions discussed in Section \ref{sec:site_interactions}), they proposed that a dry patch is created when a bubble crowding hindering rewetting is reached, which was translated as:

\begin{equation}
\phi_{w,CHF} = \phi_{1b}N_{sit,a} \parth{1 - \mathcal{P}\parth{N \geq N_{c}}}
\end{equation}
where $N_{c}=5$ and $\mathcal{P}$ is the probability to find $N$ bubbles in the area of influence of a single bubble.


This model was recently re-used by \cite{papier_ICONE_Stephane}.


\subsection{Model of Demarly \etal : Stability of the Heat Flux Partitioning}

In the framework of Heat Flux Partitioning modeling, Demarly proposed a modeling of the wall area in direct contact with vapor (similar to $A_{c,V}$ in Section \ref{sec:hfp_new}). Over the direct area beneath single bubbles, Demarly accounts for bubble interaction and models the total wall dry are $S_{dry}$ as:

\begin{equation}
S_{dry} = ft_{g,d} N_{sit,a} \pi \parth{\zeta \underbrace{ e^{f t_{g,d}N_{sit,a} \pi R_{d}^{2} }_{\text{Bubble interaction}} } \sin{\theta} R_{d} }
\end{equation} 
where $\zeta=0.15$ based on data-fitting of Richenderfer \cite{richenderfer_experimental_2018} experiments.

The total heat flux is finally written as:

\begin{equation}
\phi_{w}=\parth{1-S_{dry}}\parth{\phi_{c,L} + \phi_{e} + \phi_{q}} + S_{dry} \phi_{dry}
\end{equation}
where $\phi_{dry}$ is the heat flux through the dry area, which is nearly negligible versus the other heat fluxes.

\npar

This modeling allows to impose a decrease of the total heat flux by gradually enhancing the dry area, thus displaying a maximum value for $\phi_{w}$ when covering a whole boiling curve (Figure \ref{fig:demarly_chf}.


\subsection{Model of Zhang \etal : Bubble Interaction and Footprint distribution}




\section{Simple test of the Zhang Criterion}


