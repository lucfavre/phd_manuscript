\chapter{NEPTUNE\_CFD simulations of DEBORA-Promoteur and AGATE-Promoteur Cases}
\label{chap:prom_ncfd}

In this Chapter, simulations in 


\section{NEPTUNE\_CFD simulations of DEBORA-Promoteur cases}
\label{sec:debprom_ncfd}



%
\begin{figure}[!h]
\centering
\includegraphics[width=1.0\linewidth]{img/DEBORA-Promoteur/cfd/prom_M2_all.png}
\caption{Mesh}
\label{fig:sim_prom}
\end{figure}
%


We simulated 3 cases for each position of the mixing device, covering different local thermodynamic quality near the vanes ($x_{eq,MV}$) :

\begin{itemize}
\item 48G3P26W23Te65 \& 52G3P26W23Te65 with $x_{eq,MV}\approx -1\%$ 
\item 48G3P26W23Te69 \& 52G3P26W23Te69 with $x_{eq,MV}\approx 4\%$ 
\item 48G3P26W23Te75 \& 52G3P26W23Te75 with $x_{eq,MV}\approx 12\%$ 
\end{itemize}

Computations are conducted using two meshes for Te69 cases : a large one (M1) with $444~703$ cells and a fine one (M2) with $3~487~627$ cells. Results for void fraction profiles are shown on Figure \ref{fig:sim_prom}.


%
\begin{figure}[!htb]
\centering
\subfloat[Void fraction - Te65 cases]{
\includegraphics[width=0.33\linewidth]{img/DEBORA-Promoteur/cfd/G3P26W23Te65_alpha.pdf}
}
\subfloat[Void fraction - Te69 cases]{
\includegraphics[width=0.33\linewidth]{img/DEBORA-Promoteur/cfd/G3P26W23Te69_alpha.pdf}
}
\subfloat[Void fraction - Te75 cases]{
\includegraphics[width=0.33\linewidth]{img/DEBORA-Promoteur/cfd/G3P26W23Te75_alpha.pdf}
}
\\
\subfloat[Bubble diameter - Te65 case]{
\includegraphics[width=0.33\linewidth]{img/DEBORA-Promoteur/cfd/G3P26W23Te65_dV.pdf}
}
\subfloat[Bubble diameter - Te69 case]{
\includegraphics[width=0.33\linewidth]{img/DEBORA-Promoteur/cfd/G3P26W23Te69_dV.pdf}
}
\subfloat[Bubble diameter - Te75 case]{
\includegraphics[width=0.33\linewidth]{img/DEBORA-Promoteur/cfd/G3P26W23Te75_dV.pdf}
}
\\
\subfloat[Vapor axial velocity - Te65 case]{
\includegraphics[width=0.33\linewidth]{img/DEBORA-Promoteur/cfd/G3P26W23Te65_Uvap.pdf}
}
\subfloat[Vapor axial velocity - Te69 case]{
\includegraphics[width=0.33\linewidth]{img/DEBORA-Promoteur/cfd/G3P26W23Te69_Uvap.pdf}
}
\subfloat[Vapor axial velocity - Te75 case]{
\includegraphics[width=0.33\linewidth]{img/DEBORA-Promoteur/cfd/G3P26W23Te75_Uvap.pdf}
}
\label{fig:debprom_ncfd}
\end{figure}
%


Quantitatively speaking, it seems that NEPTUNE\_CFD reproduces the effect of vapor acculumation at the center thanks to the pressure gradient generated by the swirl induced by the mixing vanes. The radial position  of the core void fraction peak correctly matches the experimental one. 

However, measured void fraction profiles are not predicted correctly. A particularly strong overestimation of the core void fraction is observed as well as close to the wall. The CMFD results tend to rapidly reach a core void fraction around $60\%$ ($T_{in}=69\degree$C cases) and then flattens with increasing temperature ($T_{in}=75\degree$ cases). This contradicts experimental observation where the void fraction profile globally rises when inlet temperature increases, except at the wall where no peak is observed due to bubble removing effect by the liquid's rotation. Moreover, the $T_{in}=75\degree$ case with MV at $10D_{h}$ experimentally shows local $\alpha$ peaks at $R\approx \pm 6$mm which remain currently unexplained and not reproduced by the simulations. 

To investigate what could be a potential origin for the core void fraction peak overestimation, we present in Section \ref{sec:agate} single-phase flow simulations in the MV geometry.





\section{NEPTUNE\_CFD simulations of AGATE-Promoteur cases}
\label{sec:agate_ncfd}



%
\begin{figure}[!h]
\centering
\includegraphics[width=1.0\linewidth]{img/AGATE/prom_M4_all.png}
\caption{Mesh}
\label{fig:sim_prom}
\end{figure}
%

\begin{figure}[!h]
\centering
\subfloat[Axial velocity]{
\includegraphics[width=0.4\linewidth]{img/AGATE/z15/pos1112_VA.pdf}
}
\subfloat[Radial velocity]{
\includegraphics[width=0.4\linewidth]{img/AGATE/z15/pos1112_VT.pdf}
}
\\
\subfloat[Axial RMS]{
\includegraphics[width=0.4\linewidth]{img/AGATE/z15/pos1112_RMS_A.pdf}
}
\subfloat[Radial RMS]{
\includegraphics[width=0.4\linewidth]{img/AGATE/z15/pos1112_RMS_T.pdf}
}
\caption{Results for $z=0.8\ D_{h}$, diameter 11-12}
\end{figure}





\begin{figure}[!h]
\centering
\subfloat[Axial velocity]{
\includegraphics[width=0.4\linewidth]{img/AGATE/z200/pos910_VA.pdf}
}
\subfloat[Radial velocity]{
\includegraphics[width=0.4\linewidth]{img/AGATE/z200/pos910_VT.pdf}
}
\\
\subfloat[Axial RMS]{
\includegraphics[width=0.4\linewidth]{img/AGATE/z200/pos910_RMS_A.pdf}
}
\subfloat[Radial RMS]{
\includegraphics[width=0.4\linewidth]{img/AGATE/z200/pos910_RMS_T.pdf}
}
\caption{Results for $z=10.4\ D_{h}$, diameter 9-10}
\end{figure}


\begin{figure}[!h]
\centering
\subfloat[Axial velocity]{
\includegraphics[width=0.4\linewidth]{img/AGATE/z440/pos910_VA.pdf}
}
\subfloat[Radial velocity]{
\includegraphics[width=0.4\linewidth]{img/AGATE/z440/pos910_VT.pdf}
}
\\
\subfloat[Axial RMS]{
\includegraphics[width=0.4\linewidth]{img/AGATE/z440/pos910_RMS_A.pdf}
}
\subfloat[Radial RMS]{
\includegraphics[width=0.4\linewidth]{img/AGATE/z440/pos910_RMS_T.pdf}
}
\caption{Results for $z=22.9\ D_{h}$, diameter 9-10}
\end{figure}


In this penultimate section, we briefly investigate single-phase flow within the same geometry as Section \ref{sec:deb_prom}.


On Figure \ref{fig:agate1}, we present some of the results obtained with NEPTUNE\_CFD using the $R_{ij}-\varepsilon~SSG$ turbulence model on the M2 mesh, along with a smooth wall law and a rough wall law (roughness $\epsilon=0.01$mm). The turbulent fluctuations Root Mean Square (RMS) correspond, for instance, to $\sqrt{<u^{'^{2}}_{x}>}$ for the $x$ direction where $u'_{i}$ represents the fluctuating part of the velocity along compononent $i$ and $<.>$ the time-averaging operator. Subscripts $R$ and $A$ stand for radial and axial values ; $U_{0}$ is the average inlet velocity.


%
\begin{figure}[!htb]
\centering
\includegraphics[scale=0.3]{img/AGATE/test.png}
\caption{NCFD vs. Exp. - Top \& Middle : Radial velocity and turbulent RMS ($z=30$mm \& $z=440$mm) -  Bottom : Axial velocity and turbulent RMS ($z=440$mm).}
\label{fig:agate1}
\end{figure}

Non-symmetric radial velocity profiles close to the MV are quite well reproduced by the simulations. However, far downstream the MV, it appears that the fluid's rotation is overestimated by the model with a smooth wall approach, while applying a roughness helps to reduce the magnitude of the swirl. Moreover, the radial turbulent fluctuations are better estimated by the rough wall approach at $z=440$~mm.

On the other hand, it seems that the rough wall approach deteriorates the axial velocity profile compared to the experiment. As shown on the bottom part of Figure \ref{fig:agate1}, the smooth wall simulation returns a flat velocity profile closer to the experiment than the rough wall one which overestimates the core velocity peak.

Both simulations globally underestimate the turbulent fluctuations, which can have a significant influence over the observed discrepancies on velocity profiles since turbulence plays a key role to homogenize the fluid flow. 

Those results finally highlight the fact that simulation of such rotating flows may need a particular wall approach to better capture the induced swirl and its dissipation. Correct prediction of turbulent fluctuations would be of significant interest to ensure liquid velocity validation. Further investigations on boiling cases could possibly be improved by a roughness approach, which is the current correction used for two-phase wall laws (Subsection \ref{subsec:wall_func}). 