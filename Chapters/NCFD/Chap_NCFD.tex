% Chapter 1

\chapter{Presentation of the NEPTUNE\_CFD Code} % Chapter title

\label{chap:ncfd} % For referencing the chapter elsewhere, use \autoref{ch:introduction} 

%----------------------------------------------------------------------------------------

\minitoc


\section{Introduction}

The NEPTUNE\_CFD project (started in 2001) is a research program coordinated by four entities : EDF, CEA, IRSN and Framatome. The initial goals of the project were related to nuclear safety by developing a thermal-hydraulics simulation tool to :

\begin{itemize}
\item Predict the Boiling Crisis in PWR cores ;
\item Study the Loss Of Coolant Accident (LOCA) to predict fuel rod cladding temperature.
\end{itemize}

As a multiphase eulerian code, NEPTUNE\_CFD consists of a local three-dimensional  modeling based on a two fluids-one pressure approach combined with mass, momentum and energy conservation equations for each phase \cite{guelfi_neptune_2007}. 


\npar
%The NEPTUNE\_CFD solver can handle several types of multiphase flows, \eg dispersed vapor-liquid flow of a same component, mixed gas-liquid-solid particles flow of different components, or multiple forms of a single component (dispersed vapor/continuous liquid, dispersed liquid/continuous vapor, continuous liquid/continuous vapor, etc.).
The constitutive equations are solved using a pressure correction and is based on a finite-volume discretization along with a collocated arrangement of the variables. Moreover, NEPTUNE\_CFD allows the use of all type of meshes (hexahedral, tetrahedral, pyramids, etc.), even non-conforming ones, thanks to its face-based data structure. Finally, the code is well-suited for parallel computing, widening its computing capacity to very large meshes.

\npar

The simulations presented in this thesis have all been conducted using the NEPTUNE\_CFD 7.0 modeling framework for dispersed bubbly flows. In the next sections, we will detail the constitutive equations and physical modeling of the code for the simulation of boiling bubbly flows.


\section{Governing Equations for Turbulent Boiling Bubbly Flows}

To simulate two-phase dispersed boiling flows, NEPTUNE\_CFD solves the ensemble-averaged equations of mass conservation, momentum balance and energy conservation for each phase (see Ishii \cite{ishii_thermo-fluid_2011} for details on the derivation).
%As a reminder, the ensemble-averaging of a physical quantity $F_{k}$ for any phase $k$ in a multiphase flows is defined as:



\subsection{Mass Conservation}
\begin{equation}
%MASS
	\dtime{\alpha_{k}\rho_{k}} + \divg{\alpha_{k}\rho_{k} \vect{U_{k}}} = \orangemath{\Gamma_{k}}
	\label{eq:mass}
\end{equation}

Where $\alpha_{k}$, $\rho_{k}$, $\vect{U_{k}}$ are the volumetric fraction, average density and velocity of phase $k$ ; $\orangemath{\Gamma_{k}}=\Gamma_{k,i}+\Gamma_{k,w}$ the interfacial mass transfer term per unit of volume and time split between bulk and wall contribution.
Subscripts $k= L$ or $V$ denotes the liquid or vapor phase, $i$ the interfacial quantities and $w$ the wall contribution.

\subsection{Momentum Balance}

\begin{equation}
%MOMENTUM
	\dtime{\alpha_{k}\rho_{k}\vect{U_{k}}} + \vecdivg{\alpha_{k}\rho_{k} \vect{U_{k}}\otimes \vect{U_{k}}} = -\alpha_{k}\grad{P} + \orangemath{\vect{F_{k,i}}} + \orangemath{\Gamma_{k}}\vect{U_{k,i}} + \alpha_{k}\rho_{k}\vect{g} + \vecdivg{\alpha_{k} \parth{ \tens{\tau_{k,m}} + \orangemath{\tens{\tau_{k,T}}} }}
\label{eq:momentum}
\end{equation}

Where $P$ is the pressure, $\vect{g}$ the gravity, $\orangemath{ \vect{F_{k,i}} }$ the interfacial forces accounting for momentum transfer between phases per unit of volume and time, $\vect{U_{k,i}}$ the interfacial velocity, $\tens{\tau_{k,m}}$ and $\orangemath{ \tens{\tau_{k,T}} }$ respectively the viscous and turbulent (or Reynolds) stress tensor. Subscript $m$ and $T$ respectively denote the molecular (or laminar) and turbulent terms.


\subsection{Energy Conservation}

\begin{equation}
\begin{aligned}
%ENERGY
	\dtime{\alpha_{k}\rho_{k} H_{k}} + \divg{\alpha_{k}\rho_{k} H_{k}\vect{U_{k}}} =& \dtime{\alpha_{k}P} + \orangemath{\Gamma_{k}}H_{k,i}+\orangemath{\vect{F_{k,i}}}\cdot \vect{U_{k}} + \orangemath{Q_{k,I}} + \divg{\alpha_{k}\parth{ \tens{\tau_{k}} + \orangemath{\tens{\tau_{k,T}}} }\cdot \vect{U_{k}}}\\
	& + \divg{\alpha_{k} \parth{-\parth{ \lambda_{k,m}+\orangemath{\lambda_{k,T}}}\grad{T_{k}}}} + \alpha_{k}\rho_{k}\vect{g}\cdot \vect{U_{k}} + \orangemath{Q_{k,w}}
\end{aligned}
\label{eq:energy}
\end{equation}

Where $H_{k}=e_{k}+\dfrac{U_{k}^{2}}{2}+\dfrac{P}{\rho_{k}}=h_{k}+\dfrac{U_{k}^{2}}{2}$ is the total enthalpy of phase $k$, $H_{k,i}$ the interfacial-averaged enthalpy, $\orangemath{ Q_{k,i} }$ the interfacial heat flux per unit of volume and time, $\lambda_{k,m}$ and $\orangemath{ \lambda_{k,T} }$ respectively being the laminar and turbulent thermal conductivity, $T_{k}$ the temperature, $\orangemath{ Q_{k,w} }$ the heat flux from the wall to phase $k$ per unit of volume and time.

\npar

The viscous and Reynolds stress tensors read:

\begin{align}
\tens{\tau_{k,m}} &= \mu_{k} \parth{ \vecgrad{\vect{U_{k}}}+\vecgrad{U_{k}}^{T}  - \frac{2}{3}\divg{\vect{U_{k}}}\tens{I} }\\
\parth{\orangemath{\tens{\tau_{k,T}}} }_{i,j} &= -\rho_{k} \left< U'_{k,i}U'_{k,j} \right>_{k}
\end{align}
with $\vect{U'_{k}}$ is the fluctuating part of velocity for phase $k$.


\npar

This ensemble-average approach requires a given number of closure laws since this mathematical averaging operation removes most of the information about smaller scales physics (compared to the mesh size) such as interfacial exchanges between phases or wall-fluid interaction. Terms for which this modeling effort is needed are colored in orange in equations \ref{eq:mass}, \ref{eq:momentum} and \ref{eq:energy}. Following sections detail the physical modeling for each of those terms. 


\section{Interfacial Transfers Closure Laws}
\label{sec:int_transfers}

The interfacial transfers of mass, momentum and energy are respectively noted in equations \ref{eq:mass}, \ref{eq:momentum} and \ref{eq:energy} : $\Gamma_{k}$, $\vect{F_{k,i}}$ and $Q_{k,i}$.

\subsection{Heat and Mass Transfers}
\label{subsec:ncfd_interf_mass}


The mass transfer term, can be written as: 

\begin{align}
\Gamma_{L,i} + \Gamma_{V,i} = 0\\
%
\Gamma_{L,w} + \Gamma_{V,w} = 0
\end{align}
with $\Gamma_{V,w} \geq 0$ in the case of boiling flows. This finally gives:

\begin{equation}
\Gamma_{L}=-\Gamma_{V}
\end{equation}

\npar
The interfacial heat flux $Q_{k,i}$ can be rewritten in terms of interfacial area concentration $a_{i}$ : 

\begin{equation}
Q_{k,i}=q''_{k,i}a_{i}
\end{equation} 

\npar

Neglecting the mechanical contribution compared to the thermal terms, the energy jump condition can then be expressed as :

\begin{equation}
\sum_{k=L,V}\parth{\Gamma_{k,i}h_{k,i} + q''_{k,i}a_{i}}=0
\label{eq:ncfd_energy_jump}
\end{equation}

The estimation of $h_{k,i}$ is not straightforward since it can either be supposed as:

\begin{enumerate}
\item[H1)] The saturation enthalpy of phase $k$ at the system pressure ;
\item[H2)] The phase-averaged enthalpy.
\end{enumerate}

\npar
In NEPTUNE\_CFD, the assumption H2 is chosen, thus giving the bulk condensation rate :

\begin{equation}
\Gamma_{L,i}=\frac{a_{i}\parth{q''_{L,i}+q''_{V,i}}}{h_{V}-h_{L}}
\label{eq:ncfd_cond_rate}
\end{equation}


The interfacial heat flux densities $q''_{k,i}$ and interfacial area concentration $a_{i}$ are expressed as:

\begin{align}
q''_{k,i}=C_{k,i}\parth{T_{sat}(P)-T_{k}}\\
a_{i}=6 \alpha_{V}/D_{V}
\end{align}
with $D_{V}$ being the vapor phase Sauter mean bubble diameter and $C_{k,i}$ the interfacial heat transfer coefficient.

\npar

\begin{note*}{}
The interfacial area is computed using the transport equation of {Ruyer} \& {Seiler} \cite{ruyer_modelisation_2009} that accounts for bubble breakup and coalescence to estimate bubble diameter.
\end{note*}


\subsubsection{Subcooled Liquid}

For subcooled liquid, the following heat transfer coefficient is used \cite{ranz_evaporation_1952, manon_contribution_2000}:

\begin{align}
C_{L,i}&=\frac{\Nu_{L}\lambda_{L}}{D_{V}}\\
%
Nu_{L}&=2+0.6\Re_{b}^{1/2}\Pr_{L}^{1/3}
\label{eq:ncfd_subcooled_L}
\end{align}

Where $\Re_{b}$ is the bubble Reynolds number $Re_{b}=\dfrac{\norm{\vect{U_{V}}-\vect{U_{L}}}D_{V} }{\nu_{L}}$ and $\Pr_{L}=\dfrac{\nu_{L}}{\eta_{L}}$ the liquid Prandtl number with $\nu_{L}$ and $\eta_{L}$ respectively being the liquid kinematic viscosity and thermal diffusivity.

\begin{remark*}{}
This formulation initially proposed by Ranz \& Marshall \cite{ranz_evaporation_1952} is based on experiments for evaporating droplets. In that regard, its application to bubble condensation could be further discussed. 
\end{remark*}


\subsubsection{Superheated Liquid}

On the other hand, if the liquid is overheated, the maximum of three heat transfer coefficients accounting for different heat transfer mechanisms is taken \cite{berne_analyse_1983}:

\begin{equation}
C_{L,i}=\max{C_{L,i,1}\ ;\ C_{L,i,2}\ ;\ C_{L,i,3}}
\label{eq:ncfd_supheat_L}
\end{equation}

With $C_{L,i,n}=\dfrac{\lambda_{L}\Nu_{L,n}}{D_{V}}$ and :

\begin{equation}
\label{eq:nusselt}
 Nu_{L,1}=2 \text{ ; } Nu_{L,2}=\frac{12}{\pi} \Ja_{L} \text{ ; }  Nu_{L,3}=\sqrt{\frac{4}{\pi}Pe} 
\end{equation}


where $\Pe=\Re_{b} \Pr_{L}$ is the Peclet number, $\Ja_{L}=\dfrac{ \rho_{L}c_{p,L}\bars{T_{sat}-T_{L}}}{\rho_{V} h_{LV}} $ the liquid Jakob number and $h_{LV}$ the latent heat of vaporization. 

\npar

Those three Nusselt numbers respectively correspond to stationary conduction around a sphere, transient conduction for a spherical bubble growth in uniformly superheated liquid \cite{plesset_growth_1954} and transient convection around a sphere in a superheated flow \cite{ruckenstein_mass_1964}.

\subsubsection{Vapor Heat Transfer}

For the vapor phase, a simple law that ensures the vapor temperature to stay close to the saturation temperature is used (which is expected for small bubbles, \eg in a PWR) :

\begin{equation}
C_{V,i}a_{i}=\frac{\alpha_{V}\rho_{V}c_{p,V}}{t_{c}}
\label{eq:ncfd_vap_relaxation}
\end{equation}
where $c_{p,V}$ is the vapor heat capacity at constant pressure and $t_{c}$ a characteristic (relaxation) time given by the user (default value being $t_{c}=0.01\ \text{s}$) .

\subsection{Interfacial Forces}
\label{subsec:ncfd_interf_qdm}

The interfacial momentum transfer (excluding transfer associated to transfer of mass $\Gamma_{k}$) is assumed to be composed of 4 different forces being the, drag $D$, the added mass $AM$, the lift $L$ and the turbulent dispersion $TD$:

\begin{equation}
\vect{F_{k,i}}=\vect{F_{k,D}} + \vect{F_{k,AM}} +\vect{F_{k,L}} +\vect{F_{k,TD}}
\label{eq:ncfd_force_balance}
\end{equation}

\npar

The turbulent dispersion force $\vect{F_{k,TD}}$ originates from the averaging operation conducted on the three other forces expressions, detailed in equations \ref{eq:ncfd_drag}, \ref{eq:ncfd_added_mass}, \ref{eq:ncfd_lift} and \ref{eq:ncfd_turb_disp}.

\subsubsection{Drag Force}

The drag force is modeled according to Ishii \& Zuber \cite{ishii_drag_1979}:

\begin{equation}
\vect{F_{V,D}}=-\vect{F_{L,D}}=-\frac{1}{8}a_{i}\rho_{L}C_{D}\norm{ \vect{U_{V}}-\vect{U_{L}} }\parth{\vect{U_{V}}-\vect{U_{L}} }
\label{eq:ncfd_drag}
\end{equation}

\begin{equation}
C_{D} = \frac{2}{3}D_{V}\sqrt{\dfrac{g\parth{\rho_{L}-\rho_{V}}}{\sigma}}\parth{ \frac{1 + 17.67~f\parth{\alpha}^{1.67}}{18.67f\parth{\alpha}} }, \ f\parth{\alpha} = \parth{1-\alpha}^{1.5}\ \text{for distorted bubbles.}
\end{equation}

\begin{equation}
C_{D} = \frac{8}{3} \parth{1-\alpha}^{2}\ \text{for churn-turbulent regime.}
\end{equation}

\subsubsection{Added Mass Force}

The added mass force is modeled  following Zuber \cite{zuber_dispersed_1964}:

\begin{align}
\vect{F_{V,AM}}=-\vect{F_{L,AM}}=&-C_{AM} \frac{1+2\alpha_{V}}{1-\alpha_{V}}\alpha_{V}\rho_{L}\\
&\times \crocht{ \parth{\dtime{\vect{U_{V}}}+\vecgrad{\vect{U_{V}}}\cdot \vect{U_{V}} } - \parth{\dtime{\vect{U_{L}}}+\vecgrad{\vect{U_{L}}}\cdot \vect{U_{L}} } }
\label{eq:ncfd_added_mass}
\end{align}
with $C_{AM}=\dfrac{1}{2}$ and the term $\dfrac{1+2\alpha_{V}}{1-\alpha_{V}}$ accounts for the impact of bubble concentration.

\subsubsection{Lift Force}

The lift force is modeled based on the experiments of Tomiyama \etal \cite{tomiyama_transverse_2002}:

\begin{equation}
\vect{F_{V,L}}=-\vect{F_{L,L}}=-C_{L}\alpha_{V}\rho_{L}\parth{\vect{U_{V}}-\vect{U_{L}} }\wedge \parth{\rot{\vect{U_{L}}} }
\label{eq:ncfd_lift}
\end{equation}

\begin{equation}
C_{L} = \begin{cases}
    \min{0.288\tanh{0.121\Re_{b}}\ ;\ 0.00105 \Eo_{H}^{3}-0.0159\Eo_{H}^{2} - 0.0204\Eo_{H}+0.474}, & \text{if}\ \Eo_{H}<4\\
   0.00105 \Eo_{H}^{3}-0.0159\Eo_{H}^{2} - 0.0204\Eo_{H}+0.474, & \text{if}\ 4\leq \Eo_{H} \leq 10\\
   -0.27, & \text{if}\ \Eo_{H}>10
  \end{cases}
\end{equation}
where $\Eo_{H} = \dfrac{g\parth{\rho_{V}-\rho_{L}}D_{H}^{2}}{\sigma}$ is a modified E\"otv\"os number with:

\begin{equation}
D_{H} = D_{V}\sqrt[3]{1+0.163^Eo^{0.757}}
\end{equation}

\subsubsection{Turbulent Dispersion Force}

The turbulent dispersion force is computed following the General Turbulent Dispersion approach presented by Lavieville \etal \cite{lavieville_generalized_2017}:

\begin{equation}
\vect{F_{V,TD}}=-\vect{F_{L,TD}}=-\frac{2}{3}\alpha_{L}\alpha_{V}C_{TD}\grad{\alpha_{V}}
\label{eq:ncfd_turb_disp}
\end{equation}

The value of $C_{TD}$ notably depends on the fluid turbulence, drag force, added mass force. Further details on its derivation are presented in Lavieville \etal \cite{lavieville_generalized_2017}.

%\begin{align}
%C_{TD} = \frac{F_{12}\tau_{12}^{t}}{3\rho_{L}} \frac{b+\eta_{r}}{1+\eta_{r}} + \frac{C_{12}}{\rho_{L}}\frac{b^{2} + \eta_{r}}{1+\eta_{r}} - \frac{b+\eta_{r}}{1+\eta_{r}} - \frac{\alpha_{L}}{\alpha_{V}}
%\label{eq:ncfd_CTD}\\
%%
%b&= \frac{1+C_{VM}}{\frac{\rho_{L}}{\rho_{V}} + C_{VM}}
%\end{align}

%with $C_{D}$, $C_{AM}$, $C_{L}$ and $C_{TD}$ the associated forces coefficients, respectively taken from \textsc{Ishii}\cite{ishii1967}, \textsc{Zuber}\cite{zuber1964}, \textsc{Tomiyama}\cite{tomiyama2002} and the Generalized Turbulent Dispersion model (GTD) from \textsc{Lavieville} \etal \cite{lavieville2017}.

\section{Turbulence Modeling}
\label{subsec:turbulence}

%\begin{equation}
%\dtime{R_{ij}} + U_{k}
%\end{equation}
%
%\begin{equation}
%\dtime{\tens{R}} + \vect{U_{k}}\cdot \grad{R_{ij}} = 
%\end{equation}

For bubbly flow simulations, only liquid phase turbulence is taken into account while it is neglected for the vapor phase. The prescribed model is the Reynolds Stress Model $R_{ij}-\varepsilon~SSG$ from Speziale, Sarkar and Gatski \cite{speziale_modelling_1991} adapted to two-phase boiling flows by Mimouni \etal\cite{mimouni_combined_2011}. Noting $R_{ij} = \left< U'_{L,i}U'_{L,j} \right>_{L}$ and $\alpha = \alpha_{V}$, it reads:

\begin{align}
\parth{1-\alpha}+\frac{\mathrm{D}\rho_{L}R_{ij}}{\mathrm{D}t}  = & \frac{\partial}{\partial x_{k}}\crocht{ \parth{\rho_{L}\nu_{L} + \rho_{L}C_{s} \frac{k}{\varepsilon}R_{ij}} \frac{\partial}{\partial x_{k}}\parth{ \parth{1-\alpha} R_{ij} } } \\
&+ \parth{1 - \alpha}\parth{P_{ij} + G_{ij} + \Phi_{ij} + \varepsilon_{ij}}
\end{align}
where $\mathrm{D}/\mathrm{D}t$ is the Lagrangian derivative, $k$ the turbulent kinetic energy, $\varepsilon$ the turbulent pseudo-dissipation rate, $\tens{P}$ the turbulent production term, $\tens{G}$ the work of gravity force, $\tens{\Phi}$ the pressure-strain term and $\tens{\varepsilon}$ the viscous dissipation.

\npar

All those terms need a proper modeling, for which we refer the reader to Mimouni \etal \cite{mimouni_combined_2011}.

\npar

The transport equation on $\varepsilon$ also includes the impact of $\alpha$:


\begin{align}
\parth{1-\alpha}+\frac{\mathrm{D}\varepsilon}{\mathrm{D}t}  &= \frac{\partial}{\partial x_{k}}\crocht{ C_{\varepsilon} \frac{k}{\varepsilon}\left< U'_{L,k}U'_{L,l} \right>_{L}  \frac{\partial \parth{1-\alpha} \epsilon}{\partial x_{l}} } \\
&+ \parth{C_{\varepsilon_{1}} + C_{\varepsilon_{3}}G + C_{\varepsilon_{4}}k\frac{\partial U_{L,k}}{\partial x_{k}} - C_{\varepsilon_{2}}\epsilon } \frac{\parth{1-\alpha}\epsilon}{k}
\end{align}

\npar

We conclude by specifying the values used for the constants in NEPTUNE\_CFD (Table \ref{tab:ncfd_ssg_constants}).


\begin{table}[!h]
\centering
\begin{tabular}{c c c c c c c c c c} 
\hline
$C_{s}$ & $C_{1}$ & $C_{2}$ & $C_{1}^{\omega}$ & $C_{2}^{\omega}$ & $C_{\varepsilon}$ & $C_{\varepsilon_{1}}$ & $C_{\varepsilon_{2}}$ & $C_{\varepsilon_{3}}$ & $C_{\varepsilon_{4}}$ \\
\hline
0.2 & 1.8 & 0.6 & 0.5 & 0.3 & 0.18 & 1.44 & 1.92 & 1.44 & 0.33\\
\hline
\end{tabular}

\caption{Constant values for the SSG model in NEPTUNE\_CFD}
\label{tab:ncfd_ssg_constants}

\end{table}




\section{Wall Boiling Model}
\label{sec:ncfd_HFP}

The modeling of the heterogeneous boiling phenomenon at the wall is based on a Heat Flux Partitioning (HFP) model,  from Kurul \& Podowski original work\cite{kurul_multidimensional_1990} who divided the wall heat flux density $\phi_{w}$ in three terms  :

\begin{itemize}
\item A single phase convective heat flux $\phi_{c,L}$ heating the liquid through the fraction of the wall area unaffected by the vapor bubbles ;
\item A vaporization heat flux $\phi_{e}$ which accounts for the generation of vapor through heterogeneous nucleation ;
\item A quenching heat flux $\phi_{q}$ to represent the thermal impact of bubbles departing from the wall and being replaced by cool liquid
\end{itemize}

A fourth flux is added to this HFP in NEPTUNE\_CFD, following Mimouni \etal\cite{mimouni_computational_2016} who consider a convective heat flux towards the vapor $\phi_{c,V}$ when the wall area is covered by a dense accumulation of bubbles. 

\npar

The model is then ponderated by a phenomenological function $f_{\alpha_{L}}$ enhancing $\phi_{c,V}$ when the void fraction at the wall becomes large. It thus gives Equation \ref{eq:ncfd_HFP} :

\begin{equation}
\phi_{w}=f_{\alpha_{L}} \parth{\phi_{c,L}+\phi_{e}+\phi_{q}} + \parth{1+f_{\alpha_{L}} }\phi_{c,V}
\label{eq:ncfd_HFP}
\end{equation}
where $f_{\alpha_{L}}$ verifies smooth conditions $\lim\limits_{\alpha_{L} \to 1} f_{\alpha_{L}} = 1$, $\lim\limits_{\alpha_{L} \to 0} f_{\alpha_{L}} = 0$, $\lim\limits_{\alpha_{L} \to 0} \dfrac{f_{\alpha_{L}} }{ \alpha_{L}} = 0$ and $\lim\limits_{\alpha_{L} \to 1} \dfrac{ 1 - f_{\alpha_{L}} }{ 1 - \alpha_{L}} = 0$.

\npar
The convective heat fluxes are expressed as:

\begin{align}
\phi_{c,k} &=A_{k}h_{k,log}\parth{T_{w}-T_{k}}\\
%
h_{k,log} &=\frac{ \rho_{k}c_{p,k}{U_{\tau}}}{{T_{L}^{+}} }
\label{eq:ncfd_hfc}
\end{align}
where $A_{k}$ the fraction of the wall area facing phase $k$, $T_{w}$ the wall temperature and $h_{k,log}$ the wall logarithmic convective heat transfer coefficient to phase $k$ based on the wall functions for friction velocity $U_{\tau}$ and non-dimensional liquid temperature $T_{L}^{+}$ described in \ref{subsec:wall_func}.

\npar
The vaporization heat flux is computed following:

\begin{equation}
\phi_{e}=N_{sit}f\rho_{V}h_{LV}\frac{\pi D_{d}^{2}}{6}
\label{eq:phie_NCFD}
\end{equation}


Closure of physical parameters needed are :

\begin{itemize}
\item $N_{sit}$ the nucleation site density modeled as \cite{lemmert_influence_1977}:
\begin{equation}
N_{sit}=\crocht{210\parth{T_{w}-T_{sat}} }^{1.8}
\label{eq:nsit_NCFD}
\end{equation}

\item $f$ the bubble detachment frequency expressed as \cite{cole_bubble_1967}: 
\begin{equation}
f=\sqrt{\frac{4}{3}\frac{g\bars{\rho_{V}-\rho_{L}} }{\rho_{L}D_{d}}}
\end{equation}

\item $D_{d}$ the bubble detachment diameter given by \"Unal correlation \cite{unal_maximum_1976} corrected by Bor\'ee \etal \cite{boree_ecoulements_1992} (Equation \ref{eq:ncfd_unal_boree}).

\end{itemize}



\begin{align}
D_{d}&=2.42\times 10^{-5} P^{0.709} \frac{a}{\sqrt{b\varphi}}\text{ with }
  \label{eq:ncfd_unal_boree}\\
  %
 a&=\frac{\parth{T_{w}-T_{sat}} \lambda_{w}}{2\rho_{V}h_{LV}\sqrt{\pi\eta_{w}}}\\
 %
  b&=\begin{cases}
    \dfrac{T_{sat}-T_{L}}{2 \parth{1-\dfrac{\rho_{V}}{\rho_{L}} }}, & \text{if $St\leq0.0065$}\\
    \dfrac{1}{2 \parth{1-\rho_{V}/\rho_{L} })}\dfrac{\phi_{c,L}+\phi_{e}+\phi_{q}}{0.0065\rho_{L}c_{p,L}\norm{\vect{U_{L}}}}, & \text{\text{if $St>0.0065$}}
    \end{cases}
\end{align}
where $\lambda_{w}$ and $\eta_{w}$ are the wall thermal conductivity and diffusivity, $St=\dfrac{\phi_{c,L}+\phi_{e}+\phi_{q}}{\rho_{L}c_{p,L}\norm{\vect{U_{L}}}\parth{T_{sat}-T_{L} }}$ is the Stanton number and $\varphi=\displaystyle \max{1 ; \parth{\frac{\norm{\vect{U_{L}}}}{U_{0}} }^{0.47} }$ with $U_{0}=0.61\text{m/s}$

\npar

Finally, the quenching heat flux follows the approach of {Del Valle} \& {Kenning} \cite{del_valle_subcooled_1985} supposing that it follows a semi-infinite transient conduction regime: 

\begin{equation}
\phi_{q}=A_{q}t_{q}f\frac{2\lambda_{L} \parth{T_{w}-T_{L}}}{\sqrt{\pi \eta_{L}t_{q}}}
\label{eq:ncfd_phiq}
\end{equation} 
where $t_{q}$ is the quenching time, supposed to be equal to $1/f$, and $A_{q}=N_{sit} \pi R^{2}$ the area experiencing quenching.



\section{Wall Function for Dispersed Boiling Flows}
\label{subsec:wall_func}

In boiling flows, the formation of bubbles at the wall may disturb the liquid velocity profile in the boundary layer. To take this phenomena into account, Mimouni \etal\cite{mimouni_computational_2016} proposed a wall function for boiling flows which tends to the single-phase formulation when $\alpha_{V} \rightarrow 0$ and depends on the bubble diameter and density at the wall: 

\begin{equation}
U^{+}=\frac{1}{\kappa}\ln{ y^{+} } + B - \Delta U^{+} \text{ with } 
\label{eq:ncfd_wall_law}
\end{equation}
where $\kappa$=0.41 is the Von Karman constant, $B=5.3$ the standard single-phase logarithmic law constant and $\Delta U^{+}$ represents the offset of $U^{+}$ due to the wall roughness induced by the presence of bubble.

\npar

The correction $\Delta u^{+}$ is computed as:

\begin{align}
  \Delta U^{+}&=\begin{cases}
    0 & \text{if $k_{r}^{+}\leq 11.3$}\\
    \dfrac{1}{\kappa}\ln{ 1+ C_{kr}k_{r}^{+} } & \text{\text{if $k_{r}^{+}>11.3$}}
  \end{cases} \\
%
k_{r}^{+}&= \dfrac{k_{r}\sqrt{U_{\tau}U_{T}}}{\nu_{L}} \\
%
k_{r}&=\alpha_{V}d_{V}\\
%
U_{T}&=C_{\mu}^{1/4}\sqrt{k_{L}}
\end{align}
with $C_{kr}=0.5$ , $k_{r}$ a "bubble roughness Reynolds number", $C_{\mu}=0.09$ defined from the $k-\varepsilon$ and $k_{L}$ the liquid turbulent kinetic energy.

\npar

The non-dimensional wall liquid temperature $T_{L}^{+}$ is modeled according following a Van Driest formulation:

\begin{equation}
T_{L}^{+} = \int_{0}^{y^{+}} \frac{2\mathrm{d}y}{1 + \dfrac{1}{2}\dfrac{\Pr_{L}}{\Pr_{T}}\sqrt{1+4\kappa^{2} {y^{+}}^{2}  \parth{1-\exp{-y^{+}/A}}^{2}}}
\end{equation}
with $A=25.6$, $\Pr_{T} = 0.9$ and $y^{+} \approx 100$ usually.

\section{Conclusions}

In this chapter, we presented the constitutive equations of NEPTUNE\_CFD. As a summary, the main features of the modeling for dispersed bubbly flows are:

\begin{itemize}
\item Conservation equation for mass, momentum and energy are solved for liquid and vapor ;
\item Interfacial momentum transfer is modeled by including different forces, with in particular recent formulations for turbulent dispersion and lift ;
\item Turbulence is modeled by accounting for vapor presence in the $R_{ij}-\varepsilon$ SSG model along with a wall law based on local bubble size ;
\item Wall boiling is based on a Heat Flux Partitioning approach extending the original formulation of Kurul \& Podowski \cite{kurul_multidimensional_1990} by accounting for vapor convection at high wall void fraction.
\end{itemize}

In order to assess this modeling framework, we will further perform simulations using NEPTUNE\_CFD. In next Chapter, we present the DEBORA database that will serve as validation reference in Chapter \ref{chap:debora_ncfd}.


%\begin{equation}
%\label{eq:wall_law} 
%  T_{L}^{+}=\begin{cases}
%    Pr_{L}~y^{+}, & \text{if $y^{+}\leq 13.2$}\\
%    8.67Pr_{L,T}\parth{ \frac{Pr_{L}}{Pr_{L,T}}-1 } \parth{\frac{Pr_{L,T}}{Pr_{L}} }^{0.25} + \frac{Pr_{L,T}}{\kappa}\ln{Ey^{+}}  & \text{\text{if $y^{+}>13.2$}}
%  \end{cases}
%\end{equation}
%
%With $Pr_{L,T}=0.9$ the turbulent liquid Prandtl number, and $E=7.76$ a constant for smoth walls.