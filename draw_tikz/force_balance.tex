
\documentclass[10pt]{article}

\usepackage[utf8]{inputenc}
\usepackage[french]{babel}

\usepackage{enumitem}


\usepackage{amsmath}
\usepackage{bm}

%%%PWR packages
\usepackage[Gray]{SIunits}
\usepackage[dvipsnames,pdftex,fixpdftex]{xcolor}


%\usepackage{gensymb} 


\usepackage[top=3cm, bottom=2cm, left=2cm, right=2cm]{geometry} %marges

\usepackage{tikz}
\usetikzlibrary{decorations.pathmorphing}
\usetikzlibrary{calc}
\usetikzlibrary{math}
\usetikzlibrary{intersections,arrows.meta,bending} 

\usepackage{pgfplots}
\pgfrealjobname{draw}
\usetikzlibrary{intersections}


\newcommand{\HRule}{\rule{\linewidth}{0.5mm}}
\newcommand{\npar}{\vspace{\baselineskip}}


\newcommand{\vect}[1]{\overline{\boldsymbol{#1}}}
%\newcommand{\dtime}[1]{\frac{\partial #1}{\partial t}}
\newcommand{\dpartial}[2]{\frac{\partial #1}{\partial #2}}
\newcommand{\divg}[1]{\nabla\cdot\left(#1\right)}
\newcommand{\grad}[1]{\overline{\boldsymbol{\nabla}}#1}
\newcommand{\lapla}[1]{\overline{\boldsymbol{\Delta}}#1}
%\newcommand{\tens}[1]{\overline{\overline{\boldsymbol{#1}}}}
\newcommand{\gradtens}[1]{\overline{\overline{\boldsymbol{\nabla}}}#1}
%\newcommand{\inv}[1]{\frac{1}{#1}}

\newcommand{\dtheta}{\mathrm{d}\theta}

\newcommand{\redmath}[1]{\begingroup\color{red}#1\endgroup}
\newcommand{\orangemath}[1]{\begingroup\color{orange}#1\endgroup}
\newcommand{\yellowmath}[1]{\begingroup\color{yellow}#1\endgroup}

\newcommand{\etal}{\textit{et al.}}


\begin{document}

%%%PROGRESSIVE FORCE BALANCE


\fbox{


\begin{tikzpicture}[scale=3.5, every node/.style={scale=0.9}]


%%%%Truncated sphere on a vertical wall


%Tilted bubble
\coordinate (Ob2) at (2.5,1.0);

\coordinate (O1) at (2.5,0);
\coordinate (O2) at (2.5,2);
\draw (O1)--(O2);

\tikzmath{\thet = 40; \thetrad= \thet * pi / 180;
\dthet=10; \dthetrad=\dthet*pi/180;
\thetadvrad=\thetrad - \dthetrad;
\thetrecrad=\thetrad + \dthetrad;
\thetadv=\thetadvrad*180/pi;
\thetrec=\thetrecrad*180/pi;
\ray=0.5; 
\rayadv=\ray *(1+cos(\thetrad r))/(1+ cos(\thetadvrad r);
\rayrec=\ray *(1+cos(\thetrad r))/(1+ cos(\thetrecrad r);};

\coordinate (Oarc) at ($(Ob2)-(0,{\ray * sin(\thetrad r)})$);


\draw (Oarc) arc({-pi+(\thetrecrad)) r}:{0 r}:\rayrec) arc ({0 r}:{pi-(\thetadvrad)) r}:\rayadv);

%Upstream angle
\draw (Oarc) --++(-90+\thetrecrad r:0.3) node[very near end, below]{$\theta + \dtheta$};
\draw ($(Oarc)+(0,-0.15)$) arc(-90:-90+\thetrecrad r:0.15) ;

%Downstream angle
\coordinate (Oarc2) at ($(Oarc) + (0,{\rayadv * sin(\thetadvrad r) + \rayrec * sin(\thetrecrad r)})$);
\draw (Oarc2) --++({90-(\thetadvrad r)}:0.3);
\coordinate (angadv) at ($(Oarc2) +({90-(\thetadvrad r)}:0.3)$);
\draw (angadv)  node[above]{$\theta - \dtheta $};
\draw ($(Oarc2)+(0,+0.15)$) arc(90:{90-(\thetadvrad r)}:0.15);



%Center and radius

\coordinate (Cb) at ($(Oarc2)+( {\ray * cos(\thetrad r)} , {-0.5 * (\rayadv * sin(\thetadvrad r) + \rayrec  * sin(\thetrecrad r) )})$);
\draw (Cb) node{$\times$} node[below right]{$O$};

\draw[densely dashed, <->, >=latex] (Cb) -- (Oarc) node[midway, above left]{$R$};


\draw[densely dashed, <->, >=latex] ($(Oarc) + (-\ray/12,+\ray/15) $) -- ($(Oarc2) + (-\ray/12,-\ray/6) $) node[midway, left]{$d_{w}$};



%Inclination angle

\draw[densely dotted] (Cb) --++ (\ray,0);
\draw[densely dotted] (Cb) --++ ({(1*\dthetrad r)}: \ray );
\draw ($(Cb) + (0.3,0)$) arc(0: {(\dthetrad r)}:0.3)  node[very near end, above]{$\dtheta$};





%%Forces

\draw[->, >=latex, violet!70!black] (Ob2)--++(\ray/2,0) node[near end, above]{$\vect{F_{CP}}$};

%\draw[->, >=latex, red!70!black!] ($(Cb)+(\ray,0)$)--++(\ray/2,0) node[very near end, above]{$\vect{F_{L}}$};
\draw[->, >=latex, red!70!black!] ($(Cb)+({sin(5)},{cos(5)*\rayrec*1.04})$)--++(0,+\ray/2) node[very near end, right]{$\vect{F_{D}}$};
\draw[->, >=latex, red] ($(Cb)+(0,+\rayrec*1.04)$)--++(0,+\ray/1.5) node[very near end, left]{$\vect{U_{b}}$};


\draw[->, >=latex, violet] (Oarc)--++(90+\thetrec:\ray/2) node[very near end, left]{$\vect{F_{C}}$};
\draw[->, >=latex, violet] (Oarc2)--++(-90-\thetadv:\ray/2) node[very near end, left]{$\vect{F_{C}}$};

%\draw[->, >=latex, blue!70!black] (Cb)--++(0,\ray/1.5) node[very near end, right]{$\vect{F_{B}}$};


\draw[->, >=latex, green!50!black!] (Cb)--++(160-\thet:\ray/1.5) node[very near end, above]{$\vect{F_{AM}}$};

%Gravity

\draw[->, >=latex, blue!30!black]  ($(Cb)+({1.5*\ray},{1.5*\ray})$)--++(0,-\ray/2) node[very near end, right]{$\vect{g}$};



%Flow arrows
\foreach \i in {2,...,14} 
{
\coordinate (Oloc) at ($(O1)+(\i/15,0.05)$);
\draw[->,>=latex, gray!70!blue] (Oloc)--++(0,{ln(1+0.03*\i)});
}
\draw[gray!70!blue] ($(Oloc)+(0.1,0.1)$) node{$\vect{U_{L}}$};

%%Referential vectors
\coordinate (Ovect) at (4.0,0);
\draw[->, >=latex] (Ovect)--++(0.3,0) node[very near end, above right]{$\vect{e_{y}}$};
\draw[->, >=latex] (Ovect)--++(0,0.3) node[very near end, above right]{$\vect{e_{x}}$};

%Control volume

\coordinate (Oarc) at ($(Ob2)-(0,{\ray * sin(\thetrad r)})$);
\coordinate (Oarcb) at ($(Oarc)+(-\thetrecrad r : 0.15*\rayrec)$);

\draw[thick] (Oarc) arc({-pi+(\thetrecrad)) r}:{0 r}:\rayrec) arc ({0 r}:{pi-(\thetadvrad)) r}:\rayadv);
\draw[dotted, green!50!black, opacity=0.7, line width = 3pt] (Oarcb) arc({-pi+(1.15*\thetrecrad)) r}:{0.3 r}:1.01*\rayrec) arc ({0.3 r}:{ 0.93*(pi-(\thetadvrad)) r}:1.01*\rayadv)--cycle;

\end{tikzpicture}

}


\end{document}