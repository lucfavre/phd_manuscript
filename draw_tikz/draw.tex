\documentclass[10pt]{article}

\usepackage[utf8]{inputenc}
\usepackage[french]{babel}

\usepackage{enumitem}


\usepackage{amsmath}
\usepackage{bm}

\usepackage{gensymb} 


\usepackage[top=3cm, bottom=2cm, left=2cm, right=2cm]{geometry} %marges

\usepackage{tikz}
\usetikzlibrary{calc}
\usetikzlibrary{math}
\usepackage{pgfplots}
\pgfrealjobname{draw}
\usetikzlibrary{intersections}


\newcommand{\HRule}{\rule{\linewidth}{0.5mm}}
\newcommand{\npar}{\vspace{\baselineskip}}


\newcommand{\vect}[1]{\overline{\boldsymbol{#1}}}
%\newcommand{\dtime}[1]{\frac{\partial #1}{\partial t}}
\newcommand{\dpartial}[2]{\frac{\partial #1}{\partial #2}}
\newcommand{\divg}[1]{\nabla\cdot\left(#1\right)}
\newcommand{\grad}[1]{\overline{\boldsymbol{\nabla}}#1}
\newcommand{\lapla}[1]{\overline{\boldsymbol{\Delta}}#1}
%\newcommand{\tens}[1]{\overline{\overline{\boldsymbol{#1}}}}
\newcommand{\gradtens}[1]{\overline{\overline{\boldsymbol{\nabla}}}#1}
%\newcommand{\inv}[1]{\frac{1}{#1}}

\newcommand{\dtheta}{\mathrm{d}\theta}

\newcommand{\redmath}[1]{\begingroup\color{red}#1\endgroup}
\newcommand{\orangemath}[1]{\begingroup\color{orange}#1\endgroup}
\newcommand{\yellowmath}[1]{\begingroup\color{yellow}#1\endgroup}

\newcommand{\etal}{\textit{et al.}}




%%TO EXTRACT A FIGURE IN PDF PROPERLY CROPPED --> IN COMMAND LINE : pdflatex --jobname=forces draw.tex






\begin{document}


\beginpgfgraphicnamed{growth_analytical}


\fbox{
\begin{tikzpicture}[scale=2.0, every node/.style={scale=1.0}]


%wall
\draw[thick] (-3.5,0) -- (2,0) ; 

%right triple contact line point
\coordinate (A) at (1,0) ; 
\draw (A) arc (-45:180+45:1.5) ; %bubble draw angles from (-alpha) to (180+alpha)

%bubble center
\coordinate (O) at ($(A)+(45+90:1.5)$) ; 
\draw (O) node[above left]{\footnotesize{$O$}};
\draw[densely dashed, <->, >=stealth] ($(O)+(-45:0.05)$) -- (A) node[midway, above]{$R$};
\draw (O) node[black]{$\times$};

%Portion angulaire de la bulle
\coordinate (T) at ($(O) + (0,1.5)$) ; %bubble top
\draw[dashed] (O) -- (T);
\coordinate(Tp) at ($(O) + (0, 1.0)$);%start for angle Theta
\draw[densely dotted, ->, >=stealth] (Tp) arc(90:0-45:1) node[midway, right] {$\Theta$};

%Hauteur coupe
\coordinate (B) at ($(O) + (0,{-1.5*cos(45)})$);
\draw[densely dashed, <->, >=stealth] ($(O) + (0,-0.05)$) -- (B) node[midway, left] {$h$};

%Rayon du pied
\coordinate (Bl) at ($(B) + (0,-0.1)$);
\coordinate (Al) at ($(A) + (0,-0.1)$);
\draw[densely dashed, <->, >=stealth] (Bl) -- (Al) node[midway, below]{$r_{w}$};

%Vecteur surface
\coordinate (S1) at ($(O) + (50:1.5)$);
\draw[thick, green] (S1) arc(50:50+10:1.5);
\coordinate (S2) at ($(O) + (55:1.5)$);
\draw[thick, olive, ->, >=stealth] (S2)--++(50:0.4) node[midway, right]{$\textstyle{\vect{d^{2}S}}$};


%Angle de contact
\draw[color=gray!80] (A) --++(45:1) ;
\draw[densely dotted] (A)++(0.3,0) arc (0:45:0.3) ;
\draw (A)++(22.5:0.6) node{$\theta$};

%\draw (1.0,0) -- (1.530331, 0.53033) ;
%\draw (1.45433, 0.22961) node {$\alpha$};

%\draw (O) node[right] {$O$} ;
%\draw(0,1) node{$\times$};

%Couche limite interface liq/vap
\coordinate (CL) at ($(O) + (-1.5-0.2,0)$);
\draw[dashed, orange] (CL) arc(180 : 110 : 1.7);
\coordinate (Db) at ($(O) + (90+45:1.5)$);
\draw[<->, orange, >=stealth] (Db) --++(90+45:0.2) node[midway, above right]{$\delta_{b}$}
node[near end, above left,purple]{$T_{L}(y)$};

\draw (Db) node[below right,purple]{$T_{sat}$};

%Repere spherique
\coordinate (M) at ($(O) + (25:1.5)$);
\draw (M) node[left, black]{$M$};
\draw[->, olive, thick] (M)--++(25:0.2) node[right,olive]{$\vect{e_{r}}$};
\draw[->, olive, thick] (M)--++(25-90:0.2) node[right,olive]{$\vect{e_{\theta}}$};
\draw[olive, thick] ($(M)+(0:0.6)$) node{$\otimes$} node[right,olive]{$\vect{e_{\varphi}}$};
\draw (M) node{$\bullet$};

%Repere cartesien
\coordinate (RC) at (-4.0, 3);
\draw[->, olive, thick] (RC)--++(0.2,0) node[right, olive]{$\vect{e_{x}}$};
\draw[->, olive, thick] (RC)--++(0,0.2) node[above, olive]{$\vect{e_{y}}$};

%Profil Tliq
\coordinate (Tinf) at (-3.0, 3.5);
\coordinate (Tw) at (-2,0);
\coordinate (delta) at ($(Tinf)+(0,-3.5)$);

\draw[purple] (Tinf) -- (Tw) node[midway, sloped, below]{$T_{L}(y)$};
\draw[purple] (Tinf) node[right]{$T_{\infty}<T_{sat}$};
\draw[purple] (Tw) node[below]{$T_{w}>T_{sat}$};
\draw[<->, orange, >=stealth, densely dashed] ($(Tinf)+(0,-0.03)$) -- (delta) node[midway, left]{$\delta$};


\end{tikzpicture}
}


\endpgfgraphicnamed


\fbox{
\begin{tikzpicture}[scale=2.0, every node/.style={scale=1.0}]

\coordinate (Od) at (0,0);
\coordinate (Ol) at (3,0);
\coordinate (Td) at ($(Od)+(0,0.3)$);
\coordinate (Bd) at ($(Od)+(0,-0.3)$);
\coordinate (Tl) at ($(Od)+(3,0.6)$);
\coordinate (Bl) at ($(Od)+(3,-0.6)$);


\draw[black, fill=orange!80!gray!20] (Td) arc(90:90+180:0.3) -- (Bl) arc(-90:-90+180:0.6) -- cycle;
%\draw[black] (Tl) arc(90:90-180:0.6);


\draw[dashed] (Td) arc(90:90-180:0.3);
\draw[dashed] (Tl) arc(90:90+180:0.6);

\draw[<->, >=stealth, dotted] (Td)--(Bd) node[midway, left]{$D_{d}$};
\draw[<->, >=stealth, dotted] (Tl)--(Bl) node[midway, right]{$D_{l}$};


\end{tikzpicture}
}


\fbox{
\begin{tikzpicture}[scale=2.0, every node/.style={scale=1.0}]

\coordinate (Od) at (0,0);
\coordinate (Ol) at (3,0);
\coordinate (Td) at ($(Od)+(0,0.3)$);
\coordinate (Bd) at ($(Od)+(0,-0.3)$);
\coordinate (Tl) at ($(Od)+(3,0.6)$);
\coordinate (Bl) at ($(Od)+(3,-0.6)$);


\draw [dashed](0,0,0)--(1,0,0)--(1,1,0)--(0,1,0)--cycle; % face arrière
\draw (0,0,1)--(1,0,1)--(1,1,1)--(0,1,1)--cycle; % face avant%arêteshorizontales,del’arrièreversl’avant
\draw (0,0,0) -- (0,0,1); % bas gauche
\draw (1,0,0) -- (1,0,1); % bas droit
\draw (1,1,0) -- (1,1,1); % haut droit
\draw (0,1,0) -- (0,1,1); % haut gauche
\end{tikzpicture}
}

\fbox{

\begin{tikzpicture}[scale=2.0, every node/.style={scale=1.0}]

\tikzmath{\alph = 70; \ray=1; \h=\ray*cos(\alph); \alphr= \alph*pi/180;};


\draw (-2,0,-2)--(2,0,-2)--(2,0,2)--(-2,0,2)--cycle;
%\draw (0.3,0) arc (-45:180+45:0.3);

\foreach \t in {1,0.9,0.8,0.7,0.6,0.5,0.4,0.3,0.2,0.1,0,-0.1,-0.2,-0.3,-0.4,-0.5,-0.6,-0.7,-0.8,-0.9,-1} {
\draw plot[domain=-(pi-\alphr):(pi-\alphr)] ({sqrt(1-(\t)*(\t))*sin(\x r)}, {sqrt(1-(\t)*(\t))*cos(\x r)+\h}, \t);
}



\end{tikzpicture}

}


\fbox{
\begin{tikzpicture}

\tikzmath{\alph = 30; \alphr= \alph*pi/180; \thetr=pi-\alphr; \thet= \thetr*180/pi; \betr=pi/2-\alphr; \bet=\betr*180/pi ; \ray=1;   \rw=\ray*sin(\alphr r); \h=\ray*cos(\alphr r);};

\coordinate (O) at (0,0);
\coordinate (BotRight) at ($(O)+(-\bet:\ray)$);
\coordinate (BotLeft) at ($(O)+(180+\bet:\ray)$);

\shade[ball color = gray!40, opacity = 0.4] (BotRight) arc (-\bet:180+\bet:\ray);
\draw (BotRight) arc (-\bet:180+\bet:\ray);

\draw[densely dashed] (BotRight) arc (0:180:{\rw} and {\rw/3});
\draw (BotLeft) arc (180:360:{\rw} and {\rw/3});



%\draw[dashed] (BotLeft) --++(1,1);
%\draw[dashed] (BotRight) --++(1,1);


%  %\shade[ball color = gray!40, opacity = 0.4] (0,0) circle (2cm);
%  %\draw (0,0) circle (2cm);
%  \draw (-2,0) arc (180:360:2 and 0.6);
%  \draw[dashed] (2,0) arc (0:180:2 and 0.6);
%  \fill[fill=black] (0,0) circle (1pt);
%  \draw[dashed] (0,0 ) -- node[above]{$r$} (2,0);
\end{tikzpicture}
}


\fbox{

\begin{tikzpicture}



\end{tikzpicture}

}


\newpage

\begin{tikzpicture}


\tikzmath{\alph=70; \alphrad=\alph*pi/180; \dalph=10; \dalphrad=\dalph*pi/180; \fV=0.25*(2-cos(\alphrad r)*((1+cos(\alphrad r))^2); \fSp=1-(\alphrad/pi)+(sin((2*\alphrad)r)/(2*pi)); \fspar=1.215*\dalphrad*(sin(\alphrad r)^2*(cos(\dalphrad r)^2)/((pi/2)^2-\dalphrad^2); \fGpar=(1-cos(\alphrad r)^2)*(sin(\dalphrad r));} %geometrical values

\tikzmath{\rholiq=594.3786; \nuliq=1.148*(10^(-7));} %water properties at P=155bar - Tsat




\begin{axis}[
    xmode=log,
    ymode=log,
    log ticks with fixed point,
    % for log axes, x filter operates on LOGS.
    % and log(x * 1000) = log(x) + log(1000):
    %x filter/.code=\pgfmathparse{#1 + 6.90775527898214},
]
\addplot table {
1 10
100 350
10000 2500
};
\end{axis}
\end{tikzpicture}


\fbox{

\begin{tikzpicture}[scale=3.0]

\coordinate (O) at (0,0);
\coordinate (A1) at (1,0);
\coordinate (A2) at (2,0);
\coordinate (A3) at (3,0);
\coordinate (A) at (4,0);


%Sections and wall
\draw (O) -- (A);
\draw ($(O)-(0,0.03)$) -- ($(A)-(0,0.03)$);
\foreach \i in {0,...,19}
{
\draw (\i*0.2,0) -- (\i*0.2+0.05,-0.03);
}



%\draw[dashed, gray!70!white] (A1) --++ (0,1);
%\draw[dashed, gray!70!white] (A2) --++ (0,1);
%\draw[dashed, gray!70!white] (A3) --++ (0,1);



%Flow arrows
\foreach \i in {1,...,12} 
{
\coordinate (Oloc) at ($(O)+(-0.1,\i/13)$);
\draw[->,>=latex, gray!50!blue] (Oloc)--++({ln(1+0.05*\i)},0);
}

\draw ($(Oloc) + ({ln(1+0.05*12)},0)$) node[below right]{${\overline{U_{L}}}$};


%Liquid heat flux

\coordinate (Ophi) at (0.4,0);

\draw[->,>=latex, thick, blue!70!gray] ($(Ophi)+(0,-0.1)$)--($(Ophi)+(0,+0.1)$);
\draw ($(Ophi)+(0,-0.15)$) node{${\phi_{c,L}}$};


%Boiling heat flux
\coordinate (Ob) at (1.0,0);

\tikzmath{\alph = 40; \alphrad= \alph * pi / 180; \ray=0.15; \rayw = sin(\alphrad r) * \ray;}; %Geom values

\coordinate (Oarc) at ($(Ob)+({\ray * sin(\alphrad r)},0)$);
\shade[ball color = gray!40, opacity = 0.4] (Oarc) arc({-(pi/2-\alphrad) r}:{(pi+pi/2-\alphrad) r}:\ray);

\draw (Oarc) arc({-(pi/2-\alphrad) r}:{(pi+pi/2-\alphrad) r}:\ray);


\draw[->,>=latex, thick, brown!80!black] plot [smooth, tension=0.5] coordinates {($(Oarc)+(0,-0.1)$) ($(Oarc)+(0.05,+0.03)$) ($(Oarc)+(-0.05,+0.1)$)};

\coordinate (Oarc2) at ($(Oarc) + (-2*\rayw,0)$);
\draw[->,>=latex, thick, brown!80!black] plot [smooth, tension=0.5] coordinates {($(Oarc2)+(0,-0.1)$) ($(Oarc2)+(-0.05,+0.03)$) ($(Oarc2)+(+0.05,+0.1)$)};

\draw ($(Ob)+(0,-0.15)$) node{${\phi_{e}}$};



%Vapor convective flux
\coordinate (Ob) at (1.75,0);

\tikzmath{\alph = 40; \alphrad= \alph * pi / 180; \ray=0.2; \rw=\ray * sin(\alphrad r);};

\coordinate (Oarc) at ($(Ob)+(\rw,0)$);
\shade[ball color = gray!40, opacity = 0.4] (Oarc) arc({-(pi/2-\alphrad) r}:{(pi+pi/2-\alphrad) r}:\ray);

\draw (Oarc) arc({-(pi/2-\alphrad) r}:{(pi+pi/2-\alphrad) r}:\ray);

\draw[red, thick, densely dashed] (Oarc) --++ (-2*\rw, 0);

\draw[->,>=latex, thick, red!80!black] ($(Ob)+(0,-0.1)$)--($(Ob)+(0,+0.1)$);
\draw ($(Ob)+(0,-0.15)$) node{${\phi_{c,V}}$};



%Quenching heat flux
\coordinate (Ob2) at (2.75,0);

\tikzmath{\alph = 40; \alphrad= \alph * pi / 180;
\dalph=20; \dalphrad=\dalph*pi/180;
\alphadvrad=\alphrad - \dalphrad;
\alphrecrad=\alphrad + \dalphrad;
\ray=0.2; 
\rayadv=\ray *(1+cos(\alphrad r))/(1+ cos(\alphadvrad r);
\rayrec=\ray *(1+cos(\alphrad r))/(1+ cos(\alphrecrad r);};


\coordinate (Cb2) at ($(Ob2)+(0.03,{\ray * cos(\alphrad r)})$);
\draw[green!50!black,->, >=latex] (Cb2)--++({\ray+0.08},0); \draw ($(Cb2)+({\ray+0.08},0)$) node[above]{$\overline{U_{b}}$ }; %Bubble velocity


\coordinate (Oarc) at ($(Ob2)+({\ray * sin(\alphrad r)},0)$);

\shade[ball color = gray!40, opacity = 0.4] (Oarc) arc ({-(pi/2-(\alphadvrad)) r}:{(pi/2) r}:\rayadv) arc ({(pi/2) r}:{(pi+pi/2-(\alphrecrad)) r}:\rayrec);

\draw (Oarc) arc ({-(pi/2-(\alphadvrad)) r}:{(pi/2) r}:\rayadv) arc ({(pi/2) r}:{(pi+pi/2-(\alphrecrad)) r}:\rayrec);

\coordinate (Ob2) at (3.5,0.5);
\tikzmath{\ray=0.25;};

\shade[ball color = gray!40, opacity = 0.4] (Ob2) circle(\ray);
\draw (Ob2) circle(\ray);



\tikzmath{\rayspi=0.07;};

\draw[->,>=stealth,gray!50!blue] plot[domain=0:3.2,smooth,xshift=65,yshift=3] ({(\x *pi) r}:{\rayspi*(1-\x/6)}) ;
\draw[->,>=stealth,gray!50!blue] plot[domain=0:3.2,smooth,xshift=70,yshift=5] ({(\x *pi) r}:{\rayspi*(1-\x/6)}) ;

\coordinate (Ophisl) at ($(Oarc) - (\rayadv+\rayrec+0.05,0)$);
\draw[->,>=latex, thick, orange!90!gray] ($(Ophisl)+(0,-0.1)$)--($(Ophisl)+(0,+0.1)$);
\draw ($(Ophisl)+(0,-0.15)$) node{${\phi_{q,sl}}$};


\tikzmath{\rayspi=0.08;};

\draw[->,>=stealth,gray!50!blue] plot[domain=0:3.2,smooth,xshift=94,yshift=3] ({(\x *pi) r}:{\rayspi*(1-\x/6)}) ;
\draw[->,>=stealth,gray!50!blue] plot[domain=0:3.2,smooth,xshift=105,yshift=3] ({(\x *pi) r}:{\rayspi*(1-\x/6)}) ;


\coordinate (Ophiq) at ($(Ob2) - (0,0.5)$);
\draw[->,>=latex, thick, orange!90!gray] ($(Ophiq)+(0,-0.1)$)--($(Ophiq)+(0,+0.1)$);
\draw ($(Ophiq)+(0,-0.15)$) node{${\phi_{q,lo}}$};

\end{tikzpicture}

}



\fbox{

\begin{tikzpicture}
\draw [red] plot [smooth cycle] coordinates {(0,0) (1,1) (3,1) (1,0) (2,-1)};

\draw [orange] plot [smooth] coordinates {(1,1) (2,2) (4,-2) (3,2)};

\draw [olive] plot [smooth,tension=0.1] coordinates {(1,1) (2,2) (4,-2) (3,2)};

\draw [blue] plot [smooth, tension=2] coordinates {(1,1) (2,2) (4,-2) (3,2)};
\end{tikzpicture}

}

\fbox{

\newcommand{\dalpha}{\text{d}\alpha}

\begin{tikzpicture}[scale=3.0, every node/.style={scale=0.7}]


\coordinate (O) at (0,0);
\coordinate (A2) at (1.4,0);
\coordinate (A) at (3.2,0);


%Sections and wall
\draw (O) -- (A);
\draw ($(O)-(0,0.05)$) -- ($(A)-(0,0.05)$);
\foreach \i in {0,...,15}
{
\draw (\i*0.2,0) -- (\i*0.2+0.1,-0.05);
}

\draw[dashed, gray!70!white] (A2) --++ (0,1);


%Non-tilded bubble
\coordinate (Ob) at (0.75,0);

\tikzmath{\alph = 40; \alphrad= \alph * pi / 180; \ray=0.5; \rw=\ray * sin(\alphrad r);};


\coordinate (Oarc) at ($(Ob)+({\ray * sin(\alphrad r)},0)$);
\draw (Oarc) arc({-(pi/2-\alphrad) r}:{(pi+pi/2-\alphrad) r}:\ray);

%Right angle
\draw (Oarc) --++(\alph:0.3);
\draw ($(Oarc)+(0.15,0)$) arc(0:\alph:0.15) node[near end, right]{$\alpha$};

%Left angle
\coordinate (Oarc2) at ($(Oarc) - (2*\rw,0)$);
\draw (Oarc2) --++(180-\alph:0.3);
\draw ($(Oarc2)+(-0.15,0)$) arc(180:180-\alph:0.15) node[near end, left]{$\alpha$};

%Center and radius

\coordinate (Cb) at ($(Ob)+(0,{\ray*cos(\alphrad r)} )$);
\draw (Cb) node{$\times$} node[above right]{$O$};

\draw[densely dashed, <->, >=latex] (Cb) -- (Oarc2) node[midway, above left]{$R$};



%Tilted bubble
\coordinate (Ob2) at (2.25,0);

\tikzmath{\alph = 40; \alphrad= \alph * pi / 180;
\dalph=20; \dalphrad=\dalph*pi/180;
\alphadvrad=\alphrad - \dalphrad;
\alphrecrad=\alphrad + \dalphrad;
\ray=0.5; 
\rayadv=\ray *(1+cos(\alphrad r))/(1+ cos(\alphadvrad r);
\rayrec=\ray *(1+cos(\alphrad r))/(1+ cos(\alphrecrad r);};

\coordinate (Oarc) at ($(Ob2)+({\ray * sin(\alphrad r)},0)$);


\draw (Oarc) arc({-(pi/2-(\alphadvrad)) r}:{(pi/2) r}:\rayadv) arc ({(pi/2) r}:{(pi+pi/2-(\alphrecrad)) r}:\rayrec);

%Right angle
\draw (Oarc) --++(\alphadvrad r:0.3);
\draw ($(Oarc)+(0.15,0)$) arc(0:\alphadvrad r:0.15) node[very near end, right]{$\alpha - \dalpha$};

%Left angle
\coordinate (Oarc2) at ($(Oarc) - ({\rayadv * sin(\alphadvrad r) + \rayrec * sin(\alphrecrad r)},0)$);
\draw (Oarc2) --++({180-(\alphrecrad r)}:0.3);
\draw ($(Oarc2)+(-0.15,0)$) arc(180:{180-(\alphrecrad r)}:0.15) node[midway, left]{$\alpha + \dalpha $};



%Center and radius

\coordinate (Cb) at ($(Oarc)+( -{0.5 * (\rayadv * sin(\alphadvrad r) + \rayrec  * sin(\alphrecrad r) } , {\ray * cos(\alphrad r)} )$);
\draw (Cb) node{$\times$} node[below right]{$O$};

\draw[densely dashed, <->, >=latex] (Cb) -- (Oarc2) node[midway, above left]{$R$};


%Inclination angle

\draw[densely dotted] (Cb) --++ (0,0.7);
\draw[densely dotted] (Cb) --++ ({90 - (1*\dalphrad r)}: 0.7 );
\draw ($(Cb) + (0,0.3)$) arc(90: {90 - (\dalphrad r)}:0.3)  node[midway, above]{$\dalpha$};



%Flow arrows
\foreach \i in {2,...,14} 
{
\coordinate (Oloc) at ($(A2)+(0.05,\i/15)$);
\draw[->,>=latex, gray!70!blue] (Oloc)--++({ln(1+0.03*\i)},0);
}



\end{tikzpicture}

}




\fbox{

\newcommand{\dalpha}{\text{d}\alpha}

\begin{tikzpicture}[scale=3.0, every node/.style={scale=0.7}]


\coordinate (O) at (0,0);
\coordinate (A2) at (1.4,0);
\coordinate (A) at (2,0);


%Sections and wall
\draw (O) -- (A);
\draw ($(O)-(0,0.05)$) -- ($(A)-(0,0.05)$);
\foreach \i in {0,...,9}
{
\draw (\i*0.2,0) -- (\i*0.2+0.1,-0.05);
}

%\draw[dashed, gray!70!white] (A2) --++ (0,1);


%Non-tilded bubble
\coordinate (Ob) at (1.3,0);

\tikzmath{\alph = 40; \alphrad= \alph * pi / 180; \ray=0.5; \rw=\ray * sin(\alphrad r);};


\coordinate (Oarc) at ($(Ob)+({\ray * sin(\alphrad r)},0)$);
\draw (Oarc) arc({-(pi/2-\alphrad) r}:{(pi+pi/2-\alphrad) r}:\ray);

%%Right angle
\draw (Oarc) --++(\alph:0.3);
\draw ($(Oarc)+(0.15,0)$) arc(0:\alph:0.15) node[near end, right]{$\alpha$};

%%Left angle
%\coordinate (Oarc2) at ($(Oarc) - (2*\rw,0)$);
%\draw (Oarc2) --++(180-\alph:0.3);
%\draw ($(Oarc2)+(-0.15,0)$) arc(180:180-\alph:0.15) node[near end, left]{$\alpha$};

%Center and radius

\coordinate (Cb) at ($(Ob)+(0,{\ray*cos(\alphrad r)} )$);
\draw (Cb) node{$\times$} node[above right]{$O$};

\draw[densely dashed, <->, >=latex] (Cb) -- (Oarc) node[midway, above right]{$R$};




%Flow arrows
\foreach \i in {2,...,14} 
{
\coordinate (Oloc) at ($(O)+(-0.1,\i/15)$);
\draw[->,>=latex, gray!70!blue] (Oloc)--++({ln(1+0.03*\i)},0);
}

%Liquid flow

\coordinate (TopB) at ($(Cb)+(0,{\ray+0.05})$);
\coordinate (TopLeftB) at ($(Cb) + ({90+\alph}:{\ray+0.05})$);
\coordinate (LeftB) at ($(Cb)+({-\ray-0.05},0)$);

%\draw (TopB) -- (LeftB);

\draw [->, >=latex, blue!70!black] plot [smooth, tension=1.0] coordinates {(TopB) (TopLeftB) (LeftB)}; \draw (TopB) node[above]{$T_{l}(y^{+}_{b})$};

\coordinate (Sp1) at (0.5,{\ray/2});
\coordinate (Sp2) at (0.75,{\ray/2});

\tikzmath{\rayspi=0.1;};

\draw[->,>=stealth,blue!70!black] plot[domain=0:3.2,smooth,xshift=9,yshift=3] ({(\x *pi) r}:{\rayspi*(1-\x/6)}) ;
\draw[->,>=stealth,blue!70!black] plot[domain=0:3.2,smooth,xshift=17,yshift=8] ({(\x *pi) r}:{\rayspi*(1-\x/6)}) ;

%\draw [orange] plot [smooth] coordinates {(1,1) (2,2) (4,-2) (3,2)};

%Bubble velocity

\draw[red,->, >=latex] (Cb)--++({\ray+0.07},0) node[above]{ $\overline{U_{b}}$ };


\end{tikzpicture}

}

\fbox{

\begin{tikzpicture}[scale=4.0]

\tikzmath{\rayspi=0.2;};

\coordinate (Cspi) at (0,0);
\coordinate (Dspi) at ($(Cspi)+(0,\rayspi)$);

\draw[->] plot[domain=0:3,smooth] ({(\x *pi) r}:{\rayspi*(1-\x/4)}) ;
%{
%\draw plot[domain=-(pi-\alphr):(pi-\alphr)] ({sqrt(1-(\t)*(\t))*sin(\x r)}, {sqrt(1-(\t)*(\t))*cos(\x r)+\h}, \t);
%}



\end{tikzpicture}
}

\fbox{

\begin{tikzpicture}[scale=2.0]

\tikzmath{\length=3; \diam=0.6;};

\coordinate (A1) at (0,0);
\coordinate (A2) at ($(A1) + (0,\length)$);
\coordinate (B1) at (\diam,0);
\coordinate (B2) at ($(B1)+(0,\length)$);

\draw (A1) -- (A2);
\draw (B1) -- (B2);

\draw [->, >=latex, blue!70!black] ($(A1)+({\diam / 2},0)$) --++ (0,0.4) node[very near start, below right]{$U_{e}$};


\foreach \z in {1,...,4}
{
\coordinate (P1) at (0,{\z * \length / 5});
\coordinate (P2) at (\diam, {\z * \length / 5});

\draw[->, >=latex, red!80!black] ($(P1)+(-0.4,0)$) -- (P1);
\draw[->, >=latex, red!80!black] ($(P2)+(0.4,0)$) -- (P2);

}


\draw[red!80!black] (0,{\length / 2}) node[left]{$\phi_{w}$};
\draw[red!80!black] (\diam,{\length / 2}) node[right]{$\phi_{w}$};

\draw[orange!50!black, densely dotted] (P1)--(P2) node[very near end, above right]{\footnotesize{~Mesures}};

\draw[green!50!black, densely dashed, <->,>=latex] ($(B1)+(0.5,0)$) -- ($(P2) + (0.5,0)$) node[midway,right]{$L_{heat}$};

\tikzmath{\bubrad=0.03;};
\foreach \i in {1,...,10}
{

\draw ({\bubrad+rnd*(\diam-2*\bubrad)}, {\length / 2 + rnd*\length /2}) circle(\bubrad);

\draw ({\bubrad+rnd*(\diam-2*\bubrad)}, {\length / 2 + rnd*\length /2}) circle(\bubrad);

\draw ({\bubrad+rnd*(\diam-2*\bubrad)}, {\length / 2 + rnd*\length /2}) circle(\bubrad);


}


\end{tikzpicture}


}



\fbox{

\begin{tikzpicture}

\node[draw] (phiw) at (0,0){$\phi_{w}$};

%HFP
\node[draw] (phicl) at (2,2){$\phi_{c,L}$};
\node[draw] (phicv) at (2,1){$\phi_{c,V}$};
\node[draw] (phiq) at (2,-1){$\phi_{q}$};
\node[draw] (phie) at (2,-2){$\phi_{e}$};

\draw[<-, >=latex] (phiw) to (phicl);
\draw[<-, >=latex] (phiw) to (phicv);
\draw[<-, >=latex] (phiw) to (phiq);
\draw[<-, >=latex] (phiw) to (phie);

%phicl
\node[draw] (acl) at (,2){$\phi_{c,L}$};
\node[draw] (hcl) at (2,1){$\phi_{c,V}$};
\node[draw] (Tliq) at (2,-1){$\phi_{q}$};
\node[draw] (Tw) at (2,-2){$\phi_{e}$};




\end{tikzpicture}

}







\fbox{


\begin{tikzpicture}[scale=3.0, every node/.style={scale=0.7}]


%%%Truncated sphere on a vertical wall

\coordinate (O1) at (0,0);
\coordinate (O2) at (0,2);

\draw (O1)--(O2);

\coordinate (Ob) at (0,1.0);

\tikzmath{\thet = 45; \thetrad= \thet * pi / 180; \ray=0.5; \rw=\ray * sin(\thetrad r);};


\coordinate (Oarc) at ($(Ob)-(0,{\ray * sin(\thetrad r)})$);
\draw (Oarc) arc({(-pi+\thetrad ) r}:{(pi-\thetrad ) r}:\ray);

%Upstream angle
\draw (Oarc) --++(-90+\thet:0.3);
\draw ($(Oarc)+(0,-0.15)$) arc(-90:-90+\thet:0.15) node[near end, below]{$\theta$};

%%Downstream angle
%\coordinate (Oarc2) at ($(Oarc) - (2*\rw,0)$);
%\draw (Oarc2) --++(180-\thet:0.3);
%\draw ($(Oarc2)+(-0.15,0)$) arc(180:180-\thet:0.15) node[near end, left]{$\alpha$};

%Center and radius

\coordinate (Cb) at ($(Ob)+({\ray*cos(\thetrad r)},0 )$);
\draw (Cb) node{$\times$} node[below right]{$O$};

\draw[densely dashed, <->, >=latex] (Cb) -- (Oarc) node[midway, below right]{$R$};


%%Forces

\draw[->, >=latex, violet!70!black] (Ob)--++(\ray/2,0) node[near end, above]{$\vect{F_{CP}}$};

\draw[->, >=latex, red!70!black!] ($(Cb)+(\ray,0)$)--++(\ray/2,0) node[very near end, above]{$\vect{F_{L}}$};
\draw[->, >=latex, red!70!black!] ($(Cb)+(0,\ray)$)--++(0,\ray/2) node[very near end, right]{$\vect{F_{D}}$};

\coordinate (Oarc2) at ($(Oarc)+(0,{2*\rw})$);
\draw[->, >=latex, violet] (Oarc)--++(90+\thet:\ray/2) node[very near end, above]{$\vect{F_{C}}$};
\draw[->, >=latex, violet] (Oarc2)--++(-90-\thet:\ray/2) node[very near end, below]{$\vect{F_{C}}$};

\draw[->, >=latex, blue!70!black] (Cb)--++(0,\ray/1.5) node[very near end, right]{$\vect{F_{B}}$};


\draw[->, >=latex, green!50!black!] (Cb)--++(90+\thet/1.5:\ray/1.5) node[very near end, above]{$\vect{F_{AM}}$};

%Gravity

\draw[->, >=latex, blue!30!black]  ($(Cb)+({1.5*\ray},{1.5*\ray})$)--++(0,-\ray/2) node[very near end, right]{$\vect{g}$};


%Flow arrows
\foreach \i in {2,...,14} 
{
\coordinate (Oloc) at ($(O1)+(\i/15,0.05)$);
\draw[->,>=latex, gray!70!blue] (Oloc)--++(0,{ln(1+0.03*\i)});
}





%Tilted bubble
\coordinate (Ob2) at (2.5,1.0);

\coordinate (O1) at (2.5,0);
\coordinate (O2) at (2.5,2);
\draw (O1)--(O2);

\tikzmath{\thet = 40; \thetrad= \thet * pi / 180;
\dthet=10; \dthetrad=\dthet*pi/180;
\thetadvrad=\thetrad - \dthetrad;
\thetrecrad=\thetrad + \dthetrad;
\thetadv=\thetadvrad*180/pi;
\thetrec=\thetrecrad*180/pi;
\ray=0.5; 
\rayadv=\ray *(1+cos(\thetrad r))/(1+ cos(\thetadvrad r);
\rayrec=\ray *(1+cos(\thetrad r))/(1+ cos(\thetrecrad r);};

\coordinate (Oarc) at ($(Ob2)-(0,{\ray * sin(\thetrad r)})$);


\draw (Oarc) arc({-pi+(\thetrecrad)) r}:{0 r}:\rayrec) arc ({0 r}:{pi-(\thetadvrad)) r}:\rayadv);

%Upstream angle
\draw (Oarc) --++(-90+\thetrecrad r:0.3) node[very near end, below]{$\theta + \dtheta$};
\draw ($(Oarc)+(0,-0.15)$) arc(-90:-90+\thetrecrad r:0.15) ;

%Downstream angle
\coordinate (Oarc2) at ($(Oarc) + (0,{\rayadv * sin(\thetadvrad r) + \rayrec * sin(\thetrecrad r)})$);
\draw (Oarc2) --++({90-(\thetadvrad r)}:0.3);
\coordinate (angadv) at ($(Oarc2) +({90-(\thetadvrad r)}:0.3)$);
\draw (angadv)  node[above]{$\theta - \dtheta $};
\draw ($(Oarc2)+(0,+0.15)$) arc(90:{90-(\thetadvrad r)}:0.15);



%Center and radius

\coordinate (Cb) at ($(Oarc2)+( {\ray * cos(\thetrad r)} , {-0.5 * (\rayadv * sin(\thetadvrad r) + \rayrec  * sin(\thetrecrad r) )})$);
\draw (Cb) node{$\times$} node[below right]{$O$};

\draw[densely dashed, <->, >=latex] (Cb) -- (Oarc) node[midway, above left]{$R$};


%Inclination angle

%\draw[densely dotted] (Cb) --++ (0.7,0);
%\draw[densely dotted] (Cb) --++ ({(1*\dthetrad r)}: 0.7 );
%\draw ($(Cb) + (0.3,0)$) arc(0: {(\dthetrad r)}:0.3)  node[midway, right]{$\dtheta$};
%
%



%%Forces

\draw[->, >=latex, violet!70!black] (Ob2)--++(\ray/2,0) node[near end, above]{$\vect{F_{CP}}$};

\draw[->, >=latex, red!70!black!] ($(Cb)+(\ray,0)$)--++(\ray/2,0) node[very near end, above]{$\vect{F_{L}}$};
\draw[->, >=latex, red!70!black!] ($(Cb)+(0,-\rayadv*0.95)$)--++(0,-\ray/2) node[very near end, right]{$\vect{F_{D}}$};
\draw[->, >=latex, red] ($(Cb)+(0,+\rayrec*1.04)$)--++(0,+\ray/2) node[very near end, right]{$\vect{U_{b}}$};


\draw[->, >=latex, violet] (Oarc)--++(90+\thetrec:\ray/2) node[very near end, above]{$\vect{F_{C}}$};
\draw[->, >=latex, violet] (Oarc2)--++(-90-\thetadv:\ray/2) node[very near end, below]{$\vect{F_{C}}$};

\draw[->, >=latex, blue!70!black] (Cb)--++(0,\ray/1.5) node[very near end, right]{$\vect{F_{B}}$};


\draw[->, >=latex, green!50!black!] (Cb)--++(90-\thet:-\ray/1.5) node[very near end, right]{$\vect{F_{AM}}$};

%Gravity

\draw[->, >=latex, blue!30!black]  ($(Cb)+({1.5*\ray},{1.5*\ray})$)--++(0,-\ray/2) node[very near end, right]{$\vect{g}$};



%Flow arrows
\foreach \i in {2,...,14} 
{
\coordinate (Oloc) at ($(O1)+(\i/15,0.05)$);
\draw[->,>=latex, gray!70!blue] (Oloc)--++(0,{ln(1+0.03*\i)});
}



\end{tikzpicture}

}




\fbox{

\begin{tikzpicture}[scale=3.0, every node/.style={scale=0.7}]


\coordinate (O) at (0,0);
\coordinate (A2) at (1.6,0);
\coordinate (A) at (3.2,0);


%Sections and wall
\draw (O) -- (A2);
\draw ($(O)-(0,0.05)$) -- ($(A2)-(0,0.05)$);
\foreach \i in {0,...,7}
{
\draw (\i*0.2,0) -- (\i*0.2+0.1,-0.05);
}


%Non-tilded bubble
\coordinate (Ob) at (0.75,0);

\tikzmath{\alph = 0; \alphrad= \alph * pi / 180; \ray=0.5; \rw=\ray * sin(\alphrad r);};


\coordinate (Oarc) at ($(Ob)+({\ray * sin(\alphrad r)},0)$);
\draw (Oarc) arc({-(pi/2-\alphrad) r}:{(pi+pi/2-\alphrad) r}:\ray);

\tikzmath{\raybis= 0.8*\ray;}
\coordinate (Oarc) at ($(Ob)+({\raybis * sin(\alphrad r)},0)$);
\draw[loosely dashed] (Oarc) arc({-(pi/2-\alphrad) r}:{(pi+pi/2-\alphrad) r}:\raybis);


\coordinate (Oarc2) at ($(Oarc) - (2*\rw,0)$);

%Center and radius

\coordinate (Cb) at ($(Ob)+(0,{\ray*cos(\alphrad r)} )$);
\draw (Cb) node{$\times$} node[above right]{$O$};

\draw[densely dashed, <->, >=latex] (Cb) -- (Oarc2) node[midway, above left]{$R$};


%Flow arrows
\foreach \i in {2,...,14} 
{
\coordinate (Oloc) at ($(O)+(-0.1,\i/15)$);
\draw[->,>=latex, gray!70!blue] (Oloc)--++({ln(1+0.03*10)},0);
}

\draw ($(Oloc) + ({ln(1+0.05*7)},0)$) node[below right]{${\overline{U_{L}}}$};

\draw[red, thick, ->, >=stealth] (Cb)--++(0.35*\ray,-0.65*\ray) node[midway, right] {$\redmath{\vect{F_{AM}}}$};

\end{tikzpicture}

}





\fbox{

\begin{tikzpicture}[scale=3.0, every node/.style={scale=0.7}]


\coordinate (O) at (0,0);
\coordinate (A2) at (1.6,0);
\coordinate (A) at (3.2,0);


%Sections and wall
\draw[dotted] (O) -- (A2);


%Non-tilded bubble
\coordinate (Ob) at (0.75,0);

\tikzmath{\alph = 90; \alphrad= \alph * pi / 180; \ray=0.5; \rw=\ray * sin(\alphrad r);};


\tikzmath{\raybis= 0.8*\ray;}
\coordinate (Oarc) at ($(Ob)+({\raybis * sin(\alphrad r)},0)$);
\draw[loosely dashed] (Oarc) arc({-(pi/2-\alphrad) r}:{(pi+pi/2-\alphrad) r}:\raybis);

\coordinate (Oarc) at ($(Ob)+({\ray * sin(\alphrad r)},0)$);
\draw (Oarc) arc({-(pi/2-\alphrad) r}:{(pi+pi/2-\alphrad) r}:\ray);




\coordinate (Oarc2) at ($(Oarc) - (2*\rw,0)$);

%Center and radius

\coordinate (Cb) at ($(Ob)+(0,{\ray*cos(\alphrad r)} )$);
\draw (Cb) node{$\times$} node[above right]{$O$};

\draw[densely dashed, <->, >=latex] (Cb) --++($({\ray*sin(-45)},{\ray*cos(-45)})$) node[midway, below left]{$R$};

\draw (Oarc)--++(-2*\ray,0);

\draw[red, thick, ->, >=stealth] ($(Cb)+(0,\ray)$)--++(0,-0.75*\ray) node[midway, right] {$\redmath{\vect{F_{AM}}}$};


\end{tikzpicture}

}





%\beginpgfgraphicnamed{forces} %%IN COMMAND LINE : pdflatex --jobname=forces draw.tex


\fbox{


\begin{tikzpicture}[scale=3.5, every node/.style={scale=0.9}]


%%%%Truncated sphere on a vertical wall


%Tilted bubble
\coordinate (Ob2) at (2.5,1.0);

\coordinate (O1) at (2.5,0);
\coordinate (O2) at (2.5,2);
\draw (O1)--(O2);

\tikzmath{\thet = 40; \thetrad= \thet * pi / 180;
\dthet=10; \dthetrad=\dthet*pi/180;
\thetadvrad=\thetrad - \dthetrad;
\thetrecrad=\thetrad + \dthetrad;
\thetadv=\thetadvrad*180/pi;
\thetrec=\thetrecrad*180/pi;
\ray=0.5; 
\rayadv=\ray *(1+cos(\thetrad r))/(1+ cos(\thetadvrad r);
\rayrec=\ray *(1+cos(\thetrad r))/(1+ cos(\thetrecrad r);};

\coordinate (Oarc) at ($(Ob2)-(0,{\ray * sin(\thetrad r)})$);


\draw (Oarc) arc({-pi+(\thetrecrad)) r}:{0 r}:\rayrec) arc ({0 r}:{pi-(\thetadvrad)) r}:\rayadv);

%Upstream angle
\draw (Oarc) --++(-90+\thetrecrad r:0.3) node[very near end, below]{$\theta + \dtheta$};
\draw ($(Oarc)+(0,-0.15)$) arc(-90:-90+\thetrecrad r:0.15) ;

%Downstream angle
\coordinate (Oarc2) at ($(Oarc) + (0,{\rayadv * sin(\thetadvrad r) + \rayrec * sin(\thetrecrad r)})$);
\draw (Oarc2) --++({90-(\thetadvrad r)}:0.3);
\coordinate (angadv) at ($(Oarc2) +({90-(\thetadvrad r)}:0.3)$);
\draw (angadv)  node[above]{$\theta - \dtheta $};
\draw ($(Oarc2)+(0,+0.15)$) arc(90:{90-(\thetadvrad r)}:0.15);



%Center and radius

\coordinate (Cb) at ($(Oarc2)+( {\ray * cos(\thetrad r)} , {-0.5 * (\rayadv * sin(\thetadvrad r) + \rayrec  * sin(\thetrecrad r) )})$);
\draw (Cb) node{$\times$} node[below right]{$O$};

\draw[densely dashed, <->, >=latex] (Cb) -- (Oarc) node[midway, above left]{$R$};


\draw[densely dashed, <->, >=latex] ($(Oarc) + (-\ray/12,+\ray/15) $) -- ($(Oarc2) + (-\ray/12,-\ray/6) $) node[midway, left]{$d_{w}$};



%Inclination angle

\draw[densely dotted] (Cb) --++ (\ray,0);
\draw[densely dotted] (Cb) --++ ({(1*\dthetrad r)}: \ray );
\draw ($(Cb) + (0.3,0)$) arc(0: {(\dthetrad r)}:0.3)  node[very near end, above]{$\dtheta$};





%%Forces

\draw[->, >=latex, violet!70!black] (Ob2)--++(\ray/2,0) node[near end, above]{$\vect{F_{CP}}$};

\draw[->, >=latex, red!70!black!] ($(Cb)+(\ray,0)$)--++(\ray/2,0) node[very near end, above]{$\vect{F_{L}}$};
\draw[->, >=latex, red!70!black!] ($(Cb)+({sin(5)},{cos(5)*\rayrec*1.04})$)--++(0,+\ray/2) node[very near end, right]{$\vect{F_{D}}$};
\draw[->, >=latex, red] ($(Cb)+(0,+\rayrec*1.04)$)--++(0,+\ray/1.5) node[very near end, left]{$\vect{U_{b}}$};


\draw[->, >=latex, violet] (Oarc)--++(90+\thetrec:\ray/2) node[very near end, left]{$\vect{F_{C}}$};
\draw[->, >=latex, violet] (Oarc2)--++(-90-\thetadv:\ray/2) node[very near end, left]{$\vect{F_{C}}$};

\draw[->, >=latex, blue!70!black] (Cb)--++(0,\ray/1.5) node[very near end, right]{$\vect{F_{B}}$};


\draw[->, >=latex, green!50!black!] (Cb)--++(160-\thet:\ray/1.5) node[very near end, above]{$\vect{F_{AM}}$};

%Gravity

\draw[->, >=latex, blue!30!black]  ($(Cb)+({1.5*\ray},{1.5*\ray})$)--++(0,-\ray/2) node[very near end, right]{$\vect{g}$};



%Flow arrows
\foreach \i in {2,...,14} 
{
\coordinate (Oloc) at ($(O1)+(\i/15,0.05)$);
\draw[->,>=latex, gray!70!blue] (Oloc)--++(0,{ln(1+0.03*\i)});
}
\draw[gray!70!blue] ($(Oloc)+(0.1,0.1)$) node{$\vect{U_{L}}$};

%%Referential vectors
\coordinate (Ovect) at (4.0,0);
\draw[->, >=latex] (Ovect)--++(0.3,0) node[very near end, above right]{$\vect{e_{y}}$};
\draw[->, >=latex] (Ovect)--++(0,0.3) node[very near end, above right]{$\vect{e_{x}}$};



\end{tikzpicture}

}


%\endpgfgraphicnamed




%\beginpgfgraphicnamed{shi_scheme}

\fbox{

\begin{tikzpicture}[scale=3.0, every node/.style={scale=0.7}]


\coordinate (O) at (0,0);
\coordinate (A2) at (2.0,0);
\coordinate (A) at (3.2,0);


%Sections and wall
\draw (O) -- (A2);
\draw ($(O)-(0,0.05)$) -- ($(A2)-(0,0.05)$);
\foreach \i in {0,...,9}
{
\draw (\i*0.2,0) -- (\i*0.2+0.1,-0.05);
}


%Parameters
\coordinate (Ob) at (1.2,0.6);

\coordinate (Ow) at (1.2,0);


\tikzmath{\alph = 0; \alphrad= \alph * pi / 180; \ray=0.5; \rw=\ray * sin(\alphrad r);};

%Viscous influence zone

\draw[dashed, black, very thick] (Ob) circle(\ray);

\fill[color = blue!20!white, opacity=0.5] (Ob) circle(\ray);

%bubble

\draw[black, very thick] (Ob) circle({\ray/1.5});

\fill[white] (Ob) circle({\ray/1.5});

\draw (Ob) node{$\times$} ; % node[above right]{$O$};

\draw[densely dashed, <->, >=latex] (Ob) --++ (0,+\ray/1.5) node[midway, left]{$R$};

\draw[densely dashed, <->, >=latex] (Ob) -- (Ow) node[near end, right]{$y$};


\draw[densely dashed, <->, >=latex] (Ob) --++ (-\ray,0) node[left]{$\dfrac{\nu_{L}}{\left| U_{rel} \right|}$};




%Center and radius


%Flow arrows
\foreach \i in {2,...,14} 
{
\coordinate (Oloc) at ($(O)+(-0.1,\i/15)$);
\draw[->,>=latex, gray!70!blue] (Oloc)--++({ln(1+0.03*\i)},0);
}



\draw[gray!70!blue] ($(Oloc)+(0.1,0.1)$) node{$\vect{U_{L}}$};




\end{tikzpicture}

}


%\endpgfgraphicnamed


%\beginpgfgraphicnamed{newbub_sugrue}


\fbox{


\begin{tikzpicture}[scale=3.5, every node/.style={scale=0.9}]


%%%%Truncated sphere on a vertical wall


%Tilted bubble
\coordinate (Ob2) at (2.5,1.0);

\coordinate (O1) at (2.5,0);
\coordinate (O2) at (2.5,2);
\draw (O1)--(O2);

\tikzmath{\thet = 30 ; \thetrad= \thet * pi / 180;
\dthet= 25 ; \dthetrad=\dthet*pi/180;
\thetadvrad=\thetrad - \dthetrad;
\thetrecrad=\thetrad + \dthetrad;
\thetadv=\thetadvrad*180/pi;
\thetrec=\thetrecrad*180/pi;
\ray=0.5; 
\rayadv=\ray *(1+cos(\thetrad r))/(1+ cos(\thetadvrad r);
\rayrec=\ray *(1+cos(\thetrad r))/(1+ cos(\thetrecrad r);};

\coordinate (Oarc) at ($(Ob2)-(0,{\ray * sin(\thetrad r)})$);


\draw (Oarc) arc({-pi+(\thetrecrad)) r}:{0 r}:\rayrec) arc ({0 r}:{pi-(\thetadvrad)) r}:\rayadv);




\end{tikzpicture}

}


\beginpgfgraphicnamed{wettability}

\fbox{

\begin{tikzpicture}[scale=2.5]
\coordinate (O) at (0,0);
\coordinate (Oliq) at (0,0.1);
\coordinate (cavL) at ($(O)+(0.5,0)$);
\coordinate (cavC) at ($(cavL)+(-45:0.5)$);
\coordinate (cavR) at ($(cavC)+(+45:0.5)$);
\coordinate (Oend) at ($(cavR)+(0.5,0)$);

%coordinates {(1,1) (2,2) (4,-2) (3,2)}


\draw[thick] (O)--(cavL)--(cavC)--(cavR)--(Oend)--++(1.5,0);
\draw[blue] plot [smooth, tension=0.3] coordinates {(Oliq) ($(cavL)+(0,0.1)$) ($(cavC)+(0,0.1)$) ($(cavR)+(0,0.1)$) ($(Oend)+(0,0.1)$)};

\coordinate (Ob) at ($(O) + (2.5,0)$);

\tikzmath{\alph = 20; \alphrad= \alph * pi / 180; \ray=0.6; \rw=\ray * sin(\alphrad r);};

\coordinate (Oarc) at ($(Ob)+({\ray * sin(\alphrad r)},0)$);
\draw (Oarc) arc({-(pi/2-\alphrad) r}:{(pi+pi/2-\alphrad) r}:\ray);

%%Right angle
\draw (Oarc) --++(\alph:0.3) node[very near end, right]{$\theta$};
\draw ($(Oarc)+(0.15,0)$) arc(0:\alph:0.15) ;

\draw ($(cavC) + (0,1.2)$) node{High wettability, low $\theta$};

\coordinate (sep) at ($(Oend)+(1.5,0)+(0.05,0)$);
\draw[dashed] (sep)--++(0,2);




\coordinate (O) at ($(sep)+(0.05,0)$);
\coordinate (Oliq) at ($(O)+(0,0.1)$);
\coordinate (cavL) at ($(O)+(0.5,0)$);
\coordinate (cavC) at ($(cavL)+(-45:0.5)$);
\coordinate (cavR) at ($(cavC)+(+45:0.5)$);
\coordinate (Oend) at ($(cavR)+(0.5,0)$);

%coordinates {(1,1) (2,2) (4,-2) (3,2)}

\draw[thick] (O)--(cavL)--(cavC)--(cavR)--(Oend)--++(1.6,0);
\draw[blue] plot [smooth, tension=0.3] coordinates {(Oliq) ($(cavL)+(0,0.1)$) ($(cavC)+(0,0.35)$) ($(cavR)+(0,0.1)$) ($(Oend)+(0,0.1)$)};

\coordinate (Ob) at ($(O) + (2.6,0)$);

\tikzmath{\alph = 120; \alphrad= \alph * pi / 180; \ray=0.6; \rw=\ray * sin(\alphrad r);};

\coordinate (Oarc) at ($(Ob)+({\ray * sin(\alphrad r)},0)$);
\draw (Oarc) arc({-(pi/2-\alphrad) r}:{(pi+pi/2-\alphrad) r}:\ray);

%%Right angle
\draw (Oarc) --++(\alph:0.3) node[very near end, right]{$\theta$};
\draw ($(Oarc)+(0.15,0)$) arc(0:\alph:0.15) ;

\draw ($(cavC) + (0,1.2)$) node{Low wettability, high $\theta$};

\end{tikzpicture}

}

\endpgfgraphicnamed

\end{document}